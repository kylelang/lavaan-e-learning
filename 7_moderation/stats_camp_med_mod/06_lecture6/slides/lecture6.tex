\documentclass{beamer}
\usetheme{ttuStatsCamp}
\usefonttheme{serif}
\usepackage[T1]{fontenc}
\usepackage[utf8]{inputenc}
\usepackage{url}
\usepackage{graphicx}
\usepackage{setspace}
\usepackage[natbibapa]{apacite}
\usepackage{color}
\usepackage{amsmath}
\usepackage{amsfonts}
\usepackage{Sweavel}
\usepackage{listings}

\def\Sweavesize{\scriptsize}
\def\Rcolor{\color{black}}
%\def\Routcolor{\color{red}}
\def\Rcommentcolor{\color{violet}}
\def\Rbackground{\color[gray]{0.85}}
\def\Routbackground{\color[gray]{0.85}}

\lstset{tabsize=2, breaklines=true, style=Rstyle}



\newcommand{\red}[0]{\textcolor{red}}
\newcommand{\green}[0]{\textcolor{green}}
\newcommand{\blue}[0]{\textcolor{blue}}
\newcommand{\comment}[1]{}
\newcommand{\va}[0]{\vspace{12pt}}
\newcommand{\vb}[0]{\vspace{6pt}}
\newcommand{\vc}[0]{\vspace{3pt}}
\newcommand{\vx}[1]{\vspace{#1pt}}

\title[Lecture 6]{Lecture 6: Longitudinal Mediation}

\author{Kyle M. Lang}

\institute[TTU IMMAP]{
  Institute for Measurement, Methodology, Analysis \& Policy\\
  Texas Tech University\\
  Lubbock, TX
}

\date{2016 Stats Camp}

\setbeamertemplate{frametitle continuation}{}

\begin{document}

\setkeys{Gin}{width=\textwidth}

\input{sweaveFiles/-001}



\begin{frame}[plain]

  \titlepage
  
\end{frame}



\begin{frame}{Outline}

  \begin{itemize}
  \item Show how to test indirect effects with longitudinal models
    \va
  \item Briefly discuss causal inference
    \va
  \item Discuss how to make \emph{more} causal conclusions about
    mediation
  \end{itemize}

\end{frame}



\begin{frame}{Simplest Longitudinal Model}

  \begin{figure}
    \includegraphics[width=\textwidth]{figures/twoWaveStructure.pdf}
  \end{figure}

\end{frame}



\begin{frame}{Better Longitudinal Model}

  \begin{figure}
    \includegraphics[width=\textwidth]{figures/threeWaveStructure.pdf}
  \end{figure}

\end{frame}


\begin{frame}{With Experimental Manipulation}

  \begin{figure}
    \includegraphics[width=\textwidth]{figures/threeWaveExpStructure.pdf}
  \end{figure}

\end{frame}



\begin{frame}{With Mediated Paths}

  \begin{figure}
    \includegraphics[width=\textwidth]{figures/twoWaveStructure2.pdf}
  \end{figure}

\end{frame}



\begin{frame}{With Mediated Paths}

  \begin{figure}
    \includegraphics[width=\textwidth]{figures/threeWaveStructure2.pdf}
  \end{figure}

\end{frame}


\begin{frame}{Alternative c' Path}

  \begin{figure}
    \includegraphics[width=\textwidth]{figures/threeWaveStructure3.pdf}
  \end{figure}

\end{frame}



\begin{frame}{Experimental Model}

  \begin{figure}
    \includegraphics[width=\textwidth]{figures/threeWaveExpStructure2.pdf}
  \end{figure}

\end{frame}



\begin{frame}{Let's Try It}

  \begin{figure}
    \includegraphics[width=\textwidth]{figures/examplePathDiagram.pdf}
  \end{figure}

\end{frame}


\begin{frame}[allowframebreaks]{Example}

\begin{Schunk}
\begin{Sinput}
 dat1 <- readRDS("../data/adamsKlpsScaleScore.rds")
 ## Partial out the mediator's effect:
 mod1 <- lm(policy ~ sysRac + polAffil, data = dat1)
 mod2 <- lm(sysRac ~ polAffil, data = dat1)
 summary(mod1)$coef
\end{Sinput}
\begin{Soutput}
              Estimate Std. Error   t value     Pr(>|t|)
(Intercept) 0.83265885 0.41246491 2.0187386 4.670428e-02
sysRac      0.72235878 0.11147514 6.4799987 5.930291e-09
polAffil    0.05121251 0.06998433 0.7317711 4.663450e-01
\end{Soutput}
\begin{Sinput}
 summary(mod2)$coef
\end{Sinput}
\begin{Soutput}
             Estimate Std. Error  t value     Pr(>|t|)
(Intercept) 2.6060451 0.28489391 9.147423 2.715546e-14
polAffil    0.2568494 0.06213495 4.133735 8.336022e-05
\end{Soutput}
\begin{Sinput}
 ## Extract important parameter estimates:
 a <- coef(mod2)["polAffil"]
 b <- coef(mod1)["sysRac"]
 ## Compute indirect effect:
 ieProd <- a * b
 ieProd
\end{Sinput}
\begin{Soutput}
 polAffil 
0.1855374 
\end{Soutput}
\begin{Sinput}
 ## Calculate Sobel's Z:
 seA <- sqrt(diag(vcov(mod2)))["polAffil"]
 seB <- sqrt(diag(vcov(mod1)))["sysRac"]
 sobelSE <- sqrt(b^2 * seA^2 + a^2 * seB^2)
 sobelZ <- ieProd / sobelSE
 sobelZ
\end{Sinput}
\begin{Soutput}
polAffil 
 3.48501 
\end{Soutput}
\begin{Sinput}
 sobelP <- 2 * pnorm(sobelZ, lower = FALSE)
 sobelP
\end{Sinput}
\begin{Soutput}
    polAffil 
0.0004921178 
\end{Soutput}
\begin{Sinput}
 sobelUB <- ieProd + 1.96 * sobelSE
 sobelLB <- ieProd - 1.96 * sobelSE
 ## 95% Sobel CI:
 c(sobelLB, sobelUB)
\end{Sinput}
\begin{Soutput}
  polAffil   polAffil 
0.08118957 0.28988525 
\end{Soutput}
\end{Schunk}


\end{frame}


\begin{frame}[allowframebreaks]{Example}
  
\begin{Schunk}
\begin{Sinput}
 parameterEstimates(out1, 
                    boot = "bca.simple")[-c(6 : 13), -c(1 : 3)]
\end{Sinput}
\begin{Soutput}
   label   est    se     z pvalue ci.lower ci.upper
1     cp 0.082 0.259 0.318  0.751   -0.408    0.641
2      b 1.390 0.215 6.465  0.000    0.939    1.786
3     a1 0.729 0.091 8.014  0.000    0.539    0.908
4     a2 0.641 0.089 7.183  0.000    0.469    0.818
5     a3 0.451 0.097 4.643  0.000    0.223    0.629
14   imm 0.628 0.169 3.706  0.000    0.296    0.980
\end{Soutput}
\end{Schunk}


\end{frame}



\begin{frame}{Assumptions}
  
  We need a few assumptions to fully generalize our findings:
  \vb
  \begin{itemize} 
  \item \textsc{Stability:} Do mean levels follow a stable trend?  
    \vb
  \item \textsc{Stationarity:} Are lagged associations equal for
    equivalent lags?  
    \vb
  \item \textsc{Equilibrium:} Are cross-sectional variances and
    covariances equal at all waves?
  \end{itemize}
  \va
  Another definition of stationarity is used in time-series analysis:
  \begin{itemize}
  \item \textsc{Weak Stationarity:}
    \vc
    \begin{enumerate}
    \item Finite variance
      \vc
    \item No trend in mean levels
      \vc
    \item Lag-$K$ auto-covariances are equal
    \end{enumerate}
    \vb
  \item \textsc{Strict Stationarity:} The distribution of the process is
    the same at all time points.
  \end{itemize}
  
\end{frame}


\begin{frame}[allowframebreaks]{Test Assumptions}
  
\begin{Schunk}
\begin{Sinput}
 mod3 <- "
 att3 ~ att2 + b2*conf2 + cp2*horn2
 att2 ~ att1 + b1*conf1 + cp1*horn1
 
 conf3 ~ conf2 + a2*horn2
 conf2 ~ conf1 + a1*horn1
 
 horn3 ~ horn2
 horn2 ~ horn1
 
 horn3 ~~ conf3 + att3
 conf3 ~~ att3
 
 horn2 ~~ conf2 + att2
 conf2 ~~ att2
 
 a1 == a2
 b1 == b2
 cp1 == cp2
 "
 out3 <- sem(mod3, data = dat1)
 summary(out3)
\end{Sinput}
\begin{Soutput}
lavaan (0.5-20) converged normally after  46 iterations

  Number of observations                           500

  Estimator                                         ML
  Minimum Function Test Statistic              294.220
  Degrees of freedom                                18
  P-value (Chi-square)                           0.000

Parameter Estimates:

  Information                                 Expected
  Standard Errors                             Standard

Regressions:
                   Estimate  Std.Err  Z-value  P(>|z|)
  att3 ~                                              
    att2              0.497    0.035   14.234    0.000
    conf2     (b2)    0.098    0.019    5.200    0.000
    horn2    (cp2)    0.083    0.072    1.157    0.247
  att2 ~                                              
    att1              0.530    0.040   13.345    0.000
    conf1     (b1)    0.098    0.019    5.200    0.000
    horn1    (cp1)    0.083    0.072    1.157    0.247
  conf3 ~                                             
    conf2             0.684    0.035   19.602    0.000
    horn2     (a2)    0.493    0.107    4.596    0.000
  conf2 ~                                             
    conf1             0.623    0.032   19.546    0.000
    horn1     (a1)    0.493    0.107    4.596    0.000
  horn3 ~                                             
    horn2             0.826    0.030   27.609    0.000
  horn2 ~                                             
    horn1             0.714    0.024   29.181    0.000

Covariances:
                   Estimate  Std.Err  Z-value  P(>|z|)
  conf3 ~~                                            
    horn3             1.016    0.155    6.556    0.000
  att3 ~~                                             
    horn3             0.322    0.093    3.483    0.000
    conf3             3.574    0.465    7.691    0.000
  conf2 ~~                                            
    horn2             0.836    0.124    6.721    0.000
  att2 ~~                                             
    horn2             0.273    0.083    3.289    0.001
    conf2             2.027    0.400    5.067    0.000

Variances:
                   Estimate  Std.Err  Z-value  P(>|z|)
    att3              6.019    0.381   15.811    0.000
    att2              6.041    0.382   15.811    0.000
    conf3            15.814    1.000   15.811    0.000
    conf2            12.570    0.795   15.811    0.000
    horn3             0.695    0.044   15.811    0.000
    horn2             0.560    0.035   15.811    0.000

Constraints:
                                               |Slack|
    a1 - (a2)                                    0.000
    b1 - (b2)                                    0.000
    cp1 - (cp2)                                  0.000
\end{Soutput}
\begin{Sinput}
 chiDiff <- fitMeasures(out3)["chisq"] -
     fitMeasures(out1)["chisq"]
 dfDiff <- fitMeasures(out3)["df"] -
     fitMeasures(out1)["df"]
 pchisq(chiDiff, dfDiff, lower = FALSE)
\end{Sinput}
\begin{Soutput}
     chisq 
0.02684148 
\end{Soutput}
\end{Schunk}


\end{frame}



\begin{frame}{Weirdness of Causation}

  Causal modeling/inference is underpinned by an odd contradiction:
  \va
  \begin{itemize}
  \item Correlation does not exist in the real world
    \va
  \item An observed correlation is merely an artifact of inadequate
    measurement.
    \vb
  \item If $X$ and $Y$ covary, then there are three possible
    \emph{underlying} reasons:
    \vb
    \begin{enumerate}
    \item $X$ causes $Y$
      \vc
    \item $Y$ causes $X$
      \vc
    \item An unmeasured third variable causes both $X$ and $Y$
    \end{enumerate}
  \end{itemize}
  
\end{frame}



\begin{frame}{Weirdness of Causation}
  
  \begin{itemize}
  \item Probability theory and statistical analysis can only encode
    correlational relations
    \va
    \begin{itemize}
    \item Probability theory can only describe the (in)dependence of $X$
      and $Y$ in terms of their joint distribution.
      \vb
    \item Statistics can only test for a (non)linear association between
      $X$ and $Y$.
    \end{itemize}
    \va
  \item Causal information must be externally imposed by the researcher
    in the form of assumptions and theory.
  \end{itemize}
  
\end{frame}



\begin{frame}{Sufficient Conditions to Infer Causation}

  Three conditions are generally sufficient to suggest that $X$ causes
  $Y$:
  \vb
  \begin{enumerate}
  \item $X$ and $Y$ must covary
    \vb
  \item $X$ must temporally precede $Y$
    \vb
  \item All alternative explanations for the covariance of $X$ and $Y$
    must be eliminated
  \end{enumerate}
  \va
  \pause
  Condition 1 is easy to confirm in any data analytic context.\\
  \va
  Condition 2 can be easily assessed with longitudinal data.\\
  \va
  Condition 3 is impossible to satisfy statistically.
  \begin{itemize}
    \vb
  \item Even the most rigorous experimental designs must make
    assumptions to satisfy Condition 3.
    \vb
  \item Tenability decreases without random assignment.
  \end{itemize}

\end{frame}



\begin{frame}{What do Longitudinal Data Give Us?}
  
  \textsc{Question:} Do we get to claim causation if we find
  mediation with a panel model?\\ 
  \va 
  \pause 
  \textsc{Answer:} No!\\ 
  \va
  \pause 
  But we can make stronger claims than we can with
  cross-sectional data.
  \vb
  \begin{itemize}
  \item Panel models can satisfy Conditions 1 and 2.
    \vb
  \item Panel models can help with Condition 3
    \begin{itemize}
    \item Due to the autoregressive paths, each case acts as its own
      control
      \vb
    \item Autoregression only controls for time-invariant traits
    \end{itemize}
    \vb
  \item We can further improve the strengths of our inferences by
    including appropriate covariates
    \vc
    \begin{itemize}
    \item Same idea as matching non-experimental cases for causal inference
    \end{itemize}
  \end{itemize}
  
\end{frame}


\begin{frame}[allowframebreaks]{Example}
  
\begin{Schunk}
\begin{Sinput}
 parameterEstimates(out2, 
                    boot = "bca.simple")[-c(7 : 18), -c(1 : 3)]
\end{Sinput}
\begin{Soutput}
   label    est    se      z pvalue ci.lower ci.upper
1    cp1 -0.103 0.190 -0.543  0.587   -0.473    0.259
2    cp2  1.099 0.204  5.380  0.000    0.639    1.472
3     b1  1.615 0.167  9.649  0.000    1.270    1.912
4     b2  0.381 0.173  2.206  0.027    0.034    0.720
5     b3  0.571 0.173  3.297  0.001    0.220    0.921
6      a  0.741 0.131  5.638  0.000    0.471    0.984
19   imm  0.424 0.153  2.773  0.006    0.174    0.781
\end{Soutput}
\end{Schunk}


\end{frame}


\end{document}
