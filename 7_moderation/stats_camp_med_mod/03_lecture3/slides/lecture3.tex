\documentclass{beamer}
\usetheme{ttuStatsCamp}
\usefonttheme{serif}
\usepackage[T1]{fontenc}
\usepackage[utf8]{inputenc}
%\usepackage{mathptmx}
\usepackage{url}
\usepackage{graphicx}
\usepackage{setspace}
%\usepackage{esint}
\usepackage[natbibapa]{apacite}
\usepackage{color}
\usepackage{amsmath}
\usepackage{amsfonts}
%\usepackage{bm}
\usepackage{Sweavel}
\usepackage{listings}

\def\Sweavesize{\scriptsize}
\def\Rcolor{\color{black}}
%\def\Routcolor{\color{red}}
\def\Rcommentcolor{\color{violet}}
\def\Rbackground{\color[gray]{0.85}}
\def\Routbackground{\color[gray]{0.85}}

\lstset{tabsize=2, breaklines=true, style=Rstyle}



\newcommand{\red}[0]{\textcolor{red}}
%\newcommand{\violet}[0]{\textcolor{violet}}
\newcommand{\green}[0]{\textcolor{green}}
\newcommand{\blue}[0]{\textcolor{blue}}
\newcommand{\comment}[1]{}
\newcommand{\kfold}[0]{\emph{K}-fold cross-validation}
\newcommand{\va}[0]{\vspace{12pt}}
\newcommand{\vb}[0]{\vspace{6pt}}
\newcommand{\vc}[0]{\vspace{3pt}}
\newcommand{\vx}[1]{\vspace{#1pt}}

\title[Lecture 3]{Lecture 3: Less-Simple Mediation Analysis}

\author{Kyle M. Lang}

\institute[TTU IMMAP]{
  Institute for Measurement, Methodology, Analysis \& Policy\\
  Texas Tech University\\
  Lubbock, TX
}

\date{2016 Stats Camp}

\setbeamertemplate{frametitle continuation}{}

\begin{document}

\setkeys{Gin}{width=\textwidth}

\input{sweaveFiles/-001}


\begin{frame}[plain]

  \titlepage
  
\end{frame}



\begin{frame}{Outline}

  \begin{itemize}
  \item Show how to estimate indirect effects with path diagrams
    \vspace{12pt}
  \item Discuss the use of \emph{bootstrapping} for testing indirect
    effects
    \vspace{12pt}
  \item Discuss the Monte Carlo simulation method for testing indirect
    effects
  \end{itemize}

\end{frame}



\begin{frame}{Recall our Basic Path Diagram}

\begin{figure}
\includegraphics[width=\textwidth]{figures/simpleMediationPathDiagram.pdf}
\end{figure}

\end{frame}


\begin{frame}{The Basic OLS Equations}

  To get all the pieces of the preceding diagram using OLS regression,
  we're forced fit three equations.

\begin{align}
Y &= i_1 + cX + e_1 \label{eq1}\\
Y &= i_2 + c'X + bM + e_2 \label{eq2}\\
M &= i_3 + aX + e_3 \label{eq3}
\end{align}

\begin{itemize}
\item Equation \ref{eq1} gives us the total effect ($c$).
  \vb
\item Equation \ref{eq2} gives us the direct effect ($c'$) and the
  partialled effect of the mediator on the outcome ($b$).
  \vb
\item Equation \ref{eq3} gives us the effect of the input on the
  outcome ($a$).
\end{itemize}

\end{frame}



\begin{frame}[shrink = 5]{Two Measures of Indirect Effect}

  Recall the two definitions of an indirect effect:
  \begin{align}
    IE_{diff} &= c - c'\\
    IE_{prod} &= a \cdot b
  \end{align}

  It pays to remember a few key points:
  \vc
  \begin{itemize}
  \item $IE_{diff}$ and $IE_{prod}$ are equivalent in simple
    mediation.
    \vb
  \item $IE_{diff}$ is only an indirect indication of the parameter
    we're truly interested in $IE_{prod}$.
    \vb
  \item A significant indirect effect can exist without a significant
    total effect.
    \vb
  \item If we hypothesize a significant indirect effect, then we don't
    care about the total effect.
  \end{itemize}
  \vb
  These points imply something interesting:
  \vc
  \begin{itemize}
    \item We don't need to estimate $c$!
  \end{itemize}

\end{frame}



\begin{frame}{Simplifying our Path Diagram}

  \textsc{Question:} If we don't care about estimating $c$, how can we
  simplify this diagram?

  \begin{figure}
    \includegraphics[width=\textwidth]{figures/simpleMediationPathDiagram.pdf}
  \end{figure}

\end{frame}



\begin{frame}{Simplifying our Path Diagram}

  \textsc{Answer:} We don't fit the upper model.
  
  \vspace{-12pt}
  
  \begin{figure}
    \includegraphics[width=\textwidth]{figures/mediationTriadPathDiagram.pdf}
  \end{figure}
  
  \vspace{-24pt}
  
  \textsc{Follow-Up Question:} Do we really need to bother?

\end{frame}



\begin{frame}{Why Path Analysis?}

  \begin{figure}
    \includegraphics[width=\textwidth]{figures/multipleMediationPathDiagrams.pdf}
  \end{figure}

\end{frame}



\begin{frame}{Example}

  Let's revisit the example from last week:

  \vspace{-12pt}
  
  \begin{figure}
    \includegraphics[width=\textwidth]{figures/adamsKlpsExample1PathDiagram.pdf}
  \end{figure}

  \vspace{-24pt}
  
  We already fit this model with the OLS approach.

\end{frame}



\begin{frame}[allowframebreaks]{Example}

\begin{Schunk}
\begin{Sinput}
 dat1 <- readRDS("../data/adamsKlpsScaleScore.rds")
 ## Partial out the mediator's effect:
 mod1 <- lm(policy ~ sysRac + polAffil, data = dat1)
 mod2 <- lm(sysRac ~ polAffil, data = dat1)
 summary(mod1)$coef
\end{Sinput}
\begin{Soutput}
              Estimate Std. Error   t value     Pr(>|t|)
(Intercept) 0.83265885 0.41246491 2.0187386 4.670428e-02
sysRac      0.72235878 0.11147514 6.4799987 5.930291e-09
polAffil    0.05121251 0.06998433 0.7317711 4.663450e-01
\end{Soutput}
\begin{Sinput}
 summary(mod2)$coef
\end{Sinput}
\begin{Soutput}
             Estimate Std. Error  t value     Pr(>|t|)
(Intercept) 2.6060451 0.28489391 9.147423 2.715546e-14
polAffil    0.2568494 0.06213495 4.133735 8.336022e-05
\end{Soutput}
\begin{Sinput}
 ## Extract important parameter estimates:
 a <- coef(mod2)["polAffil"]
 b <- coef(mod1)["sysRac"]
 ## Compute indirect effect:
 ieProd <- a * b
 ieProd
\end{Sinput}
\begin{Soutput}
 polAffil 
0.1855374 
\end{Soutput}
\begin{Sinput}
 ## Calculate Sobel's Z:
 seA <- sqrt(diag(vcov(mod2)))["polAffil"]
 seB <- sqrt(diag(vcov(mod1)))["sysRac"]
 sobelSE <- sqrt(b^2 * seA^2 + a^2 * seB^2)
 sobelZ <- ieProd / sobelSE
 sobelZ
\end{Sinput}
\begin{Soutput}
polAffil 
 3.48501 
\end{Soutput}
\begin{Sinput}
 sobelP <- 2 * pnorm(sobelZ, lower = FALSE)
 sobelP
\end{Sinput}
\begin{Soutput}
    polAffil 
0.0004921178 
\end{Soutput}
\begin{Sinput}
 sobelUB <- ieProd + 1.96 * sobelSE
 sobelLB <- ieProd - 1.96 * sobelSE
 ## 95% Sobel CI:
 c(sobelLB, sobelUB)
\end{Sinput}
\begin{Soutput}
  polAffil   polAffil 
0.08118957 0.28988525 
\end{Soutput}
\end{Schunk}


\end{frame}



\begin{frame}[allowframebreaks]{Example}

\begin{Schunk}
\begin{Sinput}
 parameterEstimates(out1, 
                    boot = "bca.simple")[-c(6 : 13), -c(1 : 3)]
\end{Sinput}
\begin{Soutput}
   label   est    se     z pvalue ci.lower ci.upper
1     cp 0.082 0.259 0.318  0.751   -0.408    0.641
2      b 1.390 0.215 6.465  0.000    0.939    1.786
3     a1 0.729 0.091 8.014  0.000    0.539    0.908
4     a2 0.641 0.089 7.183  0.000    0.469    0.818
5     a3 0.451 0.097 4.643  0.000    0.223    0.629
14   imm 0.628 0.169 3.706  0.000    0.296    0.980
\end{Soutput}
\end{Schunk}


\end{frame}



\begin{frame}[shrink = 15]{Example}
  
\begin{Schunk}
\begin{Sinput}
 mod3 <- "
 att3 ~ att2 + b2*conf2 + cp2*horn2
 att2 ~ att1 + b1*conf1 + cp1*horn1
 
 conf3 ~ conf2 + a2*horn2
 conf2 ~ conf1 + a1*horn1
 
 horn3 ~ horn2
 horn2 ~ horn1
 
 horn3 ~~ conf3 + att3
 conf3 ~~ att3
 
 horn2 ~~ conf2 + att2
 conf2 ~~ att2
 
 a1 == a2
 b1 == b2
 cp1 == cp2
 "
 out3 <- sem(mod3, data = dat1)
 summary(out3)
\end{Sinput}
\begin{Soutput}
lavaan (0.5-20) converged normally after  46 iterations

  Number of observations                           500

  Estimator                                         ML
  Minimum Function Test Statistic              294.220
  Degrees of freedom                                18
  P-value (Chi-square)                           0.000

Parameter Estimates:

  Information                                 Expected
  Standard Errors                             Standard

Regressions:
                   Estimate  Std.Err  Z-value  P(>|z|)
  att3 ~                                              
    att2              0.497    0.035   14.234    0.000
    conf2     (b2)    0.098    0.019    5.200    0.000
    horn2    (cp2)    0.083    0.072    1.157    0.247
  att2 ~                                              
    att1              0.530    0.040   13.345    0.000
    conf1     (b1)    0.098    0.019    5.200    0.000
    horn1    (cp1)    0.083    0.072    1.157    0.247
  conf3 ~                                             
    conf2             0.684    0.035   19.602    0.000
    horn2     (a2)    0.493    0.107    4.596    0.000
  conf2 ~                                             
    conf1             0.623    0.032   19.546    0.000
    horn1     (a1)    0.493    0.107    4.596    0.000
  horn3 ~                                             
    horn2             0.826    0.030   27.609    0.000
  horn2 ~                                             
    horn1             0.714    0.024   29.181    0.000

Covariances:
                   Estimate  Std.Err  Z-value  P(>|z|)
  conf3 ~~                                            
    horn3             1.016    0.155    6.556    0.000
  att3 ~~                                             
    horn3             0.322    0.093    3.483    0.000
    conf3             3.574    0.465    7.691    0.000
  conf2 ~~                                            
    horn2             0.836    0.124    6.721    0.000
  att2 ~~                                             
    horn2             0.273    0.083    3.289    0.001
    conf2             2.027    0.400    5.067    0.000

Variances:
                   Estimate  Std.Err  Z-value  P(>|z|)
    att3              6.019    0.381   15.811    0.000
    att2              6.041    0.382   15.811    0.000
    conf3            15.814    1.000   15.811    0.000
    conf2            12.570    0.795   15.811    0.000
    horn3             0.695    0.044   15.811    0.000
    horn2             0.560    0.035   15.811    0.000

Constraints:
                                               |Slack|
    a1 - (a2)                                    0.000
    b1 - (b2)                                    0.000
    cp1 - (cp2)                                  0.000
\end{Soutput}
\begin{Sinput}
 chiDiff <- fitMeasures(out3)["chisq"] -
     fitMeasures(out1)["chisq"]
 dfDiff <- fitMeasures(out3)["df"] -
     fitMeasures(out1)["df"]
 pchisq(chiDiff, dfDiff, lower = FALSE)
\end{Sinput}
\begin{Soutput}
     chisq 
0.02684148 
\end{Soutput}
\end{Schunk}


\end{frame}



\begin{frame}{Results}
  
  \begin{figure}
    \begin{center}
      \includegraphics[width = \textwidth]{figures/adamsKlpsExample1PathDiagramWithValues.pdf}
    \end{center}
  \end{figure}
  
\end{frame}



\begin{frame}{We're not there yet...}
  
  Path analysis allows us to model complex (and simple) patterns of
  association very parsimoniously, but the preceding example still
  suffers from a considerable limitation.
  \pause
  \va
  \begin{itemize}
  \item The significance test for the indirect effect is still
    conducted with the Sobel Z approach.
  \end{itemize}
  \va
  Path analysis (or full SEM) doesn't magically get around
  distributional problems associated with Sobel's Z test.\\ 
  \va 
  To get a robust significance test of the indirect effect, 
  we need to use \emph{bootstrapping}.
  
\end{frame}



\begin{frame}[shrink = 5]{Bootstrapping}
  
  Bootstrapping was introduced by \citet{efron:1979} as a tool for
  non-parametric inference.
  \vb
  \begin{itemize}
    \item Traditional inference requires that we assume a parametric
      sampling distribution for our focal parameter.
      \vb
    \item We need to make such an assumption to compute the standard
      errors we require for inferences.
      \vb
    \item If we cannot safely make these assumptions, we can use
      bootstrapping.
  \end{itemize}
  \va
  Assume our observed data $Data_0$ represent the population and:
  \vb
  \begin{enumerate}
    \item Sample rows of $Data_0$, with replacement, to create $B$ new
      samples $\{Data_b\}$.
      \vb
    \item Calculate our focal statistic on each of the $B$ bootstrap
      samples. \label{calcStatsStep}
      \vb
    \item Make inferences based on the empirical distribution of the
      $B$ estimates calculated in Step \ref{calcStatsStep}
  \end{enumerate}
  
\end{frame}



\begin{frame}{Bootstrapping}
  
  \begin{figure}
    \begin{center}
      \includegraphics[width=\textwidth]{figures/bootstrappingDiagram.pdf}
    \end{center}
  \end{figure}
  
\end{frame}



\begin{frame}{Bootstrapping}
  
  It's important to recognize that the following are legal bootstrap samples:
  
  \vspace{-12pt}
  
  \begin{figure}
    \begin{center}
      \includegraphics[width=\textwidth]{figures/perverseBootstrappingDiagram.pdf}
    \end{center}
  \end{figure}
  
\end{frame}



\begin{frame}[shrink = 5]{Example}
  
  Suppose I'm on the lookout for a retirement location. Since I want
  to relax in my old-age, I'm concerned with ensuring a low
  probability of dragon attacks, so I have a few salient
  considerations: \vb
  \begin{itemize}
    \item Shooting for a location with no dragons, whatsoever, is a
      fools errand (since dragons are, obviously, ubiquitous).
      \vb
    \item I merely require a location that has at least two times as many
      dragon-free days as other kinds.
  \end{itemize}
  \va
  I've been watching several candidate locales over the course of my
  (long and illustrious) career, and I'm particularly hopeful about
  one quiet hamlet in the Patagonian highlands.
  \vb
  \begin{itemize}
  \item To ensure that my required degree of dragon-freeness is met,
    I'll use the \emph{Dragon Risk Index} (DRI):
    \begin{align*}
      DRI = \textit{Median}\left( \frac{\text{Dragon-Free Days}}{\text{Dragonned Days}} \right)
    \end{align*}
  \end{itemize}
  
\end{frame}



\begin{frame}[allowframebreaks]{Example}
  
\begin{Schunk}
\begin{Sinput}
 parameterEstimates(out2, 
                    boot = "bca.simple")[-c(7 : 18), -c(1 : 3)]
\end{Sinput}
\begin{Soutput}
   label    est    se      z pvalue ci.lower ci.upper
1    cp1 -0.103 0.190 -0.543  0.587   -0.473    0.259
2    cp2  1.099 0.204  5.380  0.000    0.639    1.472
3     b1  1.615 0.167  9.649  0.000    1.270    1.912
4     b2  0.381 0.173  2.206  0.027    0.034    0.720
5     b3  0.571 0.173  3.297  0.001    0.220    0.921
6      a  0.741 0.131  5.638  0.000    0.471    0.984
19   imm  0.424 0.153  2.773  0.006    0.174    0.781
\end{Soutput}
\end{Schunk}


\pagebreak

\begin{Schunk}
\begin{Sinput}
 ## Completely Standardized:
 abCS <- (sdX * ab) / sdY
 abCS
\end{Sinput}
\begin{Soutput}
[1] 0.1345859
\end{Soutput}
\begin{Sinput}
 cPrimeCS <- (sdX * cPrime) / sdY
 cPrimeCS
\end{Sinput}
\begin{Soutput}
       cp 
0.1790413 
\end{Soutput}
\begin{Sinput}
 cCS <- abCS + cPrimeCS
 cCS
\end{Sinput}
\begin{Soutput}
       cp 
0.3136272 
\end{Soutput}
\end{Schunk}

\includegraphics{sweaveFiles/-006}

\pagebreak
To see if I can be confident in the dragon-freeness of my potential
home, I'll summarize the preceding distribution with a (one-tailed)
percentile confidence interval: 
\va
\begin{Schunk}
\begin{Sinput}
 sum(out3.1$fitted - out3.3$fitted)
\end{Sinput}
\begin{Soutput}
[1] -2.785328e-12
\end{Soutput}
\begin{Sinput}
 summary(out3.1)$r.squared
\end{Sinput}
\begin{Soutput}
[1] 0.9999439
\end{Soutput}
\begin{Sinput}
 summary(out3.3)$r.squared
\end{Sinput}
\begin{Soutput}
[1] 0.9999439
\end{Soutput}
\end{Schunk}


\end{frame}


\begin{frame}[shrink = 5]{Bootstrapped Path Modeling}
  
  We can apply the same procedure I just demonstrated to testing the
  indirect effect's significance with path modeling.
  \vb
  \begin{itemize}
  \item The problem with Sobel's Z is exactly the type of issue for which
    bootstrapping was designed
    \vc
    \begin{itemize}
    \item We don't know a reasonable finite-sample sampling
      distribution for the $ab$ parameter.
    \end{itemize}
    \vb  
  \item Bootstrapping will allow us to construct an empirical
    sampling distribution for $ab$ and use that empirical
    distribution to construct confidence intervals for inference.
  \end{itemize}
  \va
  All we need to do is:
  \vc
  \begin{enumerate}
  \item Resample our observed data with replacement
    \vc
  \item Fit our hypothesized path model to each bootstrap sample
    \vc
  \item Store the value of $ab$ that we get each time
    \vc
  \item Summarize the empirical distribution of $ab$ to make inferences
  \end{enumerate}
  
\end{frame}


\begin{frame}[allowframebreaks]{Example}
  
\begin{Schunk}
\begin{Sinput}
 nSams <- 1000
 abVec <- rep(NA, nSams)
 for(i in 1 : nSams) {
     ## Resample the data:
     bootSam <- 
         dat1[sample(c(1 : nrow(dat1)), replace = TRUE), ]
     ## Fit the path model:
     bootOut <- sem(mod2, data = bootSam)
     ## Store the estimated indirect effect:
     abVec[i] <- prod(coef(bootOut)[c("a", "b")])
 }
\end{Sinput}
\end{Schunk}


\pagebreak

\begin{Schunk}
\begin{Sinput}
 ## Conditional process model with a, b, c paths moderated:
 mod4 <- "
 agree ~ b1*open + b2*consc + cp1*extra + cp2*neuro + 
         cp3*extraXneuro + b3*openXconsc
 open ~ a1*extra + a2*neuro + a3*extraXneuro
 
 cpLo  := cp1 + cp3*(-0.962268)
 cpMid := cp1 + cp3*(-0.162268)
 cpHi  := cp1 + cp3*0.837732
 
 abLoLo  := (a1 + a3*(-0.962268)) * (b1 + b3*(-0.4045))
 abLoMid := (a1 + a3*(-0.962268)) * (b1 + b3*(-0.0045))
 abLoHi  := (a1 + a3*(-0.962268)) * (b1 + b3*0.3955)
 
 abMidLo  := (a1 + a3*(-0.162268)) * (b1 + b3*(-0.4045))
 abMidMid := (a1 + a3*(-0.162268)) * (b1 + b3*(-0.0045))
 abMidHi  := (a1 + a3*(-0.162268)) * (b1 + b3*0.3955)
 
 abHiLo  := (a1 + a3*0.837732) * (b1 + b3*(-0.4045))
 abHiMid := (a1 + a3*0.837732) * (b1 + b3*(-0.0045))
 abHiHi  := (a1 + a3*0.837732) * (b1 + b3*0.3955)
 "
\end{Sinput}
\end{Schunk}

\includegraphics{sweaveFiles/-009}

\pagebreak

\begin{Schunk}
\begin{Sinput}
 parameterEstimates(out2.2, boot = bootType)[ , -c(1 : 3)]
\end{Sinput}
\begin{Soutput}
   label    est    se      z pvalue ci.lower ci.upper
1     cp  0.135 0.083  1.638  0.102   -0.031    0.286
2     b2  0.597 0.137  4.359  0.000    0.311    0.858
3     a1 -0.266 0.061 -4.353  0.000   -0.392   -0.148
4    d21 -0.367 0.098 -3.764  0.000   -0.546   -0.166
5         0.987 0.166  5.946  0.000    0.731    1.402
6         0.719 0.116  6.218  0.000    0.527    0.991
7         0.705 0.094  7.520  0.000    0.552    0.926
8         2.444 0.000     NA     NA    2.444    2.444
9 fullIE  0.058 0.025  2.318  0.020    0.019    0.123
\end{Soutput}
\end{Schunk}


\end{frame}


\begin{frame}[shrink = 15]{Example}
  
\begin{Schunk}
\begin{Sinput}
 ## Test Differences between Indirect Effects
 ## in Serial Multiple Mediator Model (Method 1):
 mod2.3 <- "
 policy ~ cp*polAffil + b1*merit + b2*sysRac
 merit ~ a1*polAffil
 sysRac ~ a2*polAffil + d21*merit
 
 ab1 := a1*b1
 ab2 := a2*b2
 fullIE := a1*d21*b2
 totalIE := ab1 + ab2 + fullIE 
 
 fullIE == ab1
 fullIE == ab2
 "
 out2.3 <- 
     sem(mod2.3, data = dat1, se = "boot", boot = nBoot)
 summary(out2.3)
\end{Sinput}
\begin{Soutput}
lavaan (0.5-20) converged normally after 213 iterations

  Number of observations                            87

  Estimator                                         ML
  Minimum Function Test Statistic                1.334
  Degrees of freedom                                 2
  P-value (Chi-square)                           0.513

Parameter Estimates:

  Information                                 Observed
  Standard Errors                            Bootstrap
  Number of requested bootstrap draws             2500
  Number of successful bootstrap draws            2500

Regressions:
                   Estimate  Std.Err  Z-value  P(>|z|)
  policy ~                                            
    polAffil  (cp)    0.108    0.084    1.281    0.200
    merit     (b1)   -0.150    0.047   -3.183    0.001
    sysRac    (b2)    0.521    0.125    4.157    0.000
  merit ~                                             
    polAffil  (a1)   -0.271    0.057   -4.750    0.000
  sysRac ~                                            
    polAffil  (a2)    0.078    0.025    3.125    0.002
    merit    (d21)   -0.287    0.075   -3.814    0.000

Variances:
                   Estimate  Std.Err  Z-value  P(>|z|)
    policy            1.001    0.171    5.854    0.000
    merit             0.719    0.114    6.330    0.000
    sysRac            0.690    0.090    7.632    0.000

Defined Parameters:
                   Estimate  Std.Err  Z-value  P(>|z|)
    ab1               0.041    0.014    2.983    0.003
    ab2               0.041    0.014    2.983    0.003
    fullIE            0.041    0.014    2.983    0.003
    totalIE           0.122    0.041    2.983    0.003

Constraints:
                                               |Slack|
    fullIE - (ab1)                               0.000
    fullIE - (ab2)                               0.000
\end{Soutput}
\begin{Sinput}
 ## Conduct a chi-squared difference test:
 chiDiff <- fitMeasures(out2.3)["chisq"] - 
     fitMeasures(out2.1)["chisq"]
 dfDiff <- fitMeasures(out2.3)["df"] - 
     fitMeasures(out2.1)["df"]
 pchisq(chiDiff, dfDiff, lower = FALSE)
\end{Sinput}
\begin{Soutput}
    chisq 
0.5131246 
\end{Soutput}
\end{Schunk}


\end{frame}
  

\begin{frame}[shrink = 5]{Monte Carlo Method/Parametric Bootstrap}
  
  We can also use a \emph{parametric bootstrap} of the individual
  parameters $a$ and $b$ to get a somewhat robust test of the indirect
  effect.  
  \vb
  \begin{itemize}
  \item Assuming normal sampling distributions for $a$ and $b$ is
    not, generally, problematic
    \vb
  \item We can save ourselves a lot of computational effort by
    assuming normality for $a$ and $b$, then:
    \vb
    \begin{enumerate}
    \item Fit the hypothesized path model to the raw data
      \vb
    \item Extract $a$, $b$, and $\textit{ACOV}(\{a, b\})$ from the fitted
      path model 
      \vb
    \item Parameterize a bivariate normal distribution $\text{N}(a, b
      |\mu, \Sigma)$ with $\mu = \{a, b\}$ and $\Sigma =
      \textit{ACOV}(\{a, b\})$ 
      \vb
    \item Draw simulated values $\{\tilde{a}, \tilde{b}\}$ from
      $\text{N}(a, b | \mu, \Sigma)$ 
      \vb
    \item Compute the simulated indirect effect
      $\widetilde{ab} = \tilde{a} \cdot \tilde{b}$ and store it
      \vb
    \item Summarize the empirical distribution of $\widetilde{ab}$
      for inference.
    \end{enumerate}
  \end{itemize}
  
\end{frame}



\begin{frame}[allowframebreaks]{Example}
  
\begin{Schunk}
\begin{Sinput}
 ## Serial Multiple Mediator Model with 3 Mediators:
 mod3.1 <- "
 policy ~ b1*merit + b2*sysRac + b3*revDisc + cp*polAffil
 revDisc ~ d31*merit + d32*sysRac + a3*polAffil
 sysRac ~ d21*merit + a2*polAffil
 merit ~ a1*polAffil
 
 ab1 := a1*b1
 ab2 := a2*b2
 ab3 := a3*b3
 
 partIE1 := a1*d31*b3
 partIE2 := a1*d21*b2
 partIE3 := a2*d32*b3
 
 fullIE := a1*d21*d32*b3
 
 totalIE := ab1 + ab2 + ab3 + partIE1 + partIE2 + partIE3 + fullIE 
 "
 out3.1 <- 
     sem(mod3.1, data = dat1, se = "boot", boot = nBoot)
 summary(out3.1)
\end{Sinput}
\begin{Soutput}
lavaan (0.5-20) converged normally after  23 iterations

  Number of observations                            87

  Estimator                                         ML
  Minimum Function Test Statistic                0.000
  Degrees of freedom                                 0

Parameter Estimates:

  Information                                 Observed
  Standard Errors                            Bootstrap
  Number of requested bootstrap draws             2500
  Number of successful bootstrap draws            2498

Regressions:
                   Estimate  Std.Err  Z-value  P(>|z|)
  policy ~                                            
    merit     (b1)    0.005    0.144    0.035    0.972
    sysRac    (b2)    0.589    0.151    3.895    0.000
    revDisc   (b3)   -0.026    0.080   -0.330    0.741
    polAffil  (cp)    0.130    0.080    1.616    0.106
  revDisc ~                                           
    merit    (d31)    0.473    0.190    2.490    0.013
    sysRac   (d32)   -0.196    0.243   -0.806    0.420
    polAffil  (a3)   -0.149    0.131   -1.140    0.254
  sysRac ~                                            
    merit    (d21)   -0.301    0.109   -2.765    0.006
    polAffil  (a2)    0.090    0.071    1.270    0.204
  merit ~                                             
    polAffil  (a1)   -0.266    0.061   -4.340    0.000

Variances:
                   Estimate  Std.Err  Z-value  P(>|z|)
    policy            0.985    0.164    6.023    0.000
    revDisc           2.361    0.307    7.698    0.000
    sysRac            0.689    0.091    7.612    0.000
    merit             0.719    0.111    6.482    0.000

Defined Parameters:
                   Estimate  Std.Err  Z-value  P(>|z|)
    ab1              -0.001    0.040   -0.033    0.973
    ab2               0.053    0.043    1.224    0.221
    ab3               0.004    0.016    0.244    0.807
    partIE1           0.003    0.012    0.273    0.785
    partIE2           0.047    0.026    1.831    0.067
    partIE3           0.000    0.003    0.150    0.881
    fullIE            0.000    0.002    0.191    0.849
    totalIE           0.107    0.052    2.052    0.040
\end{Soutput}
\end{Schunk}


\end{frame}


\begin{frame}[allowframebreaks]{Aside Regarding Asymptotic Covariances}

  The \emph{asymptotic covariance matrix} (ACOV) is (-1 times) the
  inverse of the Fisher information matrix of the model parameters.
  \vb  
  \begin{itemize}
  \item The $ACOV$ contains the expected covariance among the ML
    estimates of the model parameters.
    \vb
  \item The diagonal elements of the matrix (i.e., the asymptotic
    variances) are the square of the usual ML SE estimates.
  \end{itemize}
  \va
\begin{Schunk}
\begin{Sinput}
 parameterEstimates(out6, 
                    boot = "bca.simple")[-c(11 : 23), -c(1 : 3)]
\end{Sinput}
\begin{Soutput}
     label    est    se      z pvalue ci.lower ci.upper
1       cp  0.055 0.232  0.238  0.811   -0.431    0.474
2       b1  0.427 0.252  1.693  0.090   -0.068    0.929
3       b2  0.763 0.204  3.735  0.000    0.334    1.151
4       a2  0.473 0.073  6.480  0.000    0.329    0.625
5       d1  0.689 0.090  7.652  0.000    0.509    0.870
6       d2  0.863 0.110  7.843  0.000    0.670    1.116
7       d3  0.400 0.065  6.139  0.000    0.257    0.528
8       a1  0.720 0.089  8.110  0.000    0.521    0.883
9       a2  0.473 0.073  6.480  0.000    0.329    0.625
10      a3  0.476 0.098  4.871  0.000    0.296    0.674
24    imm1  0.203 0.128  1.580  0.114   -0.013    0.518
25     ab2  0.361 0.108  3.342  0.001    0.164    0.596
26 fullIE1  0.019 0.034  0.551  0.582   -0.020    0.134
27 fullIE2  1.295 0.387  3.344  0.001    0.562    2.102
28  mmTest -1.276 0.388 -3.291  0.001   -2.082   -0.547
\end{Soutput}
\end{Schunk}


\end{frame}


\begin{frame}[allowframebreaks]{Back to the Example}
  
\begin{Schunk}
\begin{Sinput}
 par(family = "serif", cex = 0.75)
 library(rockchalk)
 ## First we need to create a 'plotSlopes' object:
 plotOut <- plotSlopes(model = out1,
                       plotx = "ratioC",
                       modx = "perceptionC",
                       plotPoints = FALSE)
 ## Then we modify 'plotOut' to get the J-N test:
 testOut <- testSlopes(plotOut)
\end{Sinput}
\begin{Soutput}
Values of perceptionC OUTSIDE this interval:
      lo       hi 
1.327479 2.452667 
cause the slope of (b1 + b2*perceptionC)ratioC to be statistically significant
\end{Soutput}
\end{Schunk}


\pagebreak

\begin{Schunk}
\begin{Sinput}
 ## Construct product terms to facilitate J-N technique:
 dat1$openXneuro <- with(dat1, neuro*open)
 dat1$concXneuro <- with(dat1, neuro*conc)
 dat1$openXconc <- with(dat1, open*conc)
 dat1$openXconcXneuro <- with(dat1, open*conc*neuro)
\end{Sinput}
\end{Schunk}

\includegraphics{sweaveFiles/-015}

\pagebreak

\begin{Schunk}
\begin{Sinput}
 ## Calculate the percentile CI:
 lb <- sort(abVec)[0.025 * nSams]
 ub <- sort(abVec)[0.975 * nSams]
 c(lb, ub)
\end{Sinput}
\begin{Soutput}
[1] 0.08845936 0.29432389
\end{Soutput}
\end{Schunk}


\end{frame}


\begin{frame}{Kris Preacher's Website}
  
  If you don't want to program the Monte Carlo approach yourself
  (although each of you easily can), you should consider the very
  handy Web App on Professor Kris Preacher's website
  \url{http://www.quantpsy.org}.  
  \va
  \begin{itemize}
    \item Kris' website has a vast array of hugely helpful resources
      for anyone doing mediation or moderation analysis.
      \vb 
    \item You should definitely check it out!
  \end{itemize}
  
\end{frame}



\begin{frame}{References}
\bibliographystyle{apacite}
\bibliography{../../bibtexStuff/dissRefsList}
\end{frame}


\end{document}
