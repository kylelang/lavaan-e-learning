\documentclass{beamer}
\usetheme{ttuStatsCamp}
\usefonttheme{serif}
\usepackage[T1]{fontenc}
\usepackage[utf8]{inputenc}
\usepackage{url}
\usepackage{graphicx}
\usepackage{setspace}
\usepackage[natbibapa]{apacite}
\usepackage{color}
\usepackage{amsmath}
\usepackage{amsfonts}
\usepackage{Sweavel}
\usepackage{listings}
\usepackage{fancybox}

\def\Sweavesize{\scriptsize}
\def\Rcolor{\color{black}}
%\def\Routcolor{\color{red}}
\def\Rcommentcolor{\color{violet}}
\def\Rbackground{\color[gray]{0.85}}
\def\Routbackground{\color[gray]{0.85}}

\lstset{tabsize=2, breaklines=true, style=Rstyle}



\newcommand{\red}[0]{\textcolor{red}}
\newcommand{\green}[0]{\textcolor{green}}
\newcommand{\blue}[0]{\textcolor{blue}}
\newcommand{\comment}[1]{}
\newcommand{\va}[0]{\vspace{12pt}}
\newcommand{\vb}[0]{\vspace{6pt}}
\newcommand{\vc}[0]{\vspace{3pt}}
\newcommand{\vx}[1]{\vspace{#1pt}}

\title[Lecture 9]{Lecture 9: Moderation Wrap-Up}

\author{Kyle M. Lang}

\institute[TTU IMMAP]{
  Institute for Measurement, Methodology, Analysis \& Policy\\
  Texas Tech University\\
  Lubbock, TX
}

\date{2016 Stats Camp}

\setbeamertemplate{frametitle continuation}{}

\begin{document}

\setkeys{Gin}{width=\textwidth}

\input{sweaveFiles/-001}


\begin{frame}[plain]
  
  \titlepage
  
\end{frame}



\begin{frame}{Outline}

  \begin{itemize}
  \item Multiple Moderators
    \va
  \item Moderated Moderation
    \va
  \item Categorical Moderation
    \va
  \item Multiple Group Modeling
  \end{itemize}

\end{frame}



\begin{frame}{Starting Point}

  So far, we've been looking at this type of model:

  \begin{figure}
    \includegraphics[width=0.7\textwidth]{figures/simpleConceptual.pdf}
  \end{figure}

  We've had one focal variable and one moderator.
  \begin{itemize}
    \item We've been asking questions about how the focal effect
      changes as a function of the moderator.
    \item There's no reason we need to restrict ourselves to a single
      moderator.
  \end{itemize}
  
\end{frame}



\begin{frame}{Multiple Moderation}
  
  Maybe we suspect that the focal effect changes as a function of two other variables.
  \begin{itemize}
    \item We could fit this type of model:
  \end{itemize}

  \begin{figure}
    \includegraphics[width=0.7\textwidth]{figures/twoModConceptual.pdf}
  \end{figure}

  Now, the focal effect of $X$ on $Y$ changes as a function of both
  $Z$ and $W$.
  
\end{frame}


\begin{frame}{Multiple Moderation}
  
  The preceding diagram implies the following formula:
  \begin{align*}
    Y = \alpha + f(Z, W) X + \beta_2Z + \beta_3W + e,
  \end{align*}\\
  \va 
  Taking $f(Z, W)$ to be the following simple slope:
  \begin{align*}
    f(Z, W) = \beta_1 + \beta_4Z + \beta_5W
  \end{align*}\\
  \va 
  Produces the following analytic equation:
  \begin{align*}
    Y = \alpha + \beta_1X + \beta_2Z + \beta_3W + \beta_4XZ + \beta_5XW + e
  \end{align*}\\
  \va
  We can easily fit this model in any regression software
  \vb
  \begin{itemize}
  \item We can test for significant moderating effects of $Z$ and $W$
    by testing for non-zero $\beta_4$ and $\beta_5$, respectively.
  \end{itemize}
  
\end{frame}



\begin{frame}{Multiple Moderation}
  
  Our analytic diagram is predictably extended:

  \begin{figure}
    \includegraphics[width=\textwidth]{figures/twoModAnalytic.pdf}
  \end{figure}

\end{frame}

  

\begin{frame}[allowframebreaks]{Example}
    
\begin{Schunk}
\begin{Sinput}
 dat1 <- readRDS("../data/adamsKlpsScaleScore.rds")
 ## Partial out the mediator's effect:
 mod1 <- lm(policy ~ sysRac + polAffil, data = dat1)
 mod2 <- lm(sysRac ~ polAffil, data = dat1)
 summary(mod1)$coef
\end{Sinput}
\begin{Soutput}
              Estimate Std. Error   t value     Pr(>|t|)
(Intercept) 0.83265885 0.41246491 2.0187386 4.670428e-02
sysRac      0.72235878 0.11147514 6.4799987 5.930291e-09
polAffil    0.05121251 0.06998433 0.7317711 4.663450e-01
\end{Soutput}
\begin{Sinput}
 summary(mod2)$coef
\end{Sinput}
\begin{Soutput}
             Estimate Std. Error  t value     Pr(>|t|)
(Intercept) 2.6060451 0.28489391 9.147423 2.715546e-14
polAffil    0.2568494 0.06213495 4.133735 8.336022e-05
\end{Soutput}
\begin{Sinput}
 ## Extract important parameter estimates:
 a <- coef(mod2)["polAffil"]
 b <- coef(mod1)["sysRac"]
 ## Compute indirect effect:
 ieProd <- a * b
 ieProd
\end{Sinput}
\begin{Soutput}
 polAffil 
0.1855374 
\end{Soutput}
\begin{Sinput}
 ## Calculate Sobel's Z:
 seA <- sqrt(diag(vcov(mod2)))["polAffil"]
 seB <- sqrt(diag(vcov(mod1)))["sysRac"]
 sobelSE <- sqrt(b^2 * seA^2 + a^2 * seB^2)
 sobelZ <- ieProd / sobelSE
 sobelZ
\end{Sinput}
\begin{Soutput}
polAffil 
 3.48501 
\end{Soutput}
\begin{Sinput}
 sobelP <- 2 * pnorm(sobelZ, lower = FALSE)
 sobelP
\end{Sinput}
\begin{Soutput}
    polAffil 
0.0004921178 
\end{Soutput}
\begin{Sinput}
 sobelUB <- ieProd + 1.96 * sobelSE
 sobelLB <- ieProd - 1.96 * sobelSE
 ## 95% Sobel CI:
 c(sobelLB, sobelUB)
\end{Sinput}
\begin{Soutput}
  polAffil   polAffil 
0.08118957 0.28988525 
\end{Soutput}
\end{Schunk}


\pagebreak

\begin{Schunk}
\begin{Sinput}
 parameterEstimates(out1, 
                    boot = "bca.simple")[-c(6 : 13), -c(1 : 3)]
\end{Sinput}
\begin{Soutput}
   label   est    se     z pvalue ci.lower ci.upper
1     cp 0.082 0.259 0.318  0.751   -0.408    0.641
2      b 1.390 0.215 6.465  0.000    0.939    1.786
3     a1 0.729 0.091 8.014  0.000    0.539    0.908
4     a2 0.641 0.089 7.183  0.000    0.469    0.818
5     a3 0.451 0.097 4.643  0.000    0.223    0.629
14   imm 0.628 0.169 3.706  0.000    0.296    0.980
\end{Soutput}
\end{Schunk}


\pagebreak

\begin{Schunk}
\begin{Sinput}
 mod3 <- "
 att3 ~ att2 + b2*conf2 + cp2*horn2
 att2 ~ att1 + b1*conf1 + cp1*horn1
 
 conf3 ~ conf2 + a2*horn2
 conf2 ~ conf1 + a1*horn1
 
 horn3 ~ horn2
 horn2 ~ horn1
 
 horn3 ~~ conf3 + att3
 conf3 ~~ att3
 
 horn2 ~~ conf2 + att2
 conf2 ~~ att2
 
 a1 == a2
 b1 == b2
 cp1 == cp2
 "
 out3 <- sem(mod3, data = dat1)
 summary(out3)
\end{Sinput}
\begin{Soutput}
lavaan (0.5-20) converged normally after  46 iterations

  Number of observations                           500

  Estimator                                         ML
  Minimum Function Test Statistic              294.220
  Degrees of freedom                                18
  P-value (Chi-square)                           0.000

Parameter Estimates:

  Information                                 Expected
  Standard Errors                             Standard

Regressions:
                   Estimate  Std.Err  Z-value  P(>|z|)
  att3 ~                                              
    att2              0.497    0.035   14.234    0.000
    conf2     (b2)    0.098    0.019    5.200    0.000
    horn2    (cp2)    0.083    0.072    1.157    0.247
  att2 ~                                              
    att1              0.530    0.040   13.345    0.000
    conf1     (b1)    0.098    0.019    5.200    0.000
    horn1    (cp1)    0.083    0.072    1.157    0.247
  conf3 ~                                             
    conf2             0.684    0.035   19.602    0.000
    horn2     (a2)    0.493    0.107    4.596    0.000
  conf2 ~                                             
    conf1             0.623    0.032   19.546    0.000
    horn1     (a1)    0.493    0.107    4.596    0.000
  horn3 ~                                             
    horn2             0.826    0.030   27.609    0.000
  horn2 ~                                             
    horn1             0.714    0.024   29.181    0.000

Covariances:
                   Estimate  Std.Err  Z-value  P(>|z|)
  conf3 ~~                                            
    horn3             1.016    0.155    6.556    0.000
  att3 ~~                                             
    horn3             0.322    0.093    3.483    0.000
    conf3             3.574    0.465    7.691    0.000
  conf2 ~~                                            
    horn2             0.836    0.124    6.721    0.000
  att2 ~~                                             
    horn2             0.273    0.083    3.289    0.001
    conf2             2.027    0.400    5.067    0.000

Variances:
                   Estimate  Std.Err  Z-value  P(>|z|)
    att3              6.019    0.381   15.811    0.000
    att2              6.041    0.382   15.811    0.000
    conf3            15.814    1.000   15.811    0.000
    conf2            12.570    0.795   15.811    0.000
    horn3             0.695    0.044   15.811    0.000
    horn2             0.560    0.035   15.811    0.000

Constraints:
                                               |Slack|
    a1 - (a2)                                    0.000
    b1 - (b2)                                    0.000
    cp1 - (cp2)                                  0.000
\end{Soutput}
\begin{Sinput}
 chiDiff <- fitMeasures(out3)["chisq"] -
     fitMeasures(out1)["chisq"]
 dfDiff <- fitMeasures(out3)["df"] -
     fitMeasures(out1)["df"]
 pchisq(chiDiff, dfDiff, lower = FALSE)
\end{Sinput}
\begin{Soutput}
     chisq 
0.02684148 
\end{Soutput}
\end{Schunk}


\pagebreak

\begin{Schunk}
\begin{Sinput}
 parameterEstimates(out2, 
                    boot = "bca.simple")[-c(7 : 18), -c(1 : 3)]
\end{Sinput}
\begin{Soutput}
   label    est    se      z pvalue ci.lower ci.upper
1    cp1 -0.103 0.190 -0.543  0.587   -0.473    0.259
2    cp2  1.099 0.204  5.380  0.000    0.639    1.472
3     b1  1.615 0.167  9.649  0.000    1.270    1.912
4     b2  0.381 0.173  2.206  0.027    0.034    0.720
5     b3  0.571 0.173  3.297  0.001    0.220    0.921
6      a  0.741 0.131  5.638  0.000    0.471    0.984
19   imm  0.424 0.153  2.773  0.006    0.174    0.781
\end{Soutput}
\end{Schunk}


\pagebreak

\begin{Schunk}
\begin{Sinput}
 ## Completely Standardized:
 abCS <- (sdX * ab) / sdY
 abCS
\end{Sinput}
\begin{Soutput}
[1] 0.1345859
\end{Soutput}
\begin{Sinput}
 cPrimeCS <- (sdX * cPrime) / sdY
 cPrimeCS
\end{Sinput}
\begin{Soutput}
       cp 
0.1790413 
\end{Soutput}
\begin{Sinput}
 cCS <- abCS + cPrimeCS
 cCS
\end{Sinput}
\begin{Soutput}
       cp 
0.3136272 
\end{Soutput}
\end{Schunk}


\pagebreak

\begin{Schunk}
\begin{Sinput}
 sum(out3.1$fitted - out3.3$fitted)
\end{Sinput}
\begin{Soutput}
[1] -2.785328e-12
\end{Soutput}
\begin{Sinput}
 summary(out3.1)$r.squared
\end{Sinput}
\begin{Soutput}
[1] 0.9999439
\end{Soutput}
\begin{Sinput}
 summary(out3.3)$r.squared
\end{Sinput}
\begin{Soutput}
[1] 0.9999439
\end{Soutput}
\end{Schunk}


\pagebreak

\begin{Schunk}
\begin{Sinput}
 nSams <- 1000
 abVec <- rep(NA, nSams)
 for(i in 1 : nSams) {
     ## Resample the data:
     bootSam <- 
         dat1[sample(c(1 : nrow(dat1)), replace = TRUE), ]
     ## Fit the path model:
     bootOut <- sem(mod2, data = bootSam)
     ## Store the estimated indirect effect:
     abVec[i] <- prod(coef(bootOut)[c("a", "b")])
 }
\end{Sinput}
\end{Schunk}

\includegraphics{sweaveFiles/-008}

\pagebreak

\begin{Schunk}
\begin{Sinput}
 ## Conditional process model with a, b, c paths moderated:
 mod4 <- "
 agree ~ b1*open + b2*consc + cp1*extra + cp2*neuro + 
         cp3*extraXneuro + b3*openXconsc
 open ~ a1*extra + a2*neuro + a3*extraXneuro
 
 cpLo  := cp1 + cp3*(-0.962268)
 cpMid := cp1 + cp3*(-0.162268)
 cpHi  := cp1 + cp3*0.837732
 
 abLoLo  := (a1 + a3*(-0.962268)) * (b1 + b3*(-0.4045))
 abLoMid := (a1 + a3*(-0.962268)) * (b1 + b3*(-0.0045))
 abLoHi  := (a1 + a3*(-0.962268)) * (b1 + b3*0.3955)
 
 abMidLo  := (a1 + a3*(-0.162268)) * (b1 + b3*(-0.4045))
 abMidMid := (a1 + a3*(-0.162268)) * (b1 + b3*(-0.0045))
 abMidHi  := (a1 + a3*(-0.162268)) * (b1 + b3*0.3955)
 
 abHiLo  := (a1 + a3*0.837732) * (b1 + b3*(-0.4045))
 abHiMid := (a1 + a3*0.837732) * (b1 + b3*(-0.0045))
 abHiHi  := (a1 + a3*0.837732) * (b1 + b3*0.3955)
 "
\end{Sinput}
\end{Schunk}

\includegraphics{sweaveFiles/-009}

\end{frame}



\begin{frame}{Moderated Moderation}
  
  The additive two-way interaction model is more flexible than the
  simple single-moderator model, but it still imposes constraints.
  \va
  \begin{itemize}
    \item The moderating effect of $Z$ (or $W$) on the $X \rightarrow
      Y$ relation is assumed to be constant across levels of $W$ (or
      $Z$).
      \vb
    \item I.e., the moderation is not moderated
  \end{itemize}
  
  \va
  We can relax this constraint by modeling moderation of the moderated
  effect using a three-way interaction.
  
\end{frame}



\begin{frame}{Moderated Moderation}
  
  Moderated moderation implies the following conceptual diagram:
  \begin{figure}
    \includegraphics[width=\textwidth]{figures/threeWayConceptual.pdf}
  \end{figure}
  
\end{frame}



\begin{frame}{Moderated Moderation}
  
  The preceding conceptual diagram implies this analytic diagram:
  \begin{figure}
    \includegraphics[width=\textwidth]{figures/threeWayAnalytic.pdf}
  \end{figure}
  
\end{frame}



\begin{frame}[shrink = 5]{Moderated Moderation}
  
  The preceding diagram represents the following equation:
  \begin{align*}
    Y =& ~ \alpha + \beta_1X + \beta_2Z + \beta_3W +\\
    &\beta_4XZ + \beta_5XW + \beta_6ZW + \beta_7XZW + e
  \end{align*}\\
  \vb
  Which can be restructured into:
  \begin{align*}
    Y =& ~ \alpha + (\beta_1 + \beta_4Z + \beta_5W + \beta_7ZW)X + \\
    &\beta_2Z + \beta_3W + \beta_6ZW + e\\
    =& ~ \alpha + g(Z, W)X + \beta_2Z + \beta_3W + \beta_6ZW + e
  \end{align*}\\
  \vb 
  With moderated moderation, the simple slope is given by:
  \begin{align*}
    g(Z, W) = \beta_1 + \beta_4Z + \beta_5W + \beta_7ZW
  \end{align*}\\
  \vb 
  Which has the same structure as a single moderator model.
  \vc
  \begin{itemize}
  \item Three-way simple slopes represent the moderated effect of
    $Z$ on the $X \rightarrow Y$ relation at conditional values of $W$.
  \end{itemize}
  
\end{frame}



\begin{frame}[allowframebreaks]{Example}
    
\begin{Schunk}
\begin{Sinput}
 parameterEstimates(out2.2, boot = bootType)[ , -c(1 : 3)]
\end{Sinput}
\begin{Soutput}
   label    est    se      z pvalue ci.lower ci.upper
1     cp  0.135 0.083  1.638  0.102   -0.031    0.286
2     b2  0.597 0.137  4.359  0.000    0.311    0.858
3     a1 -0.266 0.061 -4.353  0.000   -0.392   -0.148
4    d21 -0.367 0.098 -3.764  0.000   -0.546   -0.166
5         0.987 0.166  5.946  0.000    0.731    1.402
6         0.719 0.116  6.218  0.000    0.527    0.991
7         0.705 0.094  7.520  0.000    0.552    0.926
8         2.444 0.000     NA     NA    2.444    2.444
9 fullIE  0.058 0.025  2.318  0.020    0.019    0.123
\end{Soutput}
\end{Schunk}


\pagebreak

\begin{Schunk}
\begin{Sinput}
 ## Test Differences between Indirect Effects
 ## in Serial Multiple Mediator Model (Method 1):
 mod2.3 <- "
 policy ~ cp*polAffil + b1*merit + b2*sysRac
 merit ~ a1*polAffil
 sysRac ~ a2*polAffil + d21*merit
 
 ab1 := a1*b1
 ab2 := a2*b2
 fullIE := a1*d21*b2
 totalIE := ab1 + ab2 + fullIE 
 
 fullIE == ab1
 fullIE == ab2
 "
 out2.3 <- 
     sem(mod2.3, data = dat1, se = "boot", boot = nBoot)
 summary(out2.3)
\end{Sinput}
\begin{Soutput}
lavaan (0.5-20) converged normally after 213 iterations

  Number of observations                            87

  Estimator                                         ML
  Minimum Function Test Statistic                1.334
  Degrees of freedom                                 2
  P-value (Chi-square)                           0.513

Parameter Estimates:

  Information                                 Observed
  Standard Errors                            Bootstrap
  Number of requested bootstrap draws             2500
  Number of successful bootstrap draws            2500

Regressions:
                   Estimate  Std.Err  Z-value  P(>|z|)
  policy ~                                            
    polAffil  (cp)    0.108    0.084    1.281    0.200
    merit     (b1)   -0.150    0.047   -3.183    0.001
    sysRac    (b2)    0.521    0.125    4.157    0.000
  merit ~                                             
    polAffil  (a1)   -0.271    0.057   -4.750    0.000
  sysRac ~                                            
    polAffil  (a2)    0.078    0.025    3.125    0.002
    merit    (d21)   -0.287    0.075   -3.814    0.000

Variances:
                   Estimate  Std.Err  Z-value  P(>|z|)
    policy            1.001    0.171    5.854    0.000
    merit             0.719    0.114    6.330    0.000
    sysRac            0.690    0.090    7.632    0.000

Defined Parameters:
                   Estimate  Std.Err  Z-value  P(>|z|)
    ab1               0.041    0.014    2.983    0.003
    ab2               0.041    0.014    2.983    0.003
    fullIE            0.041    0.014    2.983    0.003
    totalIE           0.122    0.041    2.983    0.003

Constraints:
                                               |Slack|
    fullIE - (ab1)                               0.000
    fullIE - (ab2)                               0.000
\end{Soutput}
\begin{Sinput}
 ## Conduct a chi-squared difference test:
 chiDiff <- fitMeasures(out2.3)["chisq"] - 
     fitMeasures(out2.1)["chisq"]
 dfDiff <- fitMeasures(out2.3)["df"] - 
     fitMeasures(out2.1)["df"]
 pchisq(chiDiff, dfDiff, lower = FALSE)
\end{Sinput}
\begin{Soutput}
    chisq 
0.5131246 
\end{Soutput}
\end{Schunk}


\pagebreak

\begin{Schunk}
\begin{Sinput}
 ## Serial Multiple Mediator Model with 3 Mediators:
 mod3.1 <- "
 policy ~ b1*merit + b2*sysRac + b3*revDisc + cp*polAffil
 revDisc ~ d31*merit + d32*sysRac + a3*polAffil
 sysRac ~ d21*merit + a2*polAffil
 merit ~ a1*polAffil
 
 ab1 := a1*b1
 ab2 := a2*b2
 ab3 := a3*b3
 
 partIE1 := a1*d31*b3
 partIE2 := a1*d21*b2
 partIE3 := a2*d32*b3
 
 fullIE := a1*d21*d32*b3
 
 totalIE := ab1 + ab2 + ab3 + partIE1 + partIE2 + partIE3 + fullIE 
 "
 out3.1 <- 
     sem(mod3.1, data = dat1, se = "boot", boot = nBoot)
 summary(out3.1)
\end{Sinput}
\begin{Soutput}
lavaan (0.5-20) converged normally after  23 iterations

  Number of observations                            87

  Estimator                                         ML
  Minimum Function Test Statistic                0.000
  Degrees of freedom                                 0

Parameter Estimates:

  Information                                 Observed
  Standard Errors                            Bootstrap
  Number of requested bootstrap draws             2500
  Number of successful bootstrap draws            2498

Regressions:
                   Estimate  Std.Err  Z-value  P(>|z|)
  policy ~                                            
    merit     (b1)    0.005    0.144    0.035    0.972
    sysRac    (b2)    0.589    0.151    3.895    0.000
    revDisc   (b3)   -0.026    0.080   -0.330    0.741
    polAffil  (cp)    0.130    0.080    1.616    0.106
  revDisc ~                                           
    merit    (d31)    0.473    0.190    2.490    0.013
    sysRac   (d32)   -0.196    0.243   -0.806    0.420
    polAffil  (a3)   -0.149    0.131   -1.140    0.254
  sysRac ~                                            
    merit    (d21)   -0.301    0.109   -2.765    0.006
    polAffil  (a2)    0.090    0.071    1.270    0.204
  merit ~                                             
    polAffil  (a1)   -0.266    0.061   -4.340    0.000

Variances:
                   Estimate  Std.Err  Z-value  P(>|z|)
    policy            0.985    0.164    6.023    0.000
    revDisc           2.361    0.307    7.698    0.000
    sysRac            0.689    0.091    7.612    0.000
    merit             0.719    0.111    6.482    0.000

Defined Parameters:
                   Estimate  Std.Err  Z-value  P(>|z|)
    ab1              -0.001    0.040   -0.033    0.973
    ab2               0.053    0.043    1.224    0.221
    ab3               0.004    0.016    0.244    0.807
    partIE1           0.003    0.012    0.273    0.785
    partIE2           0.047    0.026    1.831    0.067
    partIE3           0.000    0.003    0.150    0.881
    fullIE            0.000    0.002    0.191    0.849
    totalIE           0.107    0.052    2.052    0.040
\end{Soutput}
\end{Schunk}


\pagebreak

\begin{Schunk}
\begin{Sinput}
 parameterEstimates(out6, 
                    boot = "bca.simple")[-c(11 : 23), -c(1 : 3)]
\end{Sinput}
\begin{Soutput}
     label    est    se      z pvalue ci.lower ci.upper
1       cp  0.055 0.232  0.238  0.811   -0.431    0.474
2       b1  0.427 0.252  1.693  0.090   -0.068    0.929
3       b2  0.763 0.204  3.735  0.000    0.334    1.151
4       a2  0.473 0.073  6.480  0.000    0.329    0.625
5       d1  0.689 0.090  7.652  0.000    0.509    0.870
6       d2  0.863 0.110  7.843  0.000    0.670    1.116
7       d3  0.400 0.065  6.139  0.000    0.257    0.528
8       a1  0.720 0.089  8.110  0.000    0.521    0.883
9       a2  0.473 0.073  6.480  0.000    0.329    0.625
10      a3  0.476 0.098  4.871  0.000    0.296    0.674
24    imm1  0.203 0.128  1.580  0.114   -0.013    0.518
25     ab2  0.361 0.108  3.342  0.001    0.164    0.596
26 fullIE1  0.019 0.034  0.551  0.582   -0.020    0.134
27 fullIE2  1.295 0.387  3.344  0.001    0.562    2.102
28  mmTest -1.276 0.388 -3.291  0.001   -2.082   -0.547
\end{Soutput}
\end{Schunk}


\pagebreak

\begin{Schunk}
\begin{Sinput}
 par(family = "serif", cex = 0.75)
 library(rockchalk)
 ## First we need to create a 'plotSlopes' object:
 plotOut <- plotSlopes(model = out1,
                       plotx = "ratioC",
                       modx = "perceptionC",
                       plotPoints = FALSE)
 ## Then we modify 'plotOut' to get the J-N test:
 testOut <- testSlopes(plotOut)
\end{Sinput}
\begin{Soutput}
Values of perceptionC OUTSIDE this interval:
      lo       hi 
1.327479 2.452667 
cause the slope of (b1 + b2*perceptionC)ratioC to be statistically significant
\end{Soutput}
\end{Schunk}


\pagebreak

\begin{Schunk}
\begin{Sinput}
 ## Construct product terms to facilitate J-N technique:
 dat1$openXneuro <- with(dat1, neuro*open)
 dat1$concXneuro <- with(dat1, neuro*conc)
 dat1$openXconc <- with(dat1, open*conc)
 dat1$openXconcXneuro <- with(dat1, open*conc*neuro)
\end{Sinput}
\end{Schunk}


\pagebreak

\begin{Schunk}
\begin{Sinput}
 ## Calculate the percentile CI:
 lb <- sort(abVec)[0.025 * nSams]
 ub <- sort(abVec)[0.975 * nSams]
 c(lb, ub)
\end{Sinput}
\begin{Soutput}
[1] 0.08845936 0.29432389
\end{Soutput}
\end{Schunk}


\end{frame}



\begin{frame}[allowframebreaks]{Example}
  
\begin{Schunk}
\begin{Sinput}
 par(family = "serif", cex = 0.75)
 plotOut1.5 <- plotSlopes(model = out1.5,
                          plotx = "openXconc",
                          modx = "neuro",
                          plotPoints = FALSE,
                          modxVals = 
                          quantile(dat1$neuro,
                                   c(0.25, 0.5, 0.75),
                                   na.rm = TRUE)
                          )
\end{Sinput}
\end{Schunk}

\includegraphics{sweaveFiles/-017}

\pagebreak

\begin{Schunk}
\begin{Sinput}
 par(family = "serif", cex = 0.75)
 testOut1.5 <- testSlopes(plotOut1.5)
\end{Sinput}
\begin{Soutput}
Values of neuro OUTSIDE this interval:
      lo       hi 
3.343832 7.745048 
cause the slope of (b1 + b2*neuro)openXconc to be statistically significant
\end{Soutput}
\begin{Sinput}
 plot(testOut1.5)
\end{Sinput}
\end{Schunk}

\includegraphics{sweaveFiles/-018}

\end{frame}



\begin{frame}{Categorical Variable Moderation}
  
  When the moderator is a categorical variable, moderation implies
  between-group differences in the focal effect.  
  \va
  \begin{itemize}
    \item This simplifies probing considerably
      \vb
    \item The simple slopes are given (almost) directly in the output
  \end{itemize}
  \va
  Recall the simple slope formula:
  \begin{align*}
    SS = \beta_1 + \beta_3Z
  \end{align*}
  Because $Z$ is a dummy code, this formula reduces to:
  \begin{align*}
    SS &= \beta_1, \text{ or}\\
    SS &= \beta_1 + \beta_3
  \end{align*}

\end{frame}



\begin{frame}[allowframebreaks]{Example}

\begin{Schunk}
\begin{Sinput}
 ## Marginal focal effect:
 out2.1 <- lm(conc ~ neuro, data = dat1)
 summary(out2.1)
\end{Sinput}
\begin{Soutput}
Call:
lm(formula = conc ~ neuro, data = dat1)

Residuals:
     Min       1Q   Median       3Q      Max 
-2.55547 -0.33353  0.00824  0.36098  1.85381 

Coefficients:
            Estimate Std. Error t value Pr(>|t|)    
(Intercept) 3.437327   0.029659  115.90   <2e-16 ***
neuro       0.118144   0.008844   13.36   <2e-16 ***
---
Signif. codes:  0 ‘***’ 0.001 ‘**’ 0.01 ‘*’ 0.05 ‘.’ 0.1 ‘ ’ 1

Residual standard error: 0.533 on 2550 degrees of freedom
Multiple R-squared:  0.0654,	Adjusted R-squared:  0.06504 
F-statistic: 178.4 on 1 and 2550 DF,  p-value: < 2.2e-16
\end{Soutput}
\begin{Sinput}
 ## Moderated by highest education attained:
 out2.2 <- lm(conc ~ neuro*educ, data = dat1)
 summary(out2.2)
\end{Sinput}
\begin{Soutput}
Call:
lm(formula = conc ~ neuro * educ, data = dat1)

Residuals:
     Min       1Q   Median       3Q      Max 
-2.52324 -0.34119  0.01457  0.36247  1.86213 

Coefficients:
            Estimate Std. Error t value Pr(>|t|)    
(Intercept)  3.72924    0.10864  34.326  < 2e-16 ***
neuro        0.01259    0.03156   0.399 0.689990    
educ2       -0.32892    0.11497  -2.861 0.004258 ** 
educ3       -0.30738    0.12102  -2.540 0.011146 *  
neuro:educ2  0.11033    0.03346   3.297 0.000990 ***
neuro:educ3  0.12755    0.03552   3.591 0.000336 ***
---
Signif. codes:  0 ‘***’ 0.001 ‘**’ 0.01 ‘*’ 0.05 ‘.’ 0.1 ‘ ’ 1

Residual standard error: 0.5308 on 2546 degrees of freedom
Multiple R-squared:  0.0746,	Adjusted R-squared:  0.07278 
F-statistic: 41.05 on 5 and 2546 DF,  p-value: < 2.2e-16
\end{Soutput}
\begin{Sinput}
 ## Test for omnibus moderation:
 anova(out2.1, out2.2)
\end{Sinput}
\begin{Soutput}
Analysis of Variance Table

Model 1: conc ~ neuro
Model 2: conc ~ neuro * educ
  Res.Df    RSS Df Sum of Sq      F    Pr(>F)    
1   2550 724.47                                  
2   2546 717.35  4    7.1285 6.3251 4.617e-05 ***
---
Signif. codes:  0 ‘***’ 0.001 ‘**’ 0.01 ‘*’ 0.05 ‘.’ 0.1 ‘ ’ 1
\end{Soutput}
\end{Schunk}


\pagebreak

\begin{Schunk}
\begin{Sinput}
 par(family = "serif", cex = 0.75)
 plotSlopes(out2.2,
            plotx = "neuro",
            modx = "educ",
            plotPoints = FALSE)
\end{Sinput}
\end{Schunk}

\includegraphics{sweaveFiles/-020}

\pagebreak

\begin{Schunk}
\begin{Sinput}
 ## Compute simple slopes by hand:
 ssSubHS <- coef(out2.2)[2]
 ssHighSchool <- sum(coef(out2.2)[c(2, 5)])
 ssCollege <- sum(coef(out2.2)[c(2, 6)])
 ## Compute simple slopes using centering:
 dat1$educ2 <- relevel(dat1$educ, ref = "highSchool")
 dat1$educ3 <- relevel(dat1$educ, ref = "college")
 out2.3 <- lm(conc ~ neuro*educ2, data = dat1)
 out2.4 <- lm(conc ~ neuro*educ3, data = dat1)
\end{Sinput}
\end{Schunk}


\pagebreak

\begin{Schunk}
\begin{Sinput}
 ## By hand:
 ssSubHS
\end{Sinput}
\begin{Soutput}
    neuro 
0.0125915 
\end{Soutput}
\begin{Sinput}
 ## By centering:
 as.matrix(coef(out2.2))
\end{Sinput}
\begin{Soutput}
                  [,1]
(Intercept)  3.7292366
neuro        0.0125915
educ2       -0.3289176
educ3       -0.3073786
neuro:educ2  0.1103337
neuro:educ3  0.1275497
\end{Soutput}
\end{Schunk}


\pagebreak

\begin{Schunk}
\begin{Sinput}
 ## By hand:
 ssHighSchool
\end{Sinput}
\begin{Soutput}
[1] 0.1229252
\end{Soutput}
\begin{Sinput}
 ## By centering:
 as.matrix(coef(out2.3))
\end{Sinput}
\begin{Soutput}
                          [,1]
(Intercept)         3.40031894
neuro               0.12292519
educ2subHS          0.32891761
educ2college        0.02153898
neuro:educ2subHS   -0.11033369
neuro:educ2college  0.01721601
\end{Soutput}
\end{Schunk}


\pagebreak

\begin{Schunk}
\begin{Sinput}
 ## By hand:
 ssCollege
\end{Sinput}
\begin{Soutput}
[1] 0.1401412
\end{Soutput}
\begin{Sinput}
 ## By centering:
 as.matrix(coef(out2.4))
\end{Sinput}
\begin{Soutput}
                             [,1]
(Intercept)            3.42185792
neuro                  0.14014120
educ3subHS             0.30737863
educ3highSchool       -0.02153898
neuro:educ3subHS      -0.12754971
neuro:educ3highSchool -0.01721601
\end{Soutput}
\end{Schunk}


\end{frame}



\begin{frame}[shrink = 10]{Example}

\begin{Schunk}
\begin{Sinput}
 summary(out2.2)
\end{Sinput}
\begin{Soutput}
Call:
lm(formula = conc ~ neuro * educ, data = dat1)

Residuals:
     Min       1Q   Median       3Q      Max 
-2.52324 -0.34119  0.01457  0.36247  1.86213 

Coefficients:
            Estimate Std. Error t value Pr(>|t|)    
(Intercept)  3.72924    0.10864  34.326  < 2e-16 ***
neuro        0.01259    0.03156   0.399 0.689990    
educ2       -0.32892    0.11497  -2.861 0.004258 ** 
educ3       -0.30738    0.12102  -2.540 0.011146 *  
neuro:educ2  0.11033    0.03346   3.297 0.000990 ***
neuro:educ3  0.12755    0.03552   3.591 0.000336 ***
---
Signif. codes:  0 ‘***’ 0.001 ‘**’ 0.01 ‘*’ 0.05 ‘.’ 0.1 ‘ ’ 1

Residual standard error: 0.5308 on 2546 degrees of freedom
Multiple R-squared:  0.0746,	Adjusted R-squared:  0.07278 
F-statistic: 41.05 on 5 and 2546 DF,  p-value: < 2.2e-16
\end{Soutput}
\end{Schunk}


\end{frame}



\begin{frame}[shrink = 10]{Example}

\begin{Schunk}
\begin{Sinput}
 summary(out2.3)
\end{Sinput}
\begin{Soutput}
Call:
lm(formula = conc ~ neuro * educ2, data = dat1)

Residuals:
     Min       1Q   Median       3Q      Max 
-2.52324 -0.34119  0.01457  0.36247  1.86213 

Coefficients:
                   Estimate Std. Error t value Pr(>|t|)    
(Intercept)         3.40032    0.03761  90.401  < 2e-16 ***
neuro               0.12293    0.01111  11.063  < 2e-16 ***
educ2subHS          0.32892    0.11497   2.861  0.00426 ** 
educ2college        0.02154    0.06525   0.330  0.74134    
neuro:educ2subHS   -0.11033    0.03346  -3.297  0.00099 ***
neuro:educ2college  0.01722    0.01972   0.873  0.38277    
---
Signif. codes:  0 ‘***’ 0.001 ‘**’ 0.01 ‘*’ 0.05 ‘.’ 0.1 ‘ ’ 1

Residual standard error: 0.5308 on 2546 degrees of freedom
Multiple R-squared:  0.0746,	Adjusted R-squared:  0.07278 
F-statistic: 41.05 on 5 and 2546 DF,  p-value: < 2.2e-16
\end{Soutput}
\end{Schunk}


\end{frame}



\begin{frame}[shrink = 10]{Example}

\begin{Schunk}
\begin{Sinput}
 summary(out2.4)
\end{Sinput}
\begin{Soutput}
Call:
lm(formula = conc ~ neuro * educ3, data = dat1)

Residuals:
     Min       1Q   Median       3Q      Max 
-2.52324 -0.34119  0.01457  0.36247  1.86213 

Coefficients:
                      Estimate Std. Error t value Pr(>|t|)    
(Intercept)            3.42186    0.05331  64.183  < 2e-16 ***
neuro                  0.14014    0.01629   8.601  < 2e-16 ***
educ3subHS             0.30738    0.12102   2.540 0.011146 *  
educ3highSchool       -0.02154    0.06525  -0.330 0.741340    
neuro:educ3subHS      -0.12755    0.03552  -3.591 0.000336 ***
neuro:educ3highSchool -0.01722    0.01972  -0.873 0.382770    
---
Signif. codes:  0 ‘***’ 0.001 ‘**’ 0.01 ‘*’ 0.05 ‘.’ 0.1 ‘ ’ 1

Residual standard error: 0.5308 on 2546 degrees of freedom
Multiple R-squared:  0.0746,	Adjusted R-squared:  0.07278 
F-statistic: 41.05 on 5 and 2546 DF,  p-value: < 2.2e-16
\end{Soutput}
\end{Schunk}


\end{frame}




\begin{frame}{Moderation via Multiple Group SEM}
  
  When our moderator is a categorical variable, we can use multiple
  group CFA/SEM to test for moderation.
  \va
  \begin{itemize}
    \item Categorical moderators define groups
      \vb
    \item Significant moderation with categorical moderators implies
      between-group differences in the focal effect
      \vb
    \item These hypotheses are easily tested with multiple group SEM
  \end{itemize}
  \va
  \begin{center}\textsc{Whiteboard Time!}\end{center}
  
\end{frame}



\begin{frame}[allowframebreaks]{Example}

\begin{Schunk}
\begin{Sinput}
 library(lavaan)
 library(semTools)
 dat2 <- readRDS("../data/bfiData2.rds")
 ## Multiple group moderation:
 mod1 <- "
 conc =~ C1 + C2 + C3 + C4 + C5
 neuro =~ N1 + N2 + N3 + N4 + N5
 "
\end{Sinput}
\end{Schunk}


\end{frame}



\begin{frame}[shrink = 10]{Example}
  
\begin{Schunk}
\begin{Sinput}
 fit1 <- measurementInvariance(mod1,
                               data = dat2,
                               group = "educ",
                               std.lv = TRUE)
\end{Sinput}
\begin{Soutput}
Measurement invariance models:

Model 1 : fit.configural
Model 2 : fit.loadings
Model 3 : fit.intercepts
Model 4 : fit.means

Chi Square Difference Test

                Df   AIC   BIC  Chisq Chisq diff Df diff Pr(>Chisq)    
fit.configural 102 85428 85971 1039.1                                  
fit.loadings   118 85427 85877 1070.0     30.927      16  0.0137462 *  
fit.intercepts 134 85456 85813 1131.4     61.399      16  3.037e-07 ***
fit.means      138 85468 85801 1150.8     19.324       4  0.0006788 ***
---
Signif. codes:  0 ‘***’ 0.001 ‘**’ 0.01 ‘*’ 0.05 ‘.’ 0.1 ‘ ’ 1


Fit measures:

                 cfi rmsea cfi.delta rmsea.delta
fit.configural 0.874 0.104        NA          NA
fit.loadings   0.871 0.097     0.002       0.007
fit.intercepts 0.865 0.094     0.006       0.004
fit.means      0.863 0.093     0.002       0.001
\end{Soutput}
\end{Schunk}


\end{frame}


\begin{frame}[allowframebreaks]{Example}

\begin{Schunk}
\begin{Sinput}
 mod2 <- "
 conc =~ C1 + C2 + C3 + C4 + C5
 neuro =~ N1 + N2 + N3 + N4 + N5
 
 conc ~ neuro
 
 conc ~~ c(1.0, NA, NA)*conc
 neuro ~~ c(1.0, NA, NA)*neuro
 
 conc ~ c(0.0, NA, NA)*1.0
 neuro ~ c(0.0, NA, NA)*1.0
 "
\end{Sinput}
\end{Schunk}


\pagebreak

\input{sweaveFiles/-031}

\pagebreak

\input{sweaveFiles/-032}

\pagebreak

\begin{Schunk}
\begin{Sinput}
 mod3 <- "
 conc =~ C1 + C2 + C3 + C4 + C5
 neuro =~ N1 + N2 + N3 + N4 + N5
 
 conc ~ c(b1, b1, b1)*neuro
 
 conc ~~ c(1.0, NA, NA)*conc
 neuro ~~ c(1.0, NA, NA)*neuro
 
 conc ~ c(0.0, NA, NA)*1.0
 neuro ~ c(0.0, NA, NA)*1.0
 "
\end{Sinput}
\end{Schunk}


\pagebreak

\begin{Schunk}
\begin{Sinput}
 fit3 <- lavaan(mod3,
                data = dat2,
                std.lv = FALSE,
                auto.fix.first = FALSE,
                auto.var = TRUE,
                int.ov.free = TRUE,
                group = "educ",
                group.equal = c("loadings", "intercepts")
                )
\end{Sinput}
\end{Schunk}


\pagebreak

\begin{Schunk}
\begin{Sinput}
 summary(fit3)
\end{Sinput}
\begin{Soutput}
lavaan (0.5-20) converged normally after  82 iterations

  Number of observations per group         
  highSchool                                      1536
  subHS                                            192
  college                                          824

  Estimator                                         ML
  Minimum Function Test Statistic             1133.785
  Degrees of freedom                               136
  P-value (Chi-square)                           0.000

Chi-square for each group:

  highSchool                                   574.222
  subHS                                        109.473
  college                                      450.090

Parameter Estimates:

  Information                                 Expected
  Standard Errors                             Standard


Group 1 [highSchool]:

Latent Variables:
                   Estimate  Std.Err  Z-value  P(>|z|)
  conc =~                                             
    C1      (.p1.)    0.575    0.027   21.516    0.000
    C2      (.p2.)    0.679    0.029   23.743    0.000
    C3      (.p3.)    0.635    0.028   22.707    0.000
    C4      (.p4.)   -0.898    0.031  -29.262    0.000
    C5      (.p5.)   -0.949    0.036  -26.367    0.000
  neuro =~                                            
    N1      (.p6.)    1.307    0.033   39.323    0.000
    N2      (.p7.)    1.249    0.032   38.734    0.000
    N3      (.p8.)    1.207    0.034   35.334    0.000
    N4      (.p9.)    0.910    0.034   26.987    0.000
    N5      (.10.)    0.839    0.035   24.010    0.000

Regressions:
                   Estimate  Std.Err  Z-value  P(>|z|)
  conc ~                                              
    neuro     (b1)   -0.330    0.029  -11.365    0.000

Intercepts:
                   Estimate  Std.Err  Z-value  P(>|z|)
    conc              0.000                           
    neuro             0.000                           
    C1      (.26.)    4.571    0.027  170.418    0.000
    C2      (.27.)    4.442    0.029  152.673    0.000
    C3      (.28.)    4.379    0.028  154.796    0.000
    C4      (.29.)    2.434    0.032   76.041    0.000
    C5      (.30.)    3.187    0.037   85.740    0.000
    N1      (.31.)    2.940    0.039   75.506    0.000
    N2      (.32.)    3.509    0.038   93.434    0.000
    N3      (.33.)    3.224    0.039   83.699    0.000
    N4      (.34.)    3.197    0.035   91.151    0.000
    N5      (.35.)    2.975    0.035   84.126    0.000

Variances:
                   Estimate  Std.Err  Z-value  P(>|z|)
    conc              1.000                           
    neuro             1.000                           
    C1                1.107    0.045   24.765    0.000
    C2                1.184    0.050   23.859    0.000
    C3                1.199    0.049   24.404    0.000
    C4                0.898    0.048   18.626    0.000
    C5                1.603    0.073   22.047    0.000
    N1                0.838    0.046   18.279    0.000
    N2                0.828    0.044   18.968    0.000
    N3                1.219    0.055   22.286    0.000
    N4                1.703    0.067   25.564    0.000
    N5                1.963    0.075   26.136    0.000


Group 2 [subHS]:

Latent Variables:
                   Estimate  Std.Err  Z-value  P(>|z|)
  conc =~                                             
    C1      (.p1.)    0.575    0.027   21.516    0.000
    C2      (.p2.)    0.679    0.029   23.743    0.000
    C3      (.p3.)    0.635    0.028   22.707    0.000
    C4      (.p4.)   -0.898    0.031  -29.262    0.000
    C5      (.p5.)   -0.949    0.036  -26.367    0.000
  neuro =~                                            
    N1      (.p6.)    1.307    0.033   39.323    0.000
    N2      (.p7.)    1.249    0.032   38.734    0.000
    N3      (.p8.)    1.207    0.034   35.334    0.000
    N4      (.p9.)    0.910    0.034   26.987    0.000
    N5      (.10.)    0.839    0.035   24.010    0.000

Regressions:
                   Estimate  Std.Err  Z-value  P(>|z|)
  conc ~                                              
    neuro     (b1)   -0.330    0.029  -11.365    0.000

Intercepts:
                   Estimate  Std.Err  Z-value  P(>|z|)
    conc             -0.259    0.095   -2.721    0.007
    neuro             0.016    0.081    0.201    0.841
    C1      (.26.)    4.571    0.027  170.418    0.000
    C2      (.27.)    4.442    0.029  152.673    0.000
    C3      (.28.)    4.379    0.028  154.796    0.000
    C4      (.29.)    2.434    0.032   76.041    0.000
    C5      (.30.)    3.187    0.037   85.740    0.000
    N1      (.31.)    2.940    0.039   75.506    0.000
    N2      (.32.)    3.509    0.038   93.434    0.000
    N3      (.33.)    3.224    0.039   83.699    0.000
    N4      (.34.)    3.197    0.035   91.151    0.000
    N5      (.35.)    2.975    0.035   84.126    0.000

Variances:
                   Estimate  Std.Err  Z-value  P(>|z|)
    conc              1.054    0.169    6.216    0.000
    neuro             0.896    0.119    7.551    0.000
    C1                1.264    0.142    8.917    0.000
    C2                1.272    0.148    8.569    0.000
    C3                1.434    0.162    8.851    0.000
    C4                1.196    0.160    7.477    0.000
    C5                1.742    0.217    8.026    0.000
    N1                1.013    0.145    6.978    0.000
    N2                1.242    0.161    7.705    0.000
    N3                1.312    0.165    7.951    0.000
    N4                1.507    0.169    8.893    0.000
    N5                2.045    0.221    9.236    0.000


Group 3 [college]:

Latent Variables:
                   Estimate  Std.Err  Z-value  P(>|z|)
  conc =~                                             
    C1      (.p1.)    0.575    0.027   21.516    0.000
    C2      (.p2.)    0.679    0.029   23.743    0.000
    C3      (.p3.)    0.635    0.028   22.707    0.000
    C4      (.p4.)   -0.898    0.031  -29.262    0.000
    C5      (.p5.)   -0.949    0.036  -26.367    0.000
  neuro =~                                            
    N1      (.p6.)    1.307    0.033   39.323    0.000
    N2      (.p7.)    1.249    0.032   38.734    0.000
    N3      (.p8.)    1.207    0.034   35.334    0.000
    N4      (.p9.)    0.910    0.034   26.987    0.000
    N5      (.10.)    0.839    0.035   24.010    0.000

Regressions:
                   Estimate  Std.Err  Z-value  P(>|z|)
  conc ~                                              
    neuro     (b1)   -0.330    0.029  -11.365    0.000

Intercepts:
                   Estimate  Std.Err  Z-value  P(>|z|)
    conc             -0.173    0.053   -3.239    0.001
    neuro            -0.092    0.045   -2.056    0.040
    C1      (.26.)    4.571    0.027  170.418    0.000
    C2      (.27.)    4.442    0.029  152.673    0.000
    C3      (.28.)    4.379    0.028  154.796    0.000
    C4      (.29.)    2.434    0.032   76.041    0.000
    C5      (.30.)    3.187    0.037   85.740    0.000
    N1      (.31.)    2.940    0.039   75.506    0.000
    N2      (.32.)    3.509    0.038   93.434    0.000
    N3      (.33.)    3.224    0.039   83.699    0.000
    N4      (.34.)    3.197    0.035   91.151    0.000
    N5      (.35.)    2.975    0.035   84.126    0.000

Variances:
                   Estimate  Std.Err  Z-value  P(>|z|)
    conc              1.132    0.097   11.640    0.000
    neuro             0.860    0.062   13.828    0.000
    C1                1.179    0.064   18.362    0.000
    C2                1.148    0.066   17.486    0.000
    C3                1.096    0.062   17.718    0.000
    C4                0.951    0.067   14.262    0.000
    C5                1.621    0.099   16.366    0.000
    N1                0.810    0.058   13.894    0.000
    N2                0.824    0.057   14.542    0.000
    N3                1.122    0.068   16.399    0.000
    N4                1.627    0.086   18.806    0.000
    N5                1.880    0.098   19.206    0.000
\end{Soutput}
\end{Schunk}


\pagebreak

\begin{Schunk}
\begin{Sinput}
 diffVec <- fitMeasures(fit3)[c("chisq", "df")] -
     fitMeasures(fit2)[c("chisq", "df")]
 pchisq(diffVec[1], diffVec[2], lower = FALSE)
\end{Sinput}
\begin{Soutput}
    chisq 
0.3093433 
\end{Soutput}
\end{Schunk}


\pagebreak

\input{sweaveFiles/-037}

\pagebreak

\begin{Schunk}
\begin{Sinput}
 fit4 <- lavaan(mod4,
                data = dat2,
                std.lv = FALSE,
                auto.fix.first = FALSE,
                auto.var = TRUE,
                int.ov.free = TRUE,
                group = "educ",
                group.equal = c("loadings", "intercepts")
                )
\end{Sinput}
\end{Schunk}


\pagebreak

\begin{Schunk}
\begin{Sinput}
 summary(fit4)
\end{Sinput}
\begin{Soutput}
lavaan (0.5-20) converged normally after  75 iterations

  Number of observations per group         
  highSchool                                      1536
  subHS                                            192
  college                                          824

  Estimator                                         ML
  Minimum Function Test Statistic             1132.387
  Degrees of freedom                               135
  P-value (Chi-square)                           0.000

Chi-square for each group:

  highSchool                                   573.494
  subHS                                        109.779
  college                                      449.114

Parameter Estimates:

  Information                                 Expected
  Standard Errors                             Standard


Group 1 [highSchool]:

Latent Variables:
                   Estimate  Std.Err  Z-value  P(>|z|)
  conc =~                                             
    C1      (.p1.)    0.574    0.027   21.493    0.000
    C2      (.p2.)    0.679    0.029   23.735    0.000
    C3      (.p3.)    0.635    0.028   22.691    0.000
    C4      (.p4.)   -0.897    0.031  -29.229    0.000
    C5      (.p5.)   -0.947    0.036  -26.305    0.000
  neuro =~                                            
    N1      (.p6.)    1.306    0.033   39.305    0.000
    N2      (.p7.)    1.248    0.032   38.715    0.000
    N3      (.p8.)    1.205    0.034   35.313    0.000
    N4      (.p9.)    0.909    0.034   26.975    0.000
    N5      (.10.)    0.837    0.035   23.992    0.000

Regressions:
                   Estimate  Std.Err  Z-value  P(>|z|)
  conc ~                                              
    neuro     (b1)   -0.349    0.034  -10.380    0.000

Intercepts:
                   Estimate  Std.Err  Z-value  P(>|z|)
    conc              0.000                           
    neuro             0.000                           
    C1      (.26.)    4.571    0.027  170.141    0.000
    C2      (.27.)    4.442    0.029  152.312    0.000
    C3      (.28.)    4.379    0.028  154.494    0.000
    C4      (.29.)    2.434    0.032   75.849    0.000
    C5      (.30.)    3.186    0.037   85.603    0.000
    N1      (.31.)    2.940    0.039   75.537    0.000
    N2      (.32.)    3.509    0.038   93.472    0.000
    N3      (.33.)    3.224    0.038   83.753    0.000
    N4      (.34.)    3.197    0.035   91.190    0.000
    N5      (.35.)    2.975    0.035   84.164    0.000

Variances:
                   Estimate  Std.Err  Z-value  P(>|z|)
    conc              1.000                           
    neuro             1.000                           
    C1                1.108    0.045   24.771    0.000
    C2                1.186    0.050   23.857    0.000
    C3                1.200    0.049   24.403    0.000
    C4                0.895    0.048   18.603    0.000
    C5                1.601    0.073   22.063    0.000
    N1                0.839    0.046   18.296    0.000
    N2                0.829    0.044   18.988    0.000
    N3                1.219    0.055   22.298    0.000
    N4                1.702    0.067   25.566    0.000
    N5                1.963    0.075   26.139    0.000


Group 2 [subHS]:

Latent Variables:
                   Estimate  Std.Err  Z-value  P(>|z|)
  conc =~                                             
    C1      (.p1.)    0.574    0.027   21.493    0.000
    C2      (.p2.)    0.679    0.029   23.735    0.000
    C3      (.p3.)    0.635    0.028   22.691    0.000
    C4      (.p4.)   -0.897    0.031  -29.229    0.000
    C5      (.p5.)   -0.947    0.036  -26.305    0.000
  neuro =~                                            
    N1      (.p6.)    1.306    0.033   39.305    0.000
    N2      (.p7.)    1.248    0.032   38.715    0.000
    N3      (.p8.)    1.205    0.034   35.313    0.000
    N4      (.p9.)    0.909    0.034   26.975    0.000
    N5      (.10.)    0.837    0.035   23.992    0.000

Regressions:
                   Estimate  Std.Err  Z-value  P(>|z|)
  conc ~                                              
    neuro     (b1)   -0.349    0.034  -10.380    0.000

Intercepts:
                   Estimate  Std.Err  Z-value  P(>|z|)
    conc             -0.259    0.095   -2.716    0.007
    neuro             0.016    0.081    0.201    0.841
    C1      (.26.)    4.571    0.027  170.141    0.000
    C2      (.27.)    4.442    0.029  152.312    0.000
    C3      (.28.)    4.379    0.028  154.494    0.000
    C4      (.29.)    2.434    0.032   75.849    0.000
    C5      (.30.)    3.186    0.037   85.603    0.000
    N1      (.31.)    2.940    0.039   75.537    0.000
    N2      (.32.)    3.509    0.038   93.472    0.000
    N3      (.33.)    3.224    0.038   83.753    0.000
    N4      (.34.)    3.197    0.035   91.190    0.000
    N5      (.35.)    2.975    0.035   84.164    0.000

Variances:
                   Estimate  Std.Err  Z-value  P(>|z|)
    conc              1.056    0.170    6.208    0.000
    neuro             0.896    0.119    7.550    0.000
    C1                1.264    0.142    8.916    0.000
    C2                1.272    0.148    8.564    0.000
    C3                1.433    0.162    8.848    0.000
    C4                1.197    0.160    7.476    0.000
    C5                1.742    0.217    8.030    0.000
    N1                1.012    0.145    6.980    0.000
    N2                1.243    0.161    7.710    0.000
    N3                1.314    0.165    7.957    0.000
    N4                1.507    0.169    8.895    0.000
    N5                2.046    0.221    9.238    0.000


Group 3 [college]:

Latent Variables:
                   Estimate  Std.Err  Z-value  P(>|z|)
  conc =~                                             
    C1      (.p1.)    0.574    0.027   21.493    0.000
    C2      (.p2.)    0.679    0.029   23.735    0.000
    C3      (.p3.)    0.635    0.028   22.691    0.000
    C4      (.p4.)   -0.897    0.031  -29.229    0.000
    C5      (.p5.)   -0.947    0.036  -26.305    0.000
  neuro =~                                            
    N1      (.p6.)    1.306    0.033   39.305    0.000
    N2      (.p7.)    1.248    0.032   38.715    0.000
    N3      (.p8.)    1.205    0.034   35.313    0.000
    N4      (.p9.)    0.909    0.034   26.975    0.000
    N5      (.10.)    0.837    0.035   23.992    0.000

Regressions:
                   Estimate  Std.Err  Z-value  P(>|z|)
  conc ~                                              
    neuro     (b2)   -0.277    0.052   -5.348    0.000

Intercepts:
                   Estimate  Std.Err  Z-value  P(>|z|)
    conc             -0.168    0.053   -3.137    0.002
    neuro            -0.092    0.045   -2.056    0.040
    C1      (.26.)    4.571    0.027  170.141    0.000
    C2      (.27.)    4.442    0.029  152.312    0.000
    C3      (.28.)    4.379    0.028  154.494    0.000
    C4      (.29.)    2.434    0.032   75.849    0.000
    C5      (.30.)    3.186    0.037   85.603    0.000
    N1      (.31.)    2.940    0.039   75.537    0.000
    N2      (.32.)    3.509    0.038   93.472    0.000
    N3      (.33.)    3.224    0.038   83.753    0.000
    N4      (.34.)    3.197    0.035   91.190    0.000
    N5      (.35.)    2.975    0.035   84.164    0.000

Variances:
                   Estimate  Std.Err  Z-value  P(>|z|)
    conc              1.138    0.098   11.640    0.000
    neuro             0.864    0.063   13.809    0.000
    C1                1.178    0.064   18.359    0.000
    C2                1.141    0.065   17.460    0.000
    C3                1.093    0.062   17.706    0.000
    C4                0.953    0.067   14.269    0.000
    C5                1.634    0.100   16.409    0.000
    N1                0.807    0.058   13.845    0.000
    N2                0.820    0.057   14.497    0.000
    N3                1.122    0.068   16.397    0.000
    N4                1.630    0.087   18.808    0.000
    N5                1.882    0.098   19.208    0.000
\end{Soutput}
\end{Schunk}


\pagebreak

\begin{Schunk}
\begin{Sinput}
 diffVec <- fitMeasures(fit4)[c("chisq", "df")] -
     fitMeasures(fit2)[c("chisq", "df")]
 pchisq(diffVec[1], diffVec[2], lower = FALSE)
\end{Sinput}
\begin{Soutput}
    chisq 
0.3299714 
\end{Soutput}
\end{Schunk}


\end{frame}



\begin{frame}{Probing Multiple Group Moderation}
  
  Several advantages to testing moderation with multiple group SEM
  \va
  \begin{itemize}
    \item Remove measurement error from the estimates
      \vb
    \item Test for factorial invariance
      \vb
    \item \textit{All information needed to plot/probe the simple
      slopes is contained directly in the output from the unrestricted
      model}
  \end{itemize}
  
\end{frame}



\begin{frame}[allowframebreaks]{Example}
  
\begin{Schunk}
\begin{Sinput}
 summary(fit2)
\end{Sinput}
\begin{Soutput}
lavaan (0.5-20) converged normally after  79 iterations

  Number of observations per group         
  highSchool                                      1536
  subHS                                            192
  college                                          824

  Estimator                                         ML
  Minimum Function Test Statistic             1131.438
  Degrees of freedom                               134
  P-value (Chi-square)                           0.000

Chi-square for each group:

  highSchool                                   573.289
  subHS                                        108.925
  college                                      449.224

Parameter Estimates:

  Information                                 Expected
  Standard Errors                             Standard


Group 1 [highSchool]:

Latent Variables:
                   Estimate  Std.Err  Z-value  P(>|z|)
  conc =~                                             
    C1      (.p1.)    0.573    0.027   21.471    0.000
    C2      (.p2.)    0.678    0.029   23.706    0.000
    C3      (.p3.)    0.634    0.028   22.666    0.000
    C4      (.p4.)   -0.897    0.031  -29.235    0.000
    C5      (.p5.)   -0.947    0.036  -26.307    0.000
  neuro =~                                            
    N1      (.p6.)    1.305    0.033   39.285    0.000
    N2      (.p7.)    1.247    0.032   38.701    0.000
    N3      (.p8.)    1.205    0.034   35.309    0.000
    N4      (.p9.)    0.909    0.034   26.982    0.000
    N5      (.10.)    0.837    0.035   23.998    0.000

Regressions:
                   Estimate  Std.Err  Z-value  P(>|z|)
  conc ~                                              
    neuro            -0.359    0.035  -10.208    0.000

Intercepts:
                   Estimate  Std.Err  Z-value  P(>|z|)
    conc              0.000                           
    neuro             0.000                           
    C1      (.26.)    4.571    0.027  170.017    0.000
    C2      (.27.)    4.442    0.029  152.180    0.000
    C3      (.28.)    4.379    0.028  154.381    0.000
    C4      (.29.)    2.434    0.032   75.721    0.000
    C5      (.30.)    3.186    0.037   85.494    0.000
    N1      (.31.)    2.940    0.039   75.566    0.000
    N2      (.32.)    3.509    0.038   93.494    0.000
    N3      (.33.)    3.224    0.038   83.772    0.000
    N4      (.34.)    3.197    0.035   91.203    0.000
    N5      (.35.)    2.975    0.035   84.171    0.000

Variances:
                   Estimate  Std.Err  Z-value  P(>|z|)
    conc              1.000                           
    neuro             1.000                           
    C1                1.109    0.045   24.781    0.000
    C2                1.187    0.050   23.872    0.000
    C3                1.201    0.049   24.414    0.000
    C4                0.893    0.048   18.572    0.000
    C5                1.600    0.073   22.053    0.000
    N1                0.840    0.046   18.316    0.000
    N2                0.830    0.044   19.003    0.000
    N3                1.219    0.055   22.298    0.000
    N4                1.701    0.067   25.564    0.000
    N5                1.962    0.075   26.138    0.000


Group 2 [subHS]:

Latent Variables:
                   Estimate  Std.Err  Z-value  P(>|z|)
  conc =~                                             
    C1      (.p1.)    0.573    0.027   21.471    0.000
    C2      (.p2.)    0.678    0.029   23.706    0.000
    C3      (.p3.)    0.634    0.028   22.666    0.000
    C4      (.p4.)   -0.897    0.031  -29.235    0.000
    C5      (.p5.)   -0.947    0.036  -26.307    0.000
  neuro =~                                            
    N1      (.p6.)    1.305    0.033   39.285    0.000
    N2      (.p7.)    1.247    0.032   38.701    0.000
    N3      (.p8.)    1.205    0.034   35.309    0.000
    N4      (.p9.)    0.909    0.034   26.982    0.000
    N5      (.10.)    0.837    0.035   23.998    0.000

Regressions:
                   Estimate  Std.Err  Z-value  P(>|z|)
  conc ~                                              
    neuro            -0.252    0.105   -2.396    0.017

Intercepts:
                   Estimate  Std.Err  Z-value  P(>|z|)
    conc             -0.261    0.095   -2.741    0.006
    neuro             0.016    0.081    0.202    0.840
    C1      (.26.)    4.571    0.027  170.017    0.000
    C2      (.27.)    4.442    0.029  152.180    0.000
    C3      (.28.)    4.379    0.028  154.381    0.000
    C4      (.29.)    2.434    0.032   75.721    0.000
    C5      (.30.)    3.186    0.037   85.494    0.000
    N1      (.31.)    2.940    0.039   75.566    0.000
    N2      (.32.)    3.509    0.038   93.494    0.000
    N3      (.33.)    3.224    0.038   83.772    0.000
    N4      (.34.)    3.197    0.035   91.203    0.000
    N5      (.35.)    2.975    0.035   84.171    0.000

Variances:
                   Estimate  Std.Err  Z-value  P(>|z|)
    conc              1.061    0.170    6.237    0.000
    neuro             0.905    0.120    7.540    0.000
    C1                1.263    0.142    8.919    0.000
    C2                1.270    0.148    8.568    0.000
    C3                1.437    0.162    8.855    0.000
    C4                1.192    0.160    7.466    0.000
    C5                1.748    0.217    8.039    0.000
    N1                1.014    0.146    6.965    0.000
    N2                1.236    0.161    7.681    0.000
    N3                1.310    0.165    7.937    0.000
    N4                1.507    0.170    8.888    0.000
    N5                2.042    0.221    9.232    0.000


Group 3 [college]:

Latent Variables:
                   Estimate  Std.Err  Z-value  P(>|z|)
  conc =~                                             
    C1      (.p1.)    0.573    0.027   21.471    0.000
    C2      (.p2.)    0.678    0.029   23.706    0.000
    C3      (.p3.)    0.634    0.028   22.666    0.000
    C4      (.p4.)   -0.897    0.031  -29.235    0.000
    C5      (.p5.)   -0.947    0.036  -26.307    0.000
  neuro =~                                            
    N1      (.p6.)    1.305    0.033   39.285    0.000
    N2      (.p7.)    1.247    0.032   38.701    0.000
    N3      (.p8.)    1.205    0.034   35.309    0.000
    N4      (.p9.)    0.909    0.034   26.982    0.000
    N5      (.10.)    0.837    0.035   23.998    0.000

Regressions:
                   Estimate  Std.Err  Z-value  P(>|z|)
  conc ~                                              
    neuro            -0.278    0.052   -5.354    0.000

Intercepts:
                   Estimate  Std.Err  Z-value  P(>|z|)
    conc             -0.168    0.053   -3.139    0.002
    neuro            -0.092    0.045   -2.056    0.040
    C1      (.26.)    4.571    0.027  170.017    0.000
    C2      (.27.)    4.442    0.029  152.180    0.000
    C3      (.28.)    4.379    0.028  154.381    0.000
    C4      (.29.)    2.434    0.032   75.721    0.000
    C5      (.30.)    3.186    0.037   85.494    0.000
    N1      (.31.)    2.940    0.039   75.566    0.000
    N2      (.32.)    3.509    0.038   93.494    0.000
    N3      (.33.)    3.224    0.038   83.772    0.000
    N4      (.34.)    3.197    0.035   91.203    0.000
    N5      (.35.)    2.975    0.035   84.171    0.000

Variances:
                   Estimate  Std.Err  Z-value  P(>|z|)
    conc              1.139    0.098   11.634    0.000
    neuro             0.865    0.063   13.807    0.000
    C1                1.178    0.064   18.364    0.000
    C2                1.142    0.065   17.467    0.000
    C3                1.093    0.062   17.713    0.000
    C4                0.952    0.067   14.255    0.000
    C5                1.633    0.100   16.405    0.000
    N1                0.807    0.058   13.850    0.000
    N2                0.820    0.057   14.498    0.000
    N3                1.122    0.068   16.394    0.000
    N4                1.630    0.087   18.807    0.000
    N5                1.882    0.098   19.207    0.000
\end{Soutput}
\end{Schunk}


\pagebreak

\begin{Schunk}
\begin{Sinput}
 ## Extract info needed to plot simple slopes:
 ints <- c(0,
           coef(fit2)[c("conc~1.g2",
                        "conc~1.g3")]
           )
 slopes <- coef(fit2)[c("conc~neuro",
                        "conc~neuro.g2",
                        "conc~neuro.g3")]
 fScores <- do.call(rbind, predict(fit2))
\end{Sinput}
\end{Schunk}


\pagebreak

\begin{Schunk}
\begin{Sinput}
 par(family = "serif", cex = 0.75)
 plot(y = fScores[ , "conc"],
      x = fScores[ , "neuro"],
      type = "n",
      main = "Latent Simple Slopes",
      xlab = "Neuroticism",
      ylab = "Conscientiousness")
 abline(a = ints[1], b = slopes[1])
 abline(a = ints[2], b = slopes[2], col = "red")
 abline(a = ints[3], b = slopes[3], col = "blue")
 legend(x = "topright",
        inset = 0.01,
        legend =
            c("High School",
              "College",
              "< High School"),
        col =
            c("black",
              "red",
              "blue"),
        lty = 1)
\end{Sinput}
\end{Schunk}

\includegraphics{sweaveFiles/-043}

\end{frame}


\end{document}
