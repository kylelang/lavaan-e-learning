\documentclass{beamer}
\usetheme{ttuStatsCamp}
\usefonttheme{serif}
\usepackage[T1]{fontenc}
\usepackage[utf8]{inputenc}
%\usepackage{mathptmx}
\usepackage{url}
\usepackage{graphicx}
\usepackage{setspace}
%\usepackage{esint}
\usepackage[natbibapa]{apacite}
\usepackage{color}
\usepackage{amsmath}
\usepackage{amsfonts}
%\usepackage{bm}
\usepackage{Sweavel}
\usepackage{listings}

\def\Sweavesize{\scriptsize}
\def\Rcolor{\color{black}}
%\def\Routcolor{\color{red}}
\def\Rcommentcolor{\color{violet}}
\def\Rbackground{\color[gray]{0.85}}
\def\Routbackground{\color[gray]{0.85}}

\lstset{tabsize=2, breaklines=true, style=Rstyle}



\newcommand{\red}[0]{\textcolor{red}}
%\newcommand{\violet}[0]{\textcolor{violet}}
\newcommand{\green}[0]{\textcolor{green}}
\newcommand{\blue}[0]{\textcolor{blue}}
\newcommand{\comment}[1]{}
\newcommand{\kfold}[0]{\emph{K}-fold cross-validation}
\newcommand{\va}[0]{\vspace{12pt}}
\newcommand{\vb}[0]{\vspace{6pt}}
\newcommand{\vc}[0]{\vspace{3pt}}
\newcommand{\vx}[1]{\vspace{#1pt}}

\title[Lecture 2]{Lecture 2: Simple Mediation Analysis}

\author{Kyle M. Lang}

\institute[TTU IMMAP]{
  Institute for Measurement, Methodology, Analysis \& Policy\\
  Texas Tech University\\
  Lubbock, TX
}

\date{2016 Stats Camp}


\begin{document}

\setkeys{Gin}{width=\textwidth}

\input{sweaveFiles/-001}


\begin{frame}[plain]
  
  \titlepage
  
\end{frame}


\begin{frame}{Outline}

  \begin{itemize}
  \item Little review of ordinary least squares (OLS) regression
    \vspace{12pt}
  \item \citet{baronKenny:1986} Causal Steps approach
    \vspace{12pt}
  \item The \citet{sobel:1982} Z test
  \end{itemize}

\end{frame}


\begin{frame}{A Wee Bit o' Regression}

\begin{align}
Y &= \alpha_1 + \beta + e_1\\
Y &= \alpha_2 + \beta_1X + \beta_2Z + e_2
\end{align}

\begin{figure}
  \includegraphics[width=\textwidth]{figures/mlrPathDiagram.pdf}
\end{figure}

\end{frame}


\begin{frame}[allowframebreaks]{Example}

\begin{Schunk}
\begin{Sinput}
 dat1 <- readRDS("../data/adamsKlpsScaleScore.rds")
 ## Partial out the mediator's effect:
 mod1 <- lm(policy ~ sysRac + polAffil, data = dat1)
 mod2 <- lm(sysRac ~ polAffil, data = dat1)
 summary(mod1)$coef
\end{Sinput}
\begin{Soutput}
              Estimate Std. Error   t value     Pr(>|t|)
(Intercept) 0.83265885 0.41246491 2.0187386 4.670428e-02
sysRac      0.72235878 0.11147514 6.4799987 5.930291e-09
polAffil    0.05121251 0.06998433 0.7317711 4.663450e-01
\end{Soutput}
\begin{Sinput}
 summary(mod2)$coef
\end{Sinput}
\begin{Soutput}
             Estimate Std. Error  t value     Pr(>|t|)
(Intercept) 2.6060451 0.28489391 9.147423 2.715546e-14
polAffil    0.2568494 0.06213495 4.133735 8.336022e-05
\end{Soutput}
\begin{Sinput}
 ## Extract important parameter estimates:
 a <- coef(mod2)["polAffil"]
 b <- coef(mod1)["sysRac"]
 ## Compute indirect effect:
 ieProd <- a * b
 ieProd
\end{Sinput}
\begin{Soutput}
 polAffil 
0.1855374 
\end{Soutput}
\begin{Sinput}
 ## Calculate Sobel's Z:
 seA <- sqrt(diag(vcov(mod2)))["polAffil"]
 seB <- sqrt(diag(vcov(mod1)))["sysRac"]
 sobelSE <- sqrt(b^2 * seA^2 + a^2 * seB^2)
 sobelZ <- ieProd / sobelSE
 sobelZ
\end{Sinput}
\begin{Soutput}
polAffil 
 3.48501 
\end{Soutput}
\begin{Sinput}
 sobelP <- 2 * pnorm(sobelZ, lower = FALSE)
 sobelP
\end{Sinput}
\begin{Soutput}
    polAffil 
0.0004921178 
\end{Soutput}
\begin{Sinput}
 sobelUB <- ieProd + 1.96 * sobelSE
 sobelLB <- ieProd - 1.96 * sobelSE
 ## 95% Sobel CI:
 c(sobelLB, sobelUB)
\end{Sinput}
\begin{Soutput}
  polAffil   polAffil 
0.08118957 0.28988525 
\end{Soutput}
\end{Schunk}


\begin{Schunk}
\begin{Sinput}
 parameterEstimates(out1, 
                    boot = "bca.simple")[-c(6 : 13), -c(1 : 3)]
\end{Sinput}
\begin{Soutput}
   label   est    se     z pvalue ci.lower ci.upper
1     cp 0.082 0.259 0.318  0.751   -0.408    0.641
2      b 1.390 0.215 6.465  0.000    0.939    1.786
3     a1 0.729 0.091 8.014  0.000    0.539    0.908
4     a2 0.641 0.089 7.183  0.000    0.469    0.818
5     a3 0.451 0.097 4.643  0.000    0.223    0.629
14   imm 0.628 0.169 3.706  0.000    0.296    0.980
\end{Soutput}
\end{Schunk}


\end{frame}


\begin{frame}{Path Diagrams}

\begin{figure}
\includegraphics[width=\textwidth]{figures/simpleMediationPathDiagram.pdf}
\end{figure}

\end{frame}


\begin{frame}{Necessary Equations}

  To get all the pieces of the preceding diagram, we'll need to fit
  three equations.

\begin{align}
Y &= i_1 + cX + e_1 \label{eq1}\\
Y &= i_2 + c'X + bM + e_2 \label{eq2}\\
M &= i_3 + aX + e_3 \label{eq3}
\end{align}

\begin{itemize}
\item Equation \ref{eq1} gives us the total effect ($c$).
  \vb
\item Equation \ref{eq2} gives us the direct effect ($c'$) and the
  partialled effect of the mediator on the outcome ($b$).
  \vb
\item Equation \ref{eq3} gives us the effect of the input on the
  outcome ($a$).
\end{itemize}

\end{frame}


\begin{frame}{More Complex Path Diagram}

\begin{figure}
\includegraphics[width=\textwidth]{figures/complexMediationPathDiagram.pdf}
\end{figure}

\end{frame}


\begin{frame}{Two Measures of Indirect Effect}

Indirect effects can be quantified in two different ways:
\begin{align}
IE_{diff} &= c - c'\\
IE_{prod} &= a \cdot b
\end{align}

$IE_{diff}$ and $IE_{prod}$ are equivalent in simple mediation.
\va
\begin{itemize}
\item Both give us information about the proportion of the total
  effect that is transmitted through the intermediary variable.
  \vb
\item $IE_{prod}$ provides a more direct representation of the
  actual pathway we're interested in testing.
  \vb
\item $IE_{diff}$ gets at our desired hypothesis indirectly.
\end{itemize}

\end{frame}


\begin{frame}{The \emph{Causal Steps Approach}}

  \citet[][p. 1176]{baronKenny:1986} describe three/four conditions
  as being sufficient to demonstrate statistical ``mediation.''
  \va
  \begin{enumerate}
  \item Variations in levels of the independent variable significantly
    account for variations in the presumed mediator (i.e., Path
    \emph{a}).
    \begin{itemize}
      \item Need a significant $a$ path.
    \end{itemize}
    \vb
  \item Variations in the mediator significantly account for variations
    in the dependent variable (i.e., Path \emph{b}).
    \begin{itemize}
    \item Need a significant $b$ path.
    \end{itemize}
    \vb
  \item When Paths \emph{a} and \emph{b} are controlled, a previously
    significant relation between the independent and dependent
    variables is no longer significant.
    \begin{itemize}
    \item Need a significant total effect
    \item The direct effect must be ``less'' than the total effect
    \end{itemize}
  \end{enumerate}

\end{frame}


\begin{frame}[allowframebreaks]{Example}

\begin{Schunk}
\begin{Sinput}
 mod3 <- "
 att3 ~ att2 + b2*conf2 + cp2*horn2
 att2 ~ att1 + b1*conf1 + cp1*horn1
 
 conf3 ~ conf2 + a2*horn2
 conf2 ~ conf1 + a1*horn1
 
 horn3 ~ horn2
 horn2 ~ horn1
 
 horn3 ~~ conf3 + att3
 conf3 ~~ att3
 
 horn2 ~~ conf2 + att2
 conf2 ~~ att2
 
 a1 == a2
 b1 == b2
 cp1 == cp2
 "
 out3 <- sem(mod3, data = dat1)
 summary(out3)
\end{Sinput}
\begin{Soutput}
lavaan (0.5-20) converged normally after  46 iterations

  Number of observations                           500

  Estimator                                         ML
  Minimum Function Test Statistic              294.220
  Degrees of freedom                                18
  P-value (Chi-square)                           0.000

Parameter Estimates:

  Information                                 Expected
  Standard Errors                             Standard

Regressions:
                   Estimate  Std.Err  Z-value  P(>|z|)
  att3 ~                                              
    att2              0.497    0.035   14.234    0.000
    conf2     (b2)    0.098    0.019    5.200    0.000
    horn2    (cp2)    0.083    0.072    1.157    0.247
  att2 ~                                              
    att1              0.530    0.040   13.345    0.000
    conf1     (b1)    0.098    0.019    5.200    0.000
    horn1    (cp1)    0.083    0.072    1.157    0.247
  conf3 ~                                             
    conf2             0.684    0.035   19.602    0.000
    horn2     (a2)    0.493    0.107    4.596    0.000
  conf2 ~                                             
    conf1             0.623    0.032   19.546    0.000
    horn1     (a1)    0.493    0.107    4.596    0.000
  horn3 ~                                             
    horn2             0.826    0.030   27.609    0.000
  horn2 ~                                             
    horn1             0.714    0.024   29.181    0.000

Covariances:
                   Estimate  Std.Err  Z-value  P(>|z|)
  conf3 ~~                                            
    horn3             1.016    0.155    6.556    0.000
  att3 ~~                                             
    horn3             0.322    0.093    3.483    0.000
    conf3             3.574    0.465    7.691    0.000
  conf2 ~~                                            
    horn2             0.836    0.124    6.721    0.000
  att2 ~~                                             
    horn2             0.273    0.083    3.289    0.001
    conf2             2.027    0.400    5.067    0.000

Variances:
                   Estimate  Std.Err  Z-value  P(>|z|)
    att3              6.019    0.381   15.811    0.000
    att2              6.041    0.382   15.811    0.000
    conf3            15.814    1.000   15.811    0.000
    conf2            12.570    0.795   15.811    0.000
    horn3             0.695    0.044   15.811    0.000
    horn2             0.560    0.035   15.811    0.000

Constraints:
                                               |Slack|
    a1 - (a2)                                    0.000
    b1 - (b2)                                    0.000
    cp1 - (cp2)                                  0.000
\end{Soutput}
\begin{Sinput}
 chiDiff <- fitMeasures(out3)["chisq"] -
     fitMeasures(out1)["chisq"]
 dfDiff <- fitMeasures(out3)["df"] -
     fitMeasures(out1)["df"]
 pchisq(chiDiff, dfDiff, lower = FALSE)
\end{Sinput}
\begin{Soutput}
     chisq 
0.02684148 
\end{Soutput}
\end{Schunk}


\pagebreak

\begin{Schunk}
\begin{Sinput}
 parameterEstimates(out2, 
                    boot = "bca.simple")[-c(7 : 18), -c(1 : 3)]
\end{Sinput}
\begin{Soutput}
   label    est    se      z pvalue ci.lower ci.upper
1    cp1 -0.103 0.190 -0.543  0.587   -0.473    0.259
2    cp2  1.099 0.204  5.380  0.000    0.639    1.472
3     b1  1.615 0.167  9.649  0.000    1.270    1.912
4     b2  0.381 0.173  2.206  0.027    0.034    0.720
5     b3  0.571 0.173  3.297  0.001    0.220    0.921
6      a  0.741 0.131  5.638  0.000    0.471    0.984
19   imm  0.424 0.153  2.773  0.006    0.174    0.781
\end{Soutput}
\end{Schunk}


\end{frame}


\begin{frame}{Sobel's Z}

  In the previous example, do we have a \emph{significant} indirect
  effect?
  \va
  \begin{itemize}
  \item The direct effect is substantially smaller than the total
    effect, but is the difference statistically significant?
    \vb
  \item \citet{sobel:1982} developed an asymptotic standard error
    for $IE_{prod}$ that we can use to assess this hypothesis.
  \end{itemize}

  \begin{align}
    SE_{sobel} &= \sqrt{a^2 \cdot SE_b^2 + b^2 \cdot SE_a^2}\\
    Z_{sobel} &= \frac{ab}{SE_{sobel}}\\
    95\% CI_{sobel} &= ab \pm 1.96 \cdot SE_{sobel}
  \end{align}

\end{frame}


\begin{frame}{Example}

\begin{Schunk}
\begin{Sinput}
 ## Completely Standardized:
 abCS <- (sdX * ab) / sdY
 abCS
\end{Sinput}
\begin{Soutput}
[1] 0.1345859
\end{Soutput}
\begin{Sinput}
 cPrimeCS <- (sdX * cPrime) / sdY
 cPrimeCS
\end{Sinput}
\begin{Soutput}
       cp 
0.1790413 
\end{Soutput}
\begin{Sinput}
 cCS <- abCS + cPrimeCS
 cCS
\end{Sinput}
\begin{Soutput}
       cp 
0.3136272 
\end{Soutput}
\end{Schunk}


\end{frame}


\begin{frame}[shrink = 5]{Alternative formulations}
  There are, at least, two alternative formulation of the Sobel SE.
  \vb
  \begin{itemize}
  \item The first is due to \citet{aroian:1947}:
    \begin{align}
      SE_{aroian} &= \sqrt{a^2 \cdot SE_b^2 + b^2 \cdot SE_a^2 + SE_a^2 \cdot SE_b^2}
    \end{align}
    %\vc
  \item The other is due to \citet{goodman:1960}:
    \begin{align}
      SE_{goodman} &= \sqrt{a^2 \cdot SE_b^2 + b^2 \cdot SE_a^2 - SE_a^2 \cdot SE_b^2}
    \end{align}
    %\vc
  \item The Goodman formulation is unbiased, but can lead to negative
    estimated SEs.
    \vb
  \item The Aroian formulation is recommended since it does
    not assume that $SE_a^2 \cdot SE_b^2$ asymptotically vanishes.
    \vb
  \item The Aroian and Sobel versions will probably perform
    equivalently with $N \geq 50$.
  \end{itemize}
  \vb
  \tiny{\textsc{Note:} The information on this slide was drawn from
  \url{http://quantpsy.org/sobel/sobel.htm}}
\end{frame}


\begin{frame}[allowframebreaks]{Example}

\begin{Schunk}
\begin{Sinput}
 sum(out3.1$fitted - out3.3$fitted)
\end{Sinput}
\begin{Soutput}
[1] -2.785328e-12
\end{Soutput}
\begin{Sinput}
 summary(out3.1)$r.squared
\end{Sinput}
\begin{Soutput}
[1] 0.9999439
\end{Soutput}
\begin{Sinput}
 summary(out3.3)$r.squared
\end{Sinput}
\begin{Soutput}
[1] 0.9999439
\end{Soutput}
\end{Schunk}


\pagebreak

\begin{Schunk}
\begin{Sinput}
 nSams <- 1000
 abVec <- rep(NA, nSams)
 for(i in 1 : nSams) {
     ## Resample the data:
     bootSam <- 
         dat1[sample(c(1 : nrow(dat1)), replace = TRUE), ]
     ## Fit the path model:
     bootOut <- sem(mod2, data = bootSam)
     ## Store the estimated indirect effect:
     abVec[i] <- prod(coef(bootOut)[c("a", "b")])
 }
\end{Sinput}
\end{Schunk}


\end{frame}


\begin{frame}{Compare Results}

  All three formulations give similar answers for this problem:

\begin{Schunk}
\begin{Sinput}
 ## Conditional process model with a, b, c paths moderated:
 mod4 <- "
 agree ~ b1*open + b2*consc + cp1*extra + cp2*neuro + 
         cp3*extraXneuro + b3*openXconsc
 open ~ a1*extra + a2*neuro + a3*extraXneuro
 
 cpLo  := cp1 + cp3*(-0.962268)
 cpMid := cp1 + cp3*(-0.162268)
 cpHi  := cp1 + cp3*0.837732
 
 abLoLo  := (a1 + a3*(-0.962268)) * (b1 + b3*(-0.4045))
 abLoMid := (a1 + a3*(-0.962268)) * (b1 + b3*(-0.0045))
 abLoHi  := (a1 + a3*(-0.962268)) * (b1 + b3*0.3955)
 
 abMidLo  := (a1 + a3*(-0.162268)) * (b1 + b3*(-0.4045))
 abMidMid := (a1 + a3*(-0.162268)) * (b1 + b3*(-0.0045))
 abMidHi  := (a1 + a3*(-0.162268)) * (b1 + b3*0.3955)
 
 abHiLo  := (a1 + a3*0.837732) * (b1 + b3*(-0.4045))
 abHiMid := (a1 + a3*0.837732) * (b1 + b3*(-0.0045))
 abHiHi  := (a1 + a3*0.837732) * (b1 + b3*0.3955)
 "
\end{Sinput}
\end{Schunk}


\end{frame}


\begin{frame}[allowframebreaks]{Example}

\begin{Schunk}
\begin{Sinput}
 parameterEstimates(out2.2, boot = bootType)[ , -c(1 : 3)]
\end{Sinput}
\begin{Soutput}
   label    est    se      z pvalue ci.lower ci.upper
1     cp  0.135 0.083  1.638  0.102   -0.031    0.286
2     b2  0.597 0.137  4.359  0.000    0.311    0.858
3     a1 -0.266 0.061 -4.353  0.000   -0.392   -0.148
4    d21 -0.367 0.098 -3.764  0.000   -0.546   -0.166
5         0.987 0.166  5.946  0.000    0.731    1.402
6         0.719 0.116  6.218  0.000    0.527    0.991
7         0.705 0.094  7.520  0.000    0.552    0.926
8         2.444 0.000     NA     NA    2.444    2.444
9 fullIE  0.058 0.025  2.318  0.020    0.019    0.123
\end{Soutput}
\end{Schunk}


\end{frame}


\begin{frame}{Practice}

\begin{Schunk}
\begin{Sinput}
 ## Test Differences between Indirect Effects
 ## in Serial Multiple Mediator Model (Method 1):
 mod2.3 <- "
 policy ~ cp*polAffil + b1*merit + b2*sysRac
 merit ~ a1*polAffil
 sysRac ~ a2*polAffil + d21*merit
 
 ab1 := a1*b1
 ab2 := a2*b2
 fullIE := a1*d21*b2
 totalIE := ab1 + ab2 + fullIE 
 
 fullIE == ab1
 fullIE == ab2
 "
 out2.3 <- 
     sem(mod2.3, data = dat1, se = "boot", boot = nBoot)
 summary(out2.3)
\end{Sinput}
\begin{Soutput}
lavaan (0.5-20) converged normally after 213 iterations

  Number of observations                            87

  Estimator                                         ML
  Minimum Function Test Statistic                1.334
  Degrees of freedom                                 2
  P-value (Chi-square)                           0.513

Parameter Estimates:

  Information                                 Observed
  Standard Errors                            Bootstrap
  Number of requested bootstrap draws             2500
  Number of successful bootstrap draws            2500

Regressions:
                   Estimate  Std.Err  Z-value  P(>|z|)
  policy ~                                            
    polAffil  (cp)    0.108    0.084    1.281    0.200
    merit     (b1)   -0.150    0.047   -3.183    0.001
    sysRac    (b2)    0.521    0.125    4.157    0.000
  merit ~                                             
    polAffil  (a1)   -0.271    0.057   -4.750    0.000
  sysRac ~                                            
    polAffil  (a2)    0.078    0.025    3.125    0.002
    merit    (d21)   -0.287    0.075   -3.814    0.000

Variances:
                   Estimate  Std.Err  Z-value  P(>|z|)
    policy            1.001    0.171    5.854    0.000
    merit             0.719    0.114    6.330    0.000
    sysRac            0.690    0.090    7.632    0.000

Defined Parameters:
                   Estimate  Std.Err  Z-value  P(>|z|)
    ab1               0.041    0.014    2.983    0.003
    ab2               0.041    0.014    2.983    0.003
    fullIE            0.041    0.014    2.983    0.003
    totalIE           0.122    0.041    2.983    0.003

Constraints:
                                               |Slack|
    fullIE - (ab1)                               0.000
    fullIE - (ab2)                               0.000
\end{Soutput}
\begin{Sinput}
 ## Conduct a chi-squared difference test:
 chiDiff <- fitMeasures(out2.3)["chisq"] - 
     fitMeasures(out2.1)["chisq"]
 dfDiff <- fitMeasures(out2.3)["df"] - 
     fitMeasures(out2.1)["df"]
 pchisq(chiDiff, dfDiff, lower = FALSE)
\end{Sinput}
\begin{Soutput}
    chisq 
0.5131246 
\end{Soutput}
\end{Schunk}


\begin{Schunk}
\begin{Sinput}
 ## Serial Multiple Mediator Model with 3 Mediators:
 mod3.1 <- "
 policy ~ b1*merit + b2*sysRac + b3*revDisc + cp*polAffil
 revDisc ~ d31*merit + d32*sysRac + a3*polAffil
 sysRac ~ d21*merit + a2*polAffil
 merit ~ a1*polAffil
 
 ab1 := a1*b1
 ab2 := a2*b2
 ab3 := a3*b3
 
 partIE1 := a1*d31*b3
 partIE2 := a1*d21*b2
 partIE3 := a2*d32*b3
 
 fullIE := a1*d21*d32*b3
 
 totalIE := ab1 + ab2 + ab3 + partIE1 + partIE2 + partIE3 + fullIE 
 "
 out3.1 <- 
     sem(mod3.1, data = dat1, se = "boot", boot = nBoot)
 summary(out3.1)
\end{Sinput}
\begin{Soutput}
lavaan (0.5-20) converged normally after  23 iterations

  Number of observations                            87

  Estimator                                         ML
  Minimum Function Test Statistic                0.000
  Degrees of freedom                                 0

Parameter Estimates:

  Information                                 Observed
  Standard Errors                            Bootstrap
  Number of requested bootstrap draws             2500
  Number of successful bootstrap draws            2498

Regressions:
                   Estimate  Std.Err  Z-value  P(>|z|)
  policy ~                                            
    merit     (b1)    0.005    0.144    0.035    0.972
    sysRac    (b2)    0.589    0.151    3.895    0.000
    revDisc   (b3)   -0.026    0.080   -0.330    0.741
    polAffil  (cp)    0.130    0.080    1.616    0.106
  revDisc ~                                           
    merit    (d31)    0.473    0.190    2.490    0.013
    sysRac   (d32)   -0.196    0.243   -0.806    0.420
    polAffil  (a3)   -0.149    0.131   -1.140    0.254
  sysRac ~                                            
    merit    (d21)   -0.301    0.109   -2.765    0.006
    polAffil  (a2)    0.090    0.071    1.270    0.204
  merit ~                                             
    polAffil  (a1)   -0.266    0.061   -4.340    0.000

Variances:
                   Estimate  Std.Err  Z-value  P(>|z|)
    policy            0.985    0.164    6.023    0.000
    revDisc           2.361    0.307    7.698    0.000
    sysRac            0.689    0.091    7.612    0.000
    merit             0.719    0.111    6.482    0.000

Defined Parameters:
                   Estimate  Std.Err  Z-value  P(>|z|)
    ab1              -0.001    0.040   -0.033    0.973
    ab2               0.053    0.043    1.224    0.221
    ab3               0.004    0.016    0.244    0.807
    partIE1           0.003    0.012    0.273    0.785
    partIE2           0.047    0.026    1.831    0.067
    partIE3           0.000    0.003    0.150    0.881
    fullIE            0.000    0.002    0.191    0.849
    totalIE           0.107    0.052    2.052    0.040
\end{Soutput}
\end{Schunk}


\end{frame}


\begin{frame}{References}
\bibliographystyle{apacite}
\bibliography{../../bibtexStuff/dissRefsList}
\end{frame}


\end{document}
