\documentclass{beamer}
\usetheme{ttuStatsCamp}
\usefonttheme{serif}
\usepackage[T1]{fontenc}
\usepackage[utf8]{inputenc}
\usepackage{url}
\usepackage{graphicx}
\usepackage{setspace}
\usepackage[natbibapa]{apacite}
\usepackage{color}
\usepackage{amsmath}
\usepackage{amsfonts}
\usepackage{Sweavel}
\usepackage{listings}
\usepackage{fancybox}

\def\Sweavesize{\scriptsize}
\def\Rcolor{\color{black}}
%\def\Routcolor{\color{red}}
\def\Rcommentcolor{\color{violet}}
\def\Rbackground{\color[gray]{0.85}}
\def\Routbackground{\color[gray]{0.85}}

\lstset{tabsize=2, breaklines=true, style=Rstyle}



\newcommand{\red}[0]{\textcolor{red}}
\newcommand{\green}[0]{\textcolor{green}}
\newcommand{\blue}[0]{\textcolor{blue}}
\newcommand{\comment}[1]{}
\newcommand{\va}[0]{\vspace{12pt}}
\newcommand{\vb}[0]{\vspace{6pt}}
\newcommand{\vc}[0]{\vspace{3pt}}
\newcommand{\vx}[1]{\vspace{#1pt}}

\title[Lecture 11]{Lecture 11: Inference in CPA}

\author{Kyle M. Lang}

\institute[TTU IMMAP]{
  Institute for Measurement, Methodology, Analysis \& Policy\\
  Texas Tech University\\
  Lubbock, TX
}

\date{2016 Stats Camp}

\setbeamertemplate{frametitle continuation}{}

\begin{document}

\setkeys{Gin}{width=\textwidth}

\input{sweaveFiles/-001}


\begin{frame}[plain]
  
  \titlepage
  
\end{frame}


\begin{frame}{Outline}

  \begin{itemize}
  \item Index of moderated mediation
    \va
  \item Examples of inference in conditional process models
    \va
  \item Inference when multiple paths are moderated
    \va
  \item General steps of inference in conditional process analysis
  \end{itemize}
  
\end{frame}



\begin{frame}{From Last Time}
  
  Last time, we saw how to calculate the conditional indirect effects
  in a model such as the following: 
  \vb
  \begin{figure}
    \includegraphics[width=0.95\textwidth]{figures/modAwithZConceptual.pdf}
  \end{figure}
  
\end{frame}



\begin{frame}{From Last Time}
  
  The preceding diagram corresponds to the following equations:
  \begin{align}
    Y &= i_1 + bM + c'X + e_Y\\
    M &= i_2 + a_1X + a_2Z + a_3XZ + e_M \label{eq2}
  \end{align}
  The indirect effect must be interpreted as conditional
  on $Z$ due to the $a$ path being moderated by $Z$.\\ 
  \va 
  We rearrange Equation \ref{eq2} to get:
  \begin{align*}
    M = i_2 + a_2Z + \left( a_1 + a_3Z \right)X + e_M,
  \end{align*}
  and the conditional indirect effect is defined by:
  \begin{align*}
    IE = \left(a_1 + a_3Z \right) b
  \end{align*}
  
\end{frame}


\begin{frame}{Index of Moderated Mediation}
  
  Probing conditional indirect effects is all well and good, but we'd
  like a single index to test the overall hypothesis of moderated
  mediation.
  \vb
  \begin{itemize}
    \item Enter the \emph{Index of Moderated Mediation} (IMM)
      introduced by \citet{hayes:2015}.
      \vb
    \item The IMM quantifies the linear effect of the moderator on the
      indirect effect.
      \vb
    \item When IMM is different from zero, we know that the indirect
      effect is moderated, generally.
  \end{itemize}
  
\end{frame}


\begin{frame}{Index of Moderated Mediation}
  
  Applying some basic algebra to the preceding conditional indirect
  effect formula produces:
  \begin{align*}
    \left(a_1 + a_3Z \right) b = a_1b + a_3bZ,
  \end{align*}
  which is a linear function describing the effect of $Z$ on the
  indirect effect
  \begin{itemize}
    \item $a_1b$ is the intercept term
    \item $a_3b$ is the slope linking $Z$ to the indirect effect
  \end{itemize}
  \vb
  The $a_3b$ term is the IMM.
  \vb
  \begin{itemize}
    \item We test $a_3b \neq 0$ to infer moderated mediation.
    \item Normal theory tests are possible, but we want to use
      bootstrapping.
  \end{itemize}
  
\end{frame}



\begin{frame}[allowframebreaks]{Example}
  
\begin{Schunk}
\begin{Sinput}
 dat1 <- readRDS("../data/adamsKlpsScaleScore.rds")
 ## Partial out the mediator's effect:
 mod1 <- lm(policy ~ sysRac + polAffil, data = dat1)
 mod2 <- lm(sysRac ~ polAffil, data = dat1)
 summary(mod1)$coef
\end{Sinput}
\begin{Soutput}
              Estimate Std. Error   t value     Pr(>|t|)
(Intercept) 0.83265885 0.41246491 2.0187386 4.670428e-02
sysRac      0.72235878 0.11147514 6.4799987 5.930291e-09
polAffil    0.05121251 0.06998433 0.7317711 4.663450e-01
\end{Soutput}
\begin{Sinput}
 summary(mod2)$coef
\end{Sinput}
\begin{Soutput}
             Estimate Std. Error  t value     Pr(>|t|)
(Intercept) 2.6060451 0.28489391 9.147423 2.715546e-14
polAffil    0.2568494 0.06213495 4.133735 8.336022e-05
\end{Soutput}
\begin{Sinput}
 ## Extract important parameter estimates:
 a <- coef(mod2)["polAffil"]
 b <- coef(mod1)["sysRac"]
 ## Compute indirect effect:
 ieProd <- a * b
 ieProd
\end{Sinput}
\begin{Soutput}
 polAffil 
0.1855374 
\end{Soutput}
\begin{Sinput}
 ## Calculate Sobel's Z:
 seA <- sqrt(diag(vcov(mod2)))["polAffil"]
 seB <- sqrt(diag(vcov(mod1)))["sysRac"]
 sobelSE <- sqrt(b^2 * seA^2 + a^2 * seB^2)
 sobelZ <- ieProd / sobelSE
 sobelZ
\end{Sinput}
\begin{Soutput}
polAffil 
 3.48501 
\end{Soutput}
\begin{Sinput}
 sobelP <- 2 * pnorm(sobelZ, lower = FALSE)
 sobelP
\end{Sinput}
\begin{Soutput}
    polAffil 
0.0004921178 
\end{Soutput}
\begin{Sinput}
 sobelUB <- ieProd + 1.96 * sobelSE
 sobelLB <- ieProd - 1.96 * sobelSE
 ## 95% Sobel CI:
 c(sobelLB, sobelUB)
\end{Sinput}
\begin{Soutput}
  polAffil   polAffil 
0.08118957 0.28988525 
\end{Soutput}
\end{Schunk}


\pagebreak

\begin{Schunk}
\begin{Sinput}
 parameterEstimates(out1, 
                    boot = "bca.simple")[-c(6 : 13), -c(1 : 3)]
\end{Sinput}
\begin{Soutput}
   label   est    se     z pvalue ci.lower ci.upper
1     cp 0.082 0.259 0.318  0.751   -0.408    0.641
2      b 1.390 0.215 6.465  0.000    0.939    1.786
3     a1 0.729 0.091 8.014  0.000    0.539    0.908
4     a2 0.641 0.089 7.183  0.000    0.469    0.818
5     a3 0.451 0.097 4.643  0.000    0.223    0.629
14   imm 0.628 0.169 3.706  0.000    0.296    0.980
\end{Soutput}
\end{Schunk}


\end{frame}



\begin{frame}{A Little Different}
  
  Moderation of the direct effect doesn't change the calculation of
  the IMM.\\
  \va
  Consider the following model:
  \begin{figure}
    \includegraphics[width = \textwidth]{figures/modBCwithWConceptual.pdf}
  \end{figure}
  
\end{frame}



\begin{frame}{A Little Different}
  
  This analytic diagram implies the following equations:
  \begin{align}
    Y &= i_1 + c_1'X + b_1M + b_2W + c_2'XW + b_3MW + e_Y \label{eq3}\\
    M &= i_2 + aX + e_M
  \end{align}
  The direct effect is conditional:
  \begin{align*}
    DE = c_1' + c_2'W
  \end{align*}
  The conditional indirect effect is defined by:
  \begin{align*}
    IE = a \left(b_1 + b_3W \right) = ab_1 + ab_3W,
  \end{align*}
  which implies $IMM = ab_3$
  
\end{frame}



\begin{frame}[allowframebreaks]{Example}
  
\begin{Schunk}
\begin{Sinput}
 mod3 <- "
 att3 ~ att2 + b2*conf2 + cp2*horn2
 att2 ~ att1 + b1*conf1 + cp1*horn1
 
 conf3 ~ conf2 + a2*horn2
 conf2 ~ conf1 + a1*horn1
 
 horn3 ~ horn2
 horn2 ~ horn1
 
 horn3 ~~ conf3 + att3
 conf3 ~~ att3
 
 horn2 ~~ conf2 + att2
 conf2 ~~ att2
 
 a1 == a2
 b1 == b2
 cp1 == cp2
 "
 out3 <- sem(mod3, data = dat1)
 summary(out3)
\end{Sinput}
\begin{Soutput}
lavaan (0.5-20) converged normally after  46 iterations

  Number of observations                           500

  Estimator                                         ML
  Minimum Function Test Statistic              294.220
  Degrees of freedom                                18
  P-value (Chi-square)                           0.000

Parameter Estimates:

  Information                                 Expected
  Standard Errors                             Standard

Regressions:
                   Estimate  Std.Err  Z-value  P(>|z|)
  att3 ~                                              
    att2              0.497    0.035   14.234    0.000
    conf2     (b2)    0.098    0.019    5.200    0.000
    horn2    (cp2)    0.083    0.072    1.157    0.247
  att2 ~                                              
    att1              0.530    0.040   13.345    0.000
    conf1     (b1)    0.098    0.019    5.200    0.000
    horn1    (cp1)    0.083    0.072    1.157    0.247
  conf3 ~                                             
    conf2             0.684    0.035   19.602    0.000
    horn2     (a2)    0.493    0.107    4.596    0.000
  conf2 ~                                             
    conf1             0.623    0.032   19.546    0.000
    horn1     (a1)    0.493    0.107    4.596    0.000
  horn3 ~                                             
    horn2             0.826    0.030   27.609    0.000
  horn2 ~                                             
    horn1             0.714    0.024   29.181    0.000

Covariances:
                   Estimate  Std.Err  Z-value  P(>|z|)
  conf3 ~~                                            
    horn3             1.016    0.155    6.556    0.000
  att3 ~~                                             
    horn3             0.322    0.093    3.483    0.000
    conf3             3.574    0.465    7.691    0.000
  conf2 ~~                                            
    horn2             0.836    0.124    6.721    0.000
  att2 ~~                                             
    horn2             0.273    0.083    3.289    0.001
    conf2             2.027    0.400    5.067    0.000

Variances:
                   Estimate  Std.Err  Z-value  P(>|z|)
    att3              6.019    0.381   15.811    0.000
    att2              6.041    0.382   15.811    0.000
    conf3            15.814    1.000   15.811    0.000
    conf2            12.570    0.795   15.811    0.000
    horn3             0.695    0.044   15.811    0.000
    horn2             0.560    0.035   15.811    0.000

Constraints:
                                               |Slack|
    a1 - (a2)                                    0.000
    b1 - (b2)                                    0.000
    cp1 - (cp2)                                  0.000
\end{Soutput}
\begin{Sinput}
 chiDiff <- fitMeasures(out3)["chisq"] -
     fitMeasures(out1)["chisq"]
 dfDiff <- fitMeasures(out3)["df"] -
     fitMeasures(out1)["df"]
 pchisq(chiDiff, dfDiff, lower = FALSE)
\end{Sinput}
\begin{Soutput}
     chisq 
0.02684148 
\end{Soutput}
\end{Schunk}


\pagebreak

\begin{Schunk}
\begin{Sinput}
 parameterEstimates(out2, 
                    boot = "bca.simple")[-c(7 : 18), -c(1 : 3)]
\end{Sinput}
\begin{Soutput}
   label    est    se      z pvalue ci.lower ci.upper
1    cp1 -0.103 0.190 -0.543  0.587   -0.473    0.259
2    cp2  1.099 0.204  5.380  0.000    0.639    1.472
3     b1  1.615 0.167  9.649  0.000    1.270    1.912
4     b2  0.381 0.173  2.206  0.027    0.034    0.720
5     b3  0.571 0.173  3.297  0.001    0.220    0.921
6      a  0.741 0.131  5.638  0.000    0.471    0.984
19   imm  0.424 0.153  2.773  0.006    0.174    0.781
\end{Soutput}
\end{Schunk}


\end{frame}



\begin{frame}{Getting More Complicated}
  
  There's no reason that our baseline mediation model needs to be a
  simple, three-variable model.\\ 
  \va 
  Consider the following model:
  \begin{figure}
    \includegraphics[width = \textwidth]{figures/serialMediatorModB2withWConceptual.pdf}
  \end{figure}
  
\end{frame}



\begin{frame}{Getting More Complicated}
  
  The preceding implies the following equations:
  \begin{align}
    Y &= i_1 + c'X + b_1M_1 + b_2M_2 + b_3W + b_4M_2W + e_Y \label{eq3}\\
    M_2 &= i_2 + a_2X + dM_1 + e_{M2}\\
    M_1 &= i_3 + a_1X + e_{M1}
  \end{align}
  We now have several specific indirect effects:
  \begin{align*}
    IE_1 &= a_1b_1\\
    IE_2 &= a_2 \left(b_2 + b_4W \right) = a_2b_2 + a_2b_4W\\
    IE_3 &= a_1 d \left(b_2 + b_4W \right) = a_1db_1 + a_1db_4W,
  \end{align*}
  which imply $IMM_2 = a_2b_4$ and $IMM_3 = a_1db_4$.
  
\end{frame}


  
\begin{frame}[allowframebreaks]{Example}
  
\begin{Schunk}
\begin{Sinput}
 ## Completely Standardized:
 abCS <- (sdX * ab) / sdY
 abCS
\end{Sinput}
\begin{Soutput}
[1] 0.1345859
\end{Soutput}
\begin{Sinput}
 cPrimeCS <- (sdX * cPrime) / sdY
 cPrimeCS
\end{Sinput}
\begin{Soutput}
       cp 
0.1790413 
\end{Soutput}
\begin{Sinput}
 cCS <- abCS + cPrimeCS
 cCS
\end{Sinput}
\begin{Soutput}
       cp 
0.3136272 
\end{Soutput}
\end{Schunk}


\pagebreak

\begin{Schunk}
\begin{Sinput}
 sum(out3.1$fitted - out3.3$fitted)
\end{Sinput}
\begin{Soutput}
[1] -2.785328e-12
\end{Soutput}
\begin{Sinput}
 summary(out3.1)$r.squared
\end{Sinput}
\begin{Soutput}
[1] 0.9999439
\end{Soutput}
\begin{Sinput}
 summary(out3.3)$r.squared
\end{Sinput}
\begin{Soutput}
[1] 0.9999439
\end{Soutput}
\end{Schunk}


\end{frame}



\begin{frame}{Limitations of IMM}
  
  The IMM is only well-defined for \emph{linear} relations between the
  moderator and the indirect effect.
  \vb
  \begin{itemize}
  \item Indirect effects with multiple constituent paths moderated
    imply non-linear relations between the moderators and the indirect
    effect.
  \end{itemize}
  \vb
  Consider this model:
  \begin{figure}
    \includegraphics[width = 0.75\textwidth]{figures/modABwithZConceptual.pdf}
  \end{figure}
  
\end{frame}


\begin{frame}{Limitations of IMM}
  
  The preceding model implies the following equations:
  \begin{align}
    Y &= i_1 + c'X + b_1M + b_2Z + b_2MZ + e_{Y}\\
    M &= i_2 + a_1X + a_2Z + a_3XZ + e_{M1},
  \end{align}
  so we have the following conditional indirect effect:
  \begin{align*}
    IE &= \left(a_1 + a_3Z\right) \left(b_1 + b_3Z \right)\\
    &= a_1b_1 + \left(a_1b_3 + a_3b_1\right)Z + a_3b_3Z^2
  \end{align*}
  which represents a quadratic linkage between the moderator and
  indirect effect.\\ 
  \va 
  With no single term describing the
  association of the moderator and the indirect effect, the
  \citet{hayes:2015} IMM isn't directly applicable.
  
\end{frame}
  

\begin{frame}{Alternative Strategy}
  
  When we have multiple moderated paths, we need to employ an
  alternative method discussed by \citet{edwardsLambert:2007} and
  \citet{wangPreacher:2015}, among others: 
  \vb
  \begin{enumerate}
  \item Fit our moderated mediation model, as usual
  \item Compute the conditional indirect effects at two values of the
    moderator
  \item Test for significant differences between these two conditional
    indirect effects.
  \end{enumerate}
  \vb
  Finding significant differences by this method suggests overall moderation.
  \begin{itemize}
    \item The converse does not hold
    \item A lack of significance does not imply no moderation
    \item Pairwise comparisons of conditional indirect effects are
      dependent on the values chosen for the moderator values
  \end{itemize}
  
\end{frame}



\begin{frame}[allowframebreaks]{Example}
    
\begin{Schunk}
\begin{Sinput}
 nSams <- 1000
 abVec <- rep(NA, nSams)
 for(i in 1 : nSams) {
     ## Resample the data:
     bootSam <- 
         dat1[sample(c(1 : nrow(dat1)), replace = TRUE), ]
     ## Fit the path model:
     bootOut <- sem(mod2, data = bootSam)
     ## Store the estimated indirect effect:
     abVec[i] <- prod(coef(bootOut)[c("a", "b")])
 }
\end{Sinput}
\end{Schunk}


\begin{Schunk}
\begin{Sinput}
 ## Conditional process model with a, b, c paths moderated:
 mod4 <- "
 agree ~ b1*open + b2*consc + cp1*extra + cp2*neuro + 
         cp3*extraXneuro + b3*openXconsc
 open ~ a1*extra + a2*neuro + a3*extraXneuro
 
 cpLo  := cp1 + cp3*(-0.962268)
 cpMid := cp1 + cp3*(-0.162268)
 cpHi  := cp1 + cp3*0.837732
 
 abLoLo  := (a1 + a3*(-0.962268)) * (b1 + b3*(-0.4045))
 abLoMid := (a1 + a3*(-0.962268)) * (b1 + b3*(-0.0045))
 abLoHi  := (a1 + a3*(-0.962268)) * (b1 + b3*0.3955)
 
 abMidLo  := (a1 + a3*(-0.162268)) * (b1 + b3*(-0.4045))
 abMidMid := (a1 + a3*(-0.162268)) * (b1 + b3*(-0.0045))
 abMidHi  := (a1 + a3*(-0.162268)) * (b1 + b3*0.3955)
 
 abHiLo  := (a1 + a3*0.837732) * (b1 + b3*(-0.4045))
 abHiMid := (a1 + a3*0.837732) * (b1 + b3*(-0.0045))
 abHiHi  := (a1 + a3*0.837732) * (b1 + b3*0.3955)
 "
\end{Sinput}
\end{Schunk}


\pagebreak

\begin{Schunk}
\begin{Sinput}
 parameterEstimates(out2.2, boot = bootType)[ , -c(1 : 3)]
\end{Sinput}
\begin{Soutput}
   label    est    se      z pvalue ci.lower ci.upper
1     cp  0.135 0.083  1.638  0.102   -0.031    0.286
2     b2  0.597 0.137  4.359  0.000    0.311    0.858
3     a1 -0.266 0.061 -4.353  0.000   -0.392   -0.148
4    d21 -0.367 0.098 -3.764  0.000   -0.546   -0.166
5         0.987 0.166  5.946  0.000    0.731    1.402
6         0.719 0.116  6.218  0.000    0.527    0.991
7         0.705 0.094  7.520  0.000    0.552    0.926
8         2.444 0.000     NA     NA    2.444    2.444
9 fullIE  0.058 0.025  2.318  0.020    0.019    0.123
\end{Soutput}
\end{Schunk}


\end{frame}



\begin{frame}{Limitations of IMM}
  
  The same issue arises when the indirect effects has multiple paths
  moderated by different variables.\\ 
  \va 
  Consider this model:
  \begin{figure}
    \includegraphics[width = \textwidth]{figures/modAwithZ_BwithWConceptual.pdf}
  \end{figure}
  
\end{frame}



\begin{frame}{Limitations of IMM}
  
  The preceding model implies the following equations:
  \begin{align}
    Y &= i_1 + c'X + b_1M + b_2W + b_2M_2W + e_{Y}\\
    M &= i_2 + a_1X + a_2Z + a_3XZ + e_{M},
  \end{align}
  so we have the following conditional indirect effect:
  \begin{align*}
    IE &= \left(a_1 + a_3Z\right) \left(b_1 + b_3W \right)\\
    &= a_1b_1 + a_1b_3W + a_3b_1Z + a_3b_3ZW
  \end{align*}
  which represents an interactive linkage between the moderator and
  indirect effect.\\ 
  \va 
  We still have no single term describing the
  association of the moderator and the indirect effect, so the
  \citet{hayes:2015} IMM still isn't directly applicable.
  
\end{frame}
  


\begin{frame}{More Complicated Model}
  
  Okay, let's put this all together into a pretty complicated
  conditional process model:\\ 
  \va 
  Consider this model:
  \begin{figure}
    \includegraphics[width = \textwidth]{figures/serialMediatorModA1DwithZConceptual.pdf}
  \end{figure}
  
\end{frame}



\begin{frame}{More Complicated Model}
  
  The preceding model implies the following equations:
  \begin{align}
    Y &= i_1 + c'X + b_1M + b_2M_2 + e_{Y}\\
    M_2 &= i_2 + a_2X + d_1M_1 + d_2Z + d_3M_1Z + e_{M2}\\
    M_1 &= i_3 + a_1X + a_2Z + a_3XZ,
  \end{align}
  so we have the following conditional indirect effects:
  \begin{align*}
    IE_1 &= \left(a_1 + a_3Z\right) b_1\\
    IE_2 &= a_2b_2\\
    IE_3 &= \left(a_1 + a_3Z\right) \left(d_1 + d_3Z\right) b_2
  \end{align*}
  We can employ multiple inferential strategies:
  \begin{enumerate}
    \item $IE_2$ is not moderated, so we can make direct inferences
    \item $IE_1$ only contains one moderated path, so we can make
      inferences via $IMM_1 = a_3b1$
    \item $IE_3$ contains two moderated paths, so we need to use the
      pairwise comparison approach.
  \end{enumerate}
  
\end{frame}


\begin{frame}[allowframebreaks]{Example}
  
\begin{Schunk}
\begin{Sinput}
 ## Test Differences between Indirect Effects
 ## in Serial Multiple Mediator Model (Method 1):
 mod2.3 <- "
 policy ~ cp*polAffil + b1*merit + b2*sysRac
 merit ~ a1*polAffil
 sysRac ~ a2*polAffil + d21*merit
 
 ab1 := a1*b1
 ab2 := a2*b2
 fullIE := a1*d21*b2
 totalIE := ab1 + ab2 + fullIE 
 
 fullIE == ab1
 fullIE == ab2
 "
 out2.3 <- 
     sem(mod2.3, data = dat1, se = "boot", boot = nBoot)
 summary(out2.3)
\end{Sinput}
\begin{Soutput}
lavaan (0.5-20) converged normally after 213 iterations

  Number of observations                            87

  Estimator                                         ML
  Minimum Function Test Statistic                1.334
  Degrees of freedom                                 2
  P-value (Chi-square)                           0.513

Parameter Estimates:

  Information                                 Observed
  Standard Errors                            Bootstrap
  Number of requested bootstrap draws             2500
  Number of successful bootstrap draws            2500

Regressions:
                   Estimate  Std.Err  Z-value  P(>|z|)
  policy ~                                            
    polAffil  (cp)    0.108    0.084    1.281    0.200
    merit     (b1)   -0.150    0.047   -3.183    0.001
    sysRac    (b2)    0.521    0.125    4.157    0.000
  merit ~                                             
    polAffil  (a1)   -0.271    0.057   -4.750    0.000
  sysRac ~                                            
    polAffil  (a2)    0.078    0.025    3.125    0.002
    merit    (d21)   -0.287    0.075   -3.814    0.000

Variances:
                   Estimate  Std.Err  Z-value  P(>|z|)
    policy            1.001    0.171    5.854    0.000
    merit             0.719    0.114    6.330    0.000
    sysRac            0.690    0.090    7.632    0.000

Defined Parameters:
                   Estimate  Std.Err  Z-value  P(>|z|)
    ab1               0.041    0.014    2.983    0.003
    ab2               0.041    0.014    2.983    0.003
    fullIE            0.041    0.014    2.983    0.003
    totalIE           0.122    0.041    2.983    0.003

Constraints:
                                               |Slack|
    fullIE - (ab1)                               0.000
    fullIE - (ab2)                               0.000
\end{Soutput}
\begin{Sinput}
 ## Conduct a chi-squared difference test:
 chiDiff <- fitMeasures(out2.3)["chisq"] - 
     fitMeasures(out2.1)["chisq"]
 dfDiff <- fitMeasures(out2.3)["df"] - 
     fitMeasures(out2.1)["df"]
 pchisq(chiDiff, dfDiff, lower = FALSE)
\end{Sinput}
\begin{Soutput}
    chisq 
0.5131246 
\end{Soutput}
\end{Schunk}


\pagebreak

\begin{Schunk}
\begin{Sinput}
 ## Serial Multiple Mediator Model with 3 Mediators:
 mod3.1 <- "
 policy ~ b1*merit + b2*sysRac + b3*revDisc + cp*polAffil
 revDisc ~ d31*merit + d32*sysRac + a3*polAffil
 sysRac ~ d21*merit + a2*polAffil
 merit ~ a1*polAffil
 
 ab1 := a1*b1
 ab2 := a2*b2
 ab3 := a3*b3
 
 partIE1 := a1*d31*b3
 partIE2 := a1*d21*b2
 partIE3 := a2*d32*b3
 
 fullIE := a1*d21*d32*b3
 
 totalIE := ab1 + ab2 + ab3 + partIE1 + partIE2 + partIE3 + fullIE 
 "
 out3.1 <- 
     sem(mod3.1, data = dat1, se = "boot", boot = nBoot)
 summary(out3.1)
\end{Sinput}
\begin{Soutput}
lavaan (0.5-20) converged normally after  23 iterations

  Number of observations                            87

  Estimator                                         ML
  Minimum Function Test Statistic                0.000
  Degrees of freedom                                 0

Parameter Estimates:

  Information                                 Observed
  Standard Errors                            Bootstrap
  Number of requested bootstrap draws             2500
  Number of successful bootstrap draws            2498

Regressions:
                   Estimate  Std.Err  Z-value  P(>|z|)
  policy ~                                            
    merit     (b1)    0.005    0.144    0.035    0.972
    sysRac    (b2)    0.589    0.151    3.895    0.000
    revDisc   (b3)   -0.026    0.080   -0.330    0.741
    polAffil  (cp)    0.130    0.080    1.616    0.106
  revDisc ~                                           
    merit    (d31)    0.473    0.190    2.490    0.013
    sysRac   (d32)   -0.196    0.243   -0.806    0.420
    polAffil  (a3)   -0.149    0.131   -1.140    0.254
  sysRac ~                                            
    merit    (d21)   -0.301    0.109   -2.765    0.006
    polAffil  (a2)    0.090    0.071    1.270    0.204
  merit ~                                             
    polAffil  (a1)   -0.266    0.061   -4.340    0.000

Variances:
                   Estimate  Std.Err  Z-value  P(>|z|)
    policy            0.985    0.164    6.023    0.000
    revDisc           2.361    0.307    7.698    0.000
    sysRac            0.689    0.091    7.612    0.000
    merit             0.719    0.111    6.482    0.000

Defined Parameters:
                   Estimate  Std.Err  Z-value  P(>|z|)
    ab1              -0.001    0.040   -0.033    0.973
    ab2               0.053    0.043    1.224    0.221
    ab3               0.004    0.016    0.244    0.807
    partIE1           0.003    0.012    0.273    0.785
    partIE2           0.047    0.026    1.831    0.067
    partIE3           0.000    0.003    0.150    0.881
    fullIE            0.000    0.002    0.191    0.849
    totalIE           0.107    0.052    2.052    0.040
\end{Soutput}
\end{Schunk}


\pagebreak

\begin{Schunk}
\begin{Sinput}
 parameterEstimates(out6, 
                    boot = "bca.simple")[-c(11 : 23), -c(1 : 3)]
\end{Sinput}
\begin{Soutput}
     label    est    se      z pvalue ci.lower ci.upper
1       cp  0.055 0.232  0.238  0.811   -0.431    0.474
2       b1  0.427 0.252  1.693  0.090   -0.068    0.929
3       b2  0.763 0.204  3.735  0.000    0.334    1.151
4       a2  0.473 0.073  6.480  0.000    0.329    0.625
5       d1  0.689 0.090  7.652  0.000    0.509    0.870
6       d2  0.863 0.110  7.843  0.000    0.670    1.116
7       d3  0.400 0.065  6.139  0.000    0.257    0.528
8       a1  0.720 0.089  8.110  0.000    0.521    0.883
9       a2  0.473 0.073  6.480  0.000    0.329    0.625
10      a3  0.476 0.098  4.871  0.000    0.296    0.674
24    imm1  0.203 0.128  1.580  0.114   -0.013    0.518
25     ab2  0.361 0.108  3.342  0.001    0.164    0.596
26 fullIE1  0.019 0.034  0.551  0.582   -0.020    0.134
27 fullIE2  1.295 0.387  3.344  0.001    0.562    2.102
28  mmTest -1.276 0.388 -3.291  0.001   -2.082   -0.547
\end{Soutput}
\end{Schunk}


\end{frame}


\begin{frame}[allowframebreaks]{General Steps for Conditional Process Analysis}
  
  At this point, we can lay out a few general steps that should help
  structure any conditional process analysis:
  \va
  \begin{enumerate}
  \item Draw a conceptual diagram representing your hypothesized process
    \vb
  \item Translate that conceptual diagram into an analytic diagram
    \vb
  \item Translate the analytic diagram into equations \label{eqStep}
    \vb
  \item Solve for all of your conditional indirect effects \label{hStep1}
    \vb
  \item Group your indirect effects into three categories:
    \begin{itemize}
    \item Not Moderated
    \item Linearly Moderated
    \item Non-Linearly Moderated
    \end{itemize}
    
    \pagebreak
    
  \item For linearly moderated conditional indirect effects, derive
    the appropriate IMMs
    \vb
  \item For non-linearly moderated conditional indirect effects,
    choose conditional values of the moderator(s) at which to test for
    differences in the conditional direct effects \label{hStep2}
    \vb
  \item Use the bootstrapping capabilities of path modeling software
    to fit the model implied by the equations from Step \ref{eqStep}
    and test the hypotheses implied by Steps \ref{hStep1} --
    \ref{hStep2}.
  \end{enumerate}
  
\end{frame}


\begin{frame}{References}

  \bibliographystyle{apacite}
  \bibliography{../../bibtexStuff/lecture11Refs.bib}

\end{frame}


\end{document}
