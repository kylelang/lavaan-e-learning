\documentclass{beamer}
\usetheme{ttuStatsCamp}
\usefonttheme{serif}
\usepackage[T1]{fontenc}
\usepackage[utf8]{inputenc}
\usepackage{url}
\usepackage{graphicx}
\usepackage{setspace}
\usepackage[natbibapa]{apacite}
\usepackage{color}
\usepackage{amsmath}
\usepackage{amsfonts}
\usepackage{Sweavel}
\usepackage{listings}
\usepackage{fancybox}

\def\Sweavesize{\scriptsize}
\def\Rcolor{\color{black}}
%\def\Routcolor{\color{red}}
\def\Rcommentcolor{\color{violet}}
\def\Rbackground{\color[gray]{0.85}}
\def\Routbackground{\color[gray]{0.85}}

\lstset{tabsize=2, breaklines=true, style=Rstyle}



\newcommand{\red}[0]{\textcolor{red}}
\newcommand{\green}[0]{\textcolor{green}}
\newcommand{\blue}[0]{\textcolor{blue}}
\newcommand{\comment}[1]{}
\newcommand{\va}[0]{\vspace{12pt}}
\newcommand{\vb}[0]{\vspace{6pt}}
\newcommand{\vc}[0]{\vspace{3pt}}
\newcommand{\vx}[1]{\vspace{#1pt}}

\title[Lecture 7]{Lecture 7: Basic Moderation}

\author{Kyle M. Lang}

\institute[TTU IMMAP]{
  Institute for Measurement, Methodology, Analysis \& Policy\\
  Texas Tech University\\
  Lubbock, TX
}

\date{2016 Stats Camp}

\setbeamertemplate{frametitle continuation}{}

\begin{document}

\setkeys{Gin}{width=\textwidth}

\input{sweaveFiles/-001}


\begin{frame}[plain]
  
  \titlepage
  
\end{frame}


\begin{frame}{Outline}

  \begin{itemize}
  \item Review the moderation hypothesis
    \va
  \item Doing basic moderation analysis
    \va
  \item Visualizing the moderation (a little)
    \va
  \item Probing the moderation (also a little)
  \end{itemize}

\end{frame}



\begin{frame}{Intuition}

  So far we've been discussing \emph{mediation}
  \vb
  \begin{itemize}
  \item Mediation allows us to ask \emph{how} one variable ($X$)
    affects another variable ($Y$).
    \vc
    \begin{itemize}
    \item Namely, through the intermediary influence of a third
      variable ($M$).
    \end{itemize}
  \end{itemize}
  \va
  Now, we're stepping into the realm of \emph{moderation}
  \vb
  \begin{itemize}
  \item Moderation allows us to ask \emph{when} one variable ($X$) affects
    another variable ($Y$).
    \vc
    \begin{itemize}
    \item Here, we're considering the effect of $X$ on $Y$
      conditional on certain levels of a third variable $Z$.
    \end{itemize}
  \end{itemize}

\end{frame}



\begin{frame}{Conceptual Diagram}

  We can diagrammatically represent the above intuition with:

  \begin{figure}
    \includegraphics[width=\textwidth]{figures/modConcept.pdf}
  \end{figure}

\end{frame}



\begin{frame}{Equations}

  In simple additive MLR, we might have the following equation:
  \begin{align}
    Y = \alpha + \beta_1X + \beta_2Z + e_i \label{additiveEq}
  \end{align}
  This additive equation assumes that $X$ and $Z$ are independent
  predictors of $Y$.\\
  \va
  When $X$ and $Z$ are independent predictors, the following
  points are true:
  \vb
  \begin{itemize}
  \item $X$ and $Z$ \emph{can} be correlated
    \vb
  \item $\beta_1$ and $\beta_2$ are \emph{partial} regression
    coefficients
    \vb
  \item \red{The effect of $X$ on $Y$ is the same at \textbf{all levels} of
    $Z$, and the effect of $Z$ on $Y$ is the same at \textbf{all
      levels} of $X$}
  \end{itemize}

\end{frame}



\begin{frame}{Equations}

  When testing moderation, we hypothesize that the effect of $X$ on
  $Y$ in Equation \ref{additiveEq} varies as a function of $Z$.\\
  \va
  We can represent this concept with the following equation:
  \begin{align}
    Y = \alpha + f(Z)X + \beta_2Z + e_i
  \end{align}
  \pause
  If we assume that $Z$ linearly affects the relationship between $X$
  and $Y$, then we can take:
  \begin{align}
    f(Z) = \beta_1 + \beta_3Z
  \end{align}
  \pause
  Which, after substitution, leads to:
  \begin{align}
    Y = \alpha + (\beta_1 + \beta_3Z)X + \beta_2Z + e_i
  \end{align}
  \pause
  Which, after distributing $X$ and reordering terms, becomes:
  \begin{align}
    Y = \alpha + \beta_1X + \beta_2Z + \beta_3XZ + e_i
  \end{align}

\end{frame}


\begin{frame}{Analytical Model}

  We can diagrammatically represent the analytical model we'll actually
  be fitting with:

  \begin{figure}
    \includegraphics[width=\textwidth]{figures/modAnalytic.pdf}
  \end{figure}

\end{frame}


\begin{frame}{Analytical Model}

  By adding the appropriate path labels, we get:

  \begin{figure}
    \includegraphics[width=\textwidth]{figures/modAnalytic2.pdf}
  \end{figure}

\end{frame}

\begin{frame}{Testing Moderation}

  This is the equation we'll be working with:\\
  \begin{center}\ovalbox{$Y = \alpha + \beta_1X + \beta_2Z + \beta_3XZ + e_i$}\end{center}
  \va
  Or, after fitting the above to some data:
  \begin{center}\ovalbox{$\hat{Y} = \hat{\alpha} + \hat{\beta}_1X + \hat{\beta}_2Z + \hat{\beta}_3XZ$}\end{center}
  \va
  To test for significant moderation, we simply need to see if
  $\hat{\beta}_3$ is significantly different from zero.\\
  \va
  We do so using simple linear regression modeling.

\end{frame}



\begin{frame}{Example}

Data from the \emph{National Longitudinal Survey of Youth}\\
\va
We suspect that participants' weight to height ratio is predictive of
their levels of depression.\\
\va
We further suspect that this effect may be differentially expressed
depending on how the participants perceive their own weight.

\end{frame}


\begin{frame}{Example}

  This is the conceptual diagram for the model we'll fit:

  \begin{figure}
    \includegraphics[width=\textwidth]{figures/modExample1.pdf}
  \end{figure}

\end{frame}


\begin{frame}[allowframebreaks]{Example}

\begin{Schunk}
\begin{Sinput}
 dat1 <- readRDS("../data/adamsKlpsScaleScore.rds")
 ## Partial out the mediator's effect:
 mod1 <- lm(policy ~ sysRac + polAffil, data = dat1)
 mod2 <- lm(sysRac ~ polAffil, data = dat1)
 summary(mod1)$coef
\end{Sinput}
\begin{Soutput}
              Estimate Std. Error   t value     Pr(>|t|)
(Intercept) 0.83265885 0.41246491 2.0187386 4.670428e-02
sysRac      0.72235878 0.11147514 6.4799987 5.930291e-09
polAffil    0.05121251 0.06998433 0.7317711 4.663450e-01
\end{Soutput}
\begin{Sinput}
 summary(mod2)$coef
\end{Sinput}
\begin{Soutput}
             Estimate Std. Error  t value     Pr(>|t|)
(Intercept) 2.6060451 0.28489391 9.147423 2.715546e-14
polAffil    0.2568494 0.06213495 4.133735 8.336022e-05
\end{Soutput}
\begin{Sinput}
 ## Extract important parameter estimates:
 a <- coef(mod2)["polAffil"]
 b <- coef(mod1)["sysRac"]
 ## Compute indirect effect:
 ieProd <- a * b
 ieProd
\end{Sinput}
\begin{Soutput}
 polAffil 
0.1855374 
\end{Soutput}
\begin{Sinput}
 ## Calculate Sobel's Z:
 seA <- sqrt(diag(vcov(mod2)))["polAffil"]
 seB <- sqrt(diag(vcov(mod1)))["sysRac"]
 sobelSE <- sqrt(b^2 * seA^2 + a^2 * seB^2)
 sobelZ <- ieProd / sobelSE
 sobelZ
\end{Sinput}
\begin{Soutput}
polAffil 
 3.48501 
\end{Soutput}
\begin{Sinput}
 sobelP <- 2 * pnorm(sobelZ, lower = FALSE)
 sobelP
\end{Sinput}
\begin{Soutput}
    polAffil 
0.0004921178 
\end{Soutput}
\begin{Sinput}
 sobelUB <- ieProd + 1.96 * sobelSE
 sobelLB <- ieProd - 1.96 * sobelSE
 ## 95% Sobel CI:
 c(sobelLB, sobelUB)
\end{Sinput}
\begin{Soutput}
  polAffil   polAffil 
0.08118957 0.28988525 
\end{Soutput}
\end{Schunk}


\begin{Schunk}
\begin{Sinput}
 parameterEstimates(out1, 
                    boot = "bca.simple")[-c(6 : 13), -c(1 : 3)]
\end{Sinput}
\begin{Soutput}
   label   est    se     z pvalue ci.lower ci.upper
1     cp 0.082 0.259 0.318  0.751   -0.408    0.641
2      b 1.390 0.215 6.465  0.000    0.939    1.786
3     a1 0.729 0.091 8.014  0.000    0.539    0.908
4     a2 0.641 0.089 7.183  0.000    0.469    0.818
5     a3 0.451 0.097 4.643  0.000    0.223    0.629
14   imm 0.628 0.169 3.706  0.000    0.296    0.980
\end{Soutput}
\end{Schunk}

\pagebreak
\begin{Schunk}
\begin{Sinput}
 mod3 <- "
 att3 ~ att2 + b2*conf2 + cp2*horn2
 att2 ~ att1 + b1*conf1 + cp1*horn1
 
 conf3 ~ conf2 + a2*horn2
 conf2 ~ conf1 + a1*horn1
 
 horn3 ~ horn2
 horn2 ~ horn1
 
 horn3 ~~ conf3 + att3
 conf3 ~~ att3
 
 horn2 ~~ conf2 + att2
 conf2 ~~ att2
 
 a1 == a2
 b1 == b2
 cp1 == cp2
 "
 out3 <- sem(mod3, data = dat1)
 summary(out3)
\end{Sinput}
\begin{Soutput}
lavaan (0.5-20) converged normally after  46 iterations

  Number of observations                           500

  Estimator                                         ML
  Minimum Function Test Statistic              294.220
  Degrees of freedom                                18
  P-value (Chi-square)                           0.000

Parameter Estimates:

  Information                                 Expected
  Standard Errors                             Standard

Regressions:
                   Estimate  Std.Err  Z-value  P(>|z|)
  att3 ~                                              
    att2              0.497    0.035   14.234    0.000
    conf2     (b2)    0.098    0.019    5.200    0.000
    horn2    (cp2)    0.083    0.072    1.157    0.247
  att2 ~                                              
    att1              0.530    0.040   13.345    0.000
    conf1     (b1)    0.098    0.019    5.200    0.000
    horn1    (cp1)    0.083    0.072    1.157    0.247
  conf3 ~                                             
    conf2             0.684    0.035   19.602    0.000
    horn2     (a2)    0.493    0.107    4.596    0.000
  conf2 ~                                             
    conf1             0.623    0.032   19.546    0.000
    horn1     (a1)    0.493    0.107    4.596    0.000
  horn3 ~                                             
    horn2             0.826    0.030   27.609    0.000
  horn2 ~                                             
    horn1             0.714    0.024   29.181    0.000

Covariances:
                   Estimate  Std.Err  Z-value  P(>|z|)
  conf3 ~~                                            
    horn3             1.016    0.155    6.556    0.000
  att3 ~~                                             
    horn3             0.322    0.093    3.483    0.000
    conf3             3.574    0.465    7.691    0.000
  conf2 ~~                                            
    horn2             0.836    0.124    6.721    0.000
  att2 ~~                                             
    horn2             0.273    0.083    3.289    0.001
    conf2             2.027    0.400    5.067    0.000

Variances:
                   Estimate  Std.Err  Z-value  P(>|z|)
    att3              6.019    0.381   15.811    0.000
    att2              6.041    0.382   15.811    0.000
    conf3            15.814    1.000   15.811    0.000
    conf2            12.570    0.795   15.811    0.000
    horn3             0.695    0.044   15.811    0.000
    horn2             0.560    0.035   15.811    0.000

Constraints:
                                               |Slack|
    a1 - (a2)                                    0.000
    b1 - (b2)                                    0.000
    cp1 - (cp2)                                  0.000
\end{Soutput}
\begin{Sinput}
 chiDiff <- fitMeasures(out3)["chisq"] -
     fitMeasures(out1)["chisq"]
 dfDiff <- fitMeasures(out3)["df"] -
     fitMeasures(out1)["df"]
 pchisq(chiDiff, dfDiff, lower = FALSE)
\end{Sinput}
\begin{Soutput}
     chisq 
0.02684148 
\end{Soutput}
\end{Schunk}

\pagebreak
\begin{Schunk}
\begin{Sinput}
 parameterEstimates(out2, 
                    boot = "bca.simple")[-c(7 : 18), -c(1 : 3)]
\end{Sinput}
\begin{Soutput}
   label    est    se      z pvalue ci.lower ci.upper
1    cp1 -0.103 0.190 -0.543  0.587   -0.473    0.259
2    cp2  1.099 0.204  5.380  0.000    0.639    1.472
3     b1  1.615 0.167  9.649  0.000    1.270    1.912
4     b2  0.381 0.173  2.206  0.027    0.034    0.720
5     b3  0.571 0.173  3.297  0.001    0.220    0.921
6      a  0.741 0.131  5.638  0.000    0.471    0.984
19   imm  0.424 0.153  2.773  0.006    0.174    0.781
\end{Soutput}
\end{Schunk}


\end{frame}



\begin{frame}{Visualizing the Interaction}

  We can get a better idea of the patterns of moderation by plotting
  the focal effect at conditional values of the moderator:\\
  \vb
\begin{Schunk}
\begin{Sinput}
 ## Completely Standardized:
 abCS <- (sdX * ab) / sdY
 abCS
\end{Sinput}
\begin{Soutput}
[1] 0.1345859
\end{Soutput}
\begin{Sinput}
 cPrimeCS <- (sdX * cPrime) / sdY
 cPrimeCS
\end{Sinput}
\begin{Soutput}
       cp 
0.1790413 
\end{Soutput}
\begin{Sinput}
 cCS <- abCS + cPrimeCS
 cCS
\end{Sinput}
\begin{Soutput}
       cp 
0.3136272 
\end{Soutput}
\end{Schunk}

\includegraphics{sweaveFiles/-006}

\end{frame}



\begin{frame}{Probing the Interaction}

  A significant estimate of $\beta_3$ tells us that the effect of $X$
  on $Y$ depends on the level of $Z$, but nothing more.\\
  \va
  The plot on the previous slide gives a descriptive illustration of the
  pattern, but does not support statistical inference.
  \vb
  \begin{itemize}
  \item The three conditional effects we plotted look different, but
    we cannot say that they differ in any meaningful way by only the
    plot and $\hat{\beta}_3$.
  \end{itemize}
  \va
  This is the purpose of \emph{probing} the interaction.
  \vb
  \begin{itemize}
  \item Try to isolate areas of $Z$'s distribution in which
      $\hat{\beta}_3$ is significant and areas where it is not.
  \end{itemize}

\end{frame}


\begin{frame}{Probing the Interaction}

  The most popular approach to probing the interaction is the
  \emph{pick-a-point} approach AKA \emph{simple slopes analysis} or
  \emph{spotlight analysis}.\\
  \va
  The pick-a-point approach tests if the slopes of the conditional
  effects plotted above are
  significantly different from zero.\\
  \va
  To do so, pick-a-point tests the significance of \emph{simple
    slopes}.

\end{frame}


\begin{frame}{Simple Slopes}

  Recall the derivation of our moderated equation:
  \begin{align*}
    Y = \alpha + \beta_1X + \beta_2Z + \beta_3XZ + e_i
  \end{align*}
  We can reverse the process by factoring out $X$ and reordering terms
  to get back to:
  \begin{align*}
    Y = \alpha + (\beta_1 + \beta_3Z)X + \beta_2Z + e_i
  \end{align*}
  Where $f(Z) = \beta_1 + \beta_3Z$ is the linear function that shows
  how the relationship between $X$ and $Y$ changes as a function of
  $Z$.\\
  \va
  \underline{$f(Z)$ is actually our \emph{simple slope}.}
  \vb
  \begin{itemize}
  \item By plugging different values of $Z$ into $f(Z)$, we get the
    slope of the conditional effect of $X$ on $Y$ at the chosen
    value of $Z$.
  \end{itemize}

\end{frame}


\begin{frame}{Significance Testing of Simple Slopes}

  The conditional values of $Z$ used to define the simple slopes in
  the pick-a-point approach are totally arbitrary
  \vb
  \begin{itemize}
  \item The most popular choice is: $\left\{ (\bar{Z} - SD_Z), \bar{Z},
    (\bar{Z} + SD_Z) \right\}$
    \vc
  \item You could also use interesting percentiles of $Z$'s
    distribution
  \end{itemize}
  \va
  The standard error of a simple slope is given by:
  \begin{align}
    SE_{SS} = \sqrt{SE_{\beta_1}^2 + 2Z \cdot \text{COV}(\beta_1, \beta_3) + Z^2 SE_{\beta_3}^2}
  \end{align}
  So, you can test the significance of a simple slope by constructing
  a Wald statistic or confidence interval using $SE_{SS}$:
  \begin{align*}
    Wald_{SS} &= \frac{\hat{f}(Z)}{SE_{SS}}\\
    95\% CI_{SS} &= \hat{f}(Z) \pm 1.96 \cdot SE_{SS}
  \end{align*}

\end{frame}


\begin{frame}[allowframebreaks]{Example}

\begin{Schunk}
\begin{Sinput}
 sum(out3.1$fitted - out3.3$fitted)
\end{Sinput}
\begin{Soutput}
[1] -2.785328e-12
\end{Soutput}
\begin{Sinput}
 summary(out3.1)$r.squared
\end{Sinput}
\begin{Soutput}
[1] 0.9999439
\end{Soutput}
\begin{Sinput}
 summary(out3.3)$r.squared
\end{Sinput}
\begin{Soutput}
[1] 0.9999439
\end{Soutput}
\end{Schunk}

\pagebreak
\begin{Schunk}
\begin{Sinput}
 nSams <- 1000
 abVec <- rep(NA, nSams)
 for(i in 1 : nSams) {
     ## Resample the data:
     bootSam <- 
         dat1[sample(c(1 : nrow(dat1)), replace = TRUE), ]
     ## Fit the path model:
     bootOut <- sem(mod2, data = bootSam)
     ## Store the estimated indirect effect:
     abVec[i] <- prod(coef(bootOut)[c("a", "b")])
 }
\end{Sinput}
\end{Schunk}

\pagebreak
\begin{Schunk}
\begin{Sinput}
 ## Conditional process model with a, b, c paths moderated:
 mod4 <- "
 agree ~ b1*open + b2*consc + cp1*extra + cp2*neuro + 
         cp3*extraXneuro + b3*openXconsc
 open ~ a1*extra + a2*neuro + a3*extraXneuro
 
 cpLo  := cp1 + cp3*(-0.962268)
 cpMid := cp1 + cp3*(-0.162268)
 cpHi  := cp1 + cp3*0.837732
 
 abLoLo  := (a1 + a3*(-0.962268)) * (b1 + b3*(-0.4045))
 abLoMid := (a1 + a3*(-0.962268)) * (b1 + b3*(-0.0045))
 abLoHi  := (a1 + a3*(-0.962268)) * (b1 + b3*0.3955)
 
 abMidLo  := (a1 + a3*(-0.162268)) * (b1 + b3*(-0.4045))
 abMidMid := (a1 + a3*(-0.162268)) * (b1 + b3*(-0.0045))
 abMidHi  := (a1 + a3*(-0.162268)) * (b1 + b3*0.3955)
 
 abHiLo  := (a1 + a3*0.837732) * (b1 + b3*(-0.4045))
 abHiMid := (a1 + a3*0.837732) * (b1 + b3*(-0.0045))
 abHiHi  := (a1 + a3*0.837732) * (b1 + b3*0.3955)
 "
\end{Sinput}
\end{Schunk}


\end{frame}


\end{document}
