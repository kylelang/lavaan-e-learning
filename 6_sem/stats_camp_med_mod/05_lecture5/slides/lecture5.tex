\documentclass{beamer}
\usetheme{ttuStatsCamp}
\usefonttheme{serif}
\usepackage[T1]{fontenc}
\usepackage[utf8]{inputenc}
\usepackage{url}
\usepackage{graphicx}
\usepackage{setspace}
\usepackage[natbibapa]{apacite}
\usepackage{color}
\usepackage{amsmath}
\usepackage{amsfonts}
\usepackage{Sweavel}
\usepackage{listings}

\def\Sweavesize{\scriptsize}
\def\Rcolor{\color{black}}
%\def\Routcolor{\color{red}}
\def\Rcommentcolor{\color{violet}}
\def\Rbackground{\color[gray]{0.85}}
\def\Routbackground{\color[gray]{0.85}}

\lstset{tabsize=2, breaklines=true, style=Rstyle}



\newcommand{\red}[0]{\textcolor{red}}
\newcommand{\green}[0]{\textcolor{green}}
\newcommand{\blue}[0]{\textcolor{blue}}
\newcommand{\comment}[1]{}
\newcommand{\va}[0]{\vspace{12pt}}
\newcommand{\vb}[0]{\vspace{6pt}}
\newcommand{\vc}[0]{\vspace{3pt}}
\newcommand{\vx}[1]{\vspace{#1pt}}

\title[Lecture 5]{Lecture 5: A Bit Advanced Mediation}

\author{Kyle M. Lang}

\institute[TTU IMMAP]{
  Institute for Measurement, Methodology, Analysis \& Policy\\
  Texas Tech University\\
  Lubbock, TX
}

\date{2016 Stats Camp}

\setbeamertemplate{frametitle continuation}{}

\begin{document}

\setkeys{Gin}{width=\textwidth}

\input{sweaveFiles/-001}


\begin{frame}[plain]
  
  \titlepage
  
\end{frame}


\begin{frame}{Outline}
  
  \begin{itemize}
  \item Show how to test for indirect effects in latent variable models
    \va
  \item Discuss the interpretation of indirect effects
    \va
  \item Discuss effect size measures for indirect effects
  \end{itemize}
  
\end{frame}


\begin{frame}{Boring Model}

  So far, all of our models have been similar to:
  \begin{figure}
    \includegraphics[width=.8\textwidth]{figures/simpleMediationPathDiagram.pdf}
  \end{figure}
  But there is no reason that we need to restrict ourselves to mucking
  about with observed variables.

\end{frame}



\begin{frame}{Better Model}

  We can (and should) test for indirect effects using \emph{latent
    variable models} such as:
  \begin{figure}
    \includegraphics[width=.8\textwidth]{figures/semMedDiagram.pdf}
  \end{figure}
  Measurement error can be a big problem for mediation analysis, so
  latent variable modeling is highly recommended.

\end{frame}



\begin{frame}[allowframebreaks]{Example}
  
\begin{Schunk}
\begin{Sinput}
 dat1 <- readRDS("../data/adamsKlpsScaleScore.rds")
 ## Partial out the mediator's effect:
 mod1 <- lm(policy ~ sysRac + polAffil, data = dat1)
 mod2 <- lm(sysRac ~ polAffil, data = dat1)
 summary(mod1)$coef
\end{Sinput}
\begin{Soutput}
              Estimate Std. Error   t value     Pr(>|t|)
(Intercept) 0.83265885 0.41246491 2.0187386 4.670428e-02
sysRac      0.72235878 0.11147514 6.4799987 5.930291e-09
polAffil    0.05121251 0.06998433 0.7317711 4.663450e-01
\end{Soutput}
\begin{Sinput}
 summary(mod2)$coef
\end{Sinput}
\begin{Soutput}
             Estimate Std. Error  t value     Pr(>|t|)
(Intercept) 2.6060451 0.28489391 9.147423 2.715546e-14
polAffil    0.2568494 0.06213495 4.133735 8.336022e-05
\end{Soutput}
\begin{Sinput}
 ## Extract important parameter estimates:
 a <- coef(mod2)["polAffil"]
 b <- coef(mod1)["sysRac"]
 ## Compute indirect effect:
 ieProd <- a * b
 ieProd
\end{Sinput}
\begin{Soutput}
 polAffil 
0.1855374 
\end{Soutput}
\begin{Sinput}
 ## Calculate Sobel's Z:
 seA <- sqrt(diag(vcov(mod2)))["polAffil"]
 seB <- sqrt(diag(vcov(mod1)))["sysRac"]
 sobelSE <- sqrt(b^2 * seA^2 + a^2 * seB^2)
 sobelZ <- ieProd / sobelSE
 sobelZ
\end{Sinput}
\begin{Soutput}
polAffil 
 3.48501 
\end{Soutput}
\begin{Sinput}
 sobelP <- 2 * pnorm(sobelZ, lower = FALSE)
 sobelP
\end{Sinput}
\begin{Soutput}
    polAffil 
0.0004921178 
\end{Soutput}
\begin{Sinput}
 sobelUB <- ieProd + 1.96 * sobelSE
 sobelLB <- ieProd - 1.96 * sobelSE
 ## 95% Sobel CI:
 c(sobelLB, sobelUB)
\end{Sinput}
\begin{Soutput}
  polAffil   polAffil 
0.08118957 0.28988525 
\end{Soutput}
\end{Schunk}


\pagebreak

\begin{Schunk}
\begin{Sinput}
 parameterEstimates(out1, 
                    boot = "bca.simple")[-c(6 : 13), -c(1 : 3)]
\end{Sinput}
\begin{Soutput}
   label   est    se     z pvalue ci.lower ci.upper
1     cp 0.082 0.259 0.318  0.751   -0.408    0.641
2      b 1.390 0.215 6.465  0.000    0.939    1.786
3     a1 0.729 0.091 8.014  0.000    0.539    0.908
4     a2 0.641 0.089 7.183  0.000    0.469    0.818
5     a3 0.451 0.097 4.643  0.000    0.223    0.629
14   imm 0.628 0.169 3.706  0.000    0.296    0.980
\end{Soutput}
\end{Schunk}


\end{frame}



\begin{frame}{Interpretation of Indirect Effects}
  
  Although indirect effects are composed parameters, they have direct
  interpretations, independent of the interpretations of their
  constituent paths:
  \vb
  \begin{itemize}
  \item The $X \rightarrow M \rightarrow Y$ indirect effect $ab$ is
    interpreted as:
    \vb
    \begin{itemize}
    \item The expected change in $Y$ for a unit change in $X$ that
      is transmitted indirectly through $M$, or...
      \vb
    \item For a unit change in $X$, $Y$ is expected to change by
      $ab$ units, indirectly through $M$, or...
      \vb
    \item Participants who differ by one unit on $X$ are expect
      to differ by $ab$ units on $Y$ as a results of the effect
      of $X$ on $M$ which, in turn, affects $Y$.
    \end{itemize}
    \va
  \item The interpretation/scaling of the indirect effect is entirely
    defined by the input $X$ and outcome $Y$
    \vb
    \begin{itemize}
    \item The scaling of the intermediary variable $M$ does not affect
      the interpretation of the indirect effect.
    \end{itemize}
  \end{itemize}
  
\end{frame}
  
  
\begin{frame}{Partially Standardized Indirect Effect}
  
  \begin{align*}
    ab_{ps} &= \frac{ab}{SD_Y}\\
    c'_{ps} &= \frac{c'}{SD_Y}\\
    c_{ps} &= \frac{c}{SD_Y} = ab_{ps} + c'_{ps}
  \end{align*}
  
  \begin{itemize}
    \item Simple
    \item Removes binding to the scale of $Y$
    \item Still scale-bound by $X$
    \item Not clear what constitutes a ``large'' effect
  \end{itemize}
  
\end{frame}



\begin{frame}{Completely Standardized Indirect Effect}
  
  \begin{align*}
    ab_{cs} &= \frac{SD_X ab}{SD_Y}\\
    c'_{cs} &= \frac{SD_X c'}{SD_Y}\\
    c_{cs} &= \frac{SD_X c}{SD_Y} = ab_{cs} + c'_{cs}
  \end{align*}
  
  \begin{itemize}
    \item Simple
    \item Removes all scale binding
    \item Not clear what constitutes a ``large'' effect
  \end{itemize}
  
\end{frame}


\begin{frame}{Ratio of the Indirect Effect to the Total Effect}
  
  \begin{align*}
    P_M = \frac{ab}{c} = \frac{ab}{c' + ab}
  \end{align*}
  
  \begin{itemize}
  \item Very simple
  \item Not bounded by 0 and 1
  \item Explodes toward $\pm \infty$ as $c\rightarrow 0$
  \item Very unstable
    \begin{itemize}
    \item High between-sample variability
    \item Requires $N \geq 500$
    \end{itemize}
  \end{itemize}
\end{frame}



\begin{frame}{Ratio of the Indirect Effect to the Direct Effect}
  
  \begin{align*}
    R_M = \frac{ab}{c'} = \frac{P_M}{1 - P_M}
  \end{align*}
  
  \begin{itemize}
  \item Very simple
  \item Not bounded by 0 and 1
  \item Explodes toward $\pm \infty$ as $c'\rightarrow 0$
  \item Very unstable
    \begin{itemize}
    \item High between-sample variability
    \item Requires $N \geq 2000$
    \end{itemize}
  \end{itemize}
  
\end{frame}


\begin{frame}{Proportion of Variance in $Y$ Explained by the Indirect Effect}
  
  Developed by \citet{fairchildEtAl:2009}.
  \begin{itemize}
    \item Given a non-zero total effect, represents the proportion of
      variance in Y accounted for by the indirect effect.
  \end{itemize}
  
  \begin{align*}
    R_{med}^2 = r_{MY}^2 - \left( R_{Y.MX}^2-r_{XY}^2 \right)
  \end{align*}

  \begin{itemize}
  \item Mostly sensible interpretation
  \item Predicated on the assumption that $\beta_{YX} \neq 0$
  \item $|ab| > |c| \Rightarrow R_{med}^2 < 0$
    \begin{itemize}
    \item Not a strict proportion
    \end{itemize}
  \end{itemize}
  
\end{frame}


\begin{frame}{Kappa Squared}
  
  Developed by \citet{preacherKelley:2011}.
  \begin{itemize}
  \item Gives the proportion of the \emph{maximum possible} indirect
    effect represented by $ab$.
  \end{itemize}
  
  \begin{align*}
    \kappa^2 = \frac{ab}{\text{max}(ab)}
  \end{align*}
  
  \begin{itemize}
  \item Bounded by 0 and 1
  \item Values closer to 1.0 indicate a bigger effect
  \item A bit of a pain to calculate.
  \end{itemize}
  
\end{frame}


\begin{frame}{Computing $\text{max}(ab)$}
  
  \begin{align*}
    a &\in \left\{ 
    \frac{
      \sigma_{YM} \sigma_{YX} \pm 
      \sqrt{ \sigma_M^2 \sigma_Y^2 - \sigma_{YM}^2 }
      \sqrt{ \sigma_X^2 \sigma_Y^2 - \sigma_{YX}^2 } 
    }{ 
      \sigma_X^2 \sigma_Y^2 
    }
    \right\} 
    = [a_{low}, a_{high}],
    \\
    \\
    b &\in \left\{
    \pm \frac{
      \sqrt{ \sigma_X^2 \sigma_Y^2 - \sigma_{YX}^2 }
    }{
      \sqrt{ \sigma_X^2 \sigma_M^2 - \sigma_{MX}^2 }
    }
    \right\} = [b_{low}, b_{high}],
  \end{align*}
  
  \begin{align*}
    \text{max}(a) = \left\{ \begin{array}{lll}
      a_{high}, & \text{ if } & \hat{a} > 0\\
      a_{low}, & \text{ if } & \hat{a} < 0
    \end{array}
    \right.,~~
    \text{max}(b) = \left\{ \begin{array}{lll}
      b_{high}, & \text{ if } & \hat{b} > 0\\
      b_{low}, & \text{ if } & \hat{b} < 0
    \end{array}
    \right.,
  \end{align*}
  
  \begin{align*}
    \text{max}(ab) = \text{max}(a) \text{max}(b)
  \end{align*}
  
\end{frame}


\begin{frame}[allowframebreaks]{Example}
    
\begin{Schunk}
\begin{Sinput}
 mod3 <- "
 att3 ~ att2 + b2*conf2 + cp2*horn2
 att2 ~ att1 + b1*conf1 + cp1*horn1
 
 conf3 ~ conf2 + a2*horn2
 conf2 ~ conf1 + a1*horn1
 
 horn3 ~ horn2
 horn2 ~ horn1
 
 horn3 ~~ conf3 + att3
 conf3 ~~ att3
 
 horn2 ~~ conf2 + att2
 conf2 ~~ att2
 
 a1 == a2
 b1 == b2
 cp1 == cp2
 "
 out3 <- sem(mod3, data = dat1)
 summary(out3)
\end{Sinput}
\begin{Soutput}
lavaan (0.5-20) converged normally after  46 iterations

  Number of observations                           500

  Estimator                                         ML
  Minimum Function Test Statistic              294.220
  Degrees of freedom                                18
  P-value (Chi-square)                           0.000

Parameter Estimates:

  Information                                 Expected
  Standard Errors                             Standard

Regressions:
                   Estimate  Std.Err  Z-value  P(>|z|)
  att3 ~                                              
    att2              0.497    0.035   14.234    0.000
    conf2     (b2)    0.098    0.019    5.200    0.000
    horn2    (cp2)    0.083    0.072    1.157    0.247
  att2 ~                                              
    att1              0.530    0.040   13.345    0.000
    conf1     (b1)    0.098    0.019    5.200    0.000
    horn1    (cp1)    0.083    0.072    1.157    0.247
  conf3 ~                                             
    conf2             0.684    0.035   19.602    0.000
    horn2     (a2)    0.493    0.107    4.596    0.000
  conf2 ~                                             
    conf1             0.623    0.032   19.546    0.000
    horn1     (a1)    0.493    0.107    4.596    0.000
  horn3 ~                                             
    horn2             0.826    0.030   27.609    0.000
  horn2 ~                                             
    horn1             0.714    0.024   29.181    0.000

Covariances:
                   Estimate  Std.Err  Z-value  P(>|z|)
  conf3 ~~                                            
    horn3             1.016    0.155    6.556    0.000
  att3 ~~                                             
    horn3             0.322    0.093    3.483    0.000
    conf3             3.574    0.465    7.691    0.000
  conf2 ~~                                            
    horn2             0.836    0.124    6.721    0.000
  att2 ~~                                             
    horn2             0.273    0.083    3.289    0.001
    conf2             2.027    0.400    5.067    0.000

Variances:
                   Estimate  Std.Err  Z-value  P(>|z|)
    att3              6.019    0.381   15.811    0.000
    att2              6.041    0.382   15.811    0.000
    conf3            15.814    1.000   15.811    0.000
    conf2            12.570    0.795   15.811    0.000
    horn3             0.695    0.044   15.811    0.000
    horn2             0.560    0.035   15.811    0.000

Constraints:
                                               |Slack|
    a1 - (a2)                                    0.000
    b1 - (b2)                                    0.000
    cp1 - (cp2)                                  0.000
\end{Soutput}
\begin{Sinput}
 chiDiff <- fitMeasures(out3)["chisq"] -
     fitMeasures(out1)["chisq"]
 dfDiff <- fitMeasures(out3)["df"] -
     fitMeasures(out1)["df"]
 pchisq(chiDiff, dfDiff, lower = FALSE)
\end{Sinput}
\begin{Soutput}
     chisq 
0.02684148 
\end{Soutput}
\end{Schunk}


\pagebreak

\begin{Schunk}
\begin{Sinput}
 parameterEstimates(out2, 
                    boot = "bca.simple")[-c(7 : 18), -c(1 : 3)]
\end{Sinput}
\begin{Soutput}
   label    est    se      z pvalue ci.lower ci.upper
1    cp1 -0.103 0.190 -0.543  0.587   -0.473    0.259
2    cp2  1.099 0.204  5.380  0.000    0.639    1.472
3     b1  1.615 0.167  9.649  0.000    1.270    1.912
4     b2  0.381 0.173  2.206  0.027    0.034    0.720
5     b3  0.571 0.173  3.297  0.001    0.220    0.921
6      a  0.741 0.131  5.638  0.000    0.471    0.984
19   imm  0.424 0.153  2.773  0.006    0.174    0.781
\end{Soutput}
\end{Schunk}


\pagebreak

\begin{Schunk}
\begin{Sinput}
 ## Completely Standardized:
 abCS <- (sdX * ab) / sdY
 abCS
\end{Sinput}
\begin{Soutput}
[1] 0.1345859
\end{Soutput}
\begin{Sinput}
 cPrimeCS <- (sdX * cPrime) / sdY
 cPrimeCS
\end{Sinput}
\begin{Soutput}
       cp 
0.1790413 
\end{Soutput}
\begin{Sinput}
 cCS <- abCS + cPrimeCS
 cCS
\end{Sinput}
\begin{Soutput}
       cp 
0.3136272 
\end{Soutput}
\end{Schunk}


\pagebreak

\begin{Schunk}
\begin{Sinput}
 sum(out3.1$fitted - out3.3$fitted)
\end{Sinput}
\begin{Soutput}
[1] -2.785328e-12
\end{Soutput}
\begin{Sinput}
 summary(out3.1)$r.squared
\end{Sinput}
\begin{Soutput}
[1] 0.9999439
\end{Soutput}
\begin{Sinput}
 summary(out3.3)$r.squared
\end{Sinput}
\begin{Soutput}
[1] 0.9999439
\end{Soutput}
\end{Schunk}


\end{frame}


\begin{frame}[allowframebreaks]{Compute $\kappa^2$}

\begin{Schunk}
\begin{Sinput}
 nSams <- 1000
 abVec <- rep(NA, nSams)
 for(i in 1 : nSams) {
     ## Resample the data:
     bootSam <- 
         dat1[sample(c(1 : nrow(dat1)), replace = TRUE), ]
     ## Fit the path model:
     bootOut <- sem(mod2, data = bootSam)
     ## Store the estimated indirect effect:
     abVec[i] <- prod(coef(bootOut)[c("a", "b")])
 }
\end{Sinput}
\end{Schunk}


\pagebreak

\begin{Schunk}
\begin{Sinput}
 ## Conditional process model with a, b, c paths moderated:
 mod4 <- "
 agree ~ b1*open + b2*consc + cp1*extra + cp2*neuro + 
         cp3*extraXneuro + b3*openXconsc
 open ~ a1*extra + a2*neuro + a3*extraXneuro
 
 cpLo  := cp1 + cp3*(-0.962268)
 cpMid := cp1 + cp3*(-0.162268)
 cpHi  := cp1 + cp3*0.837732
 
 abLoLo  := (a1 + a3*(-0.962268)) * (b1 + b3*(-0.4045))
 abLoMid := (a1 + a3*(-0.962268)) * (b1 + b3*(-0.0045))
 abLoHi  := (a1 + a3*(-0.962268)) * (b1 + b3*0.3955)
 
 abMidLo  := (a1 + a3*(-0.162268)) * (b1 + b3*(-0.4045))
 abMidMid := (a1 + a3*(-0.162268)) * (b1 + b3*(-0.0045))
 abMidHi  := (a1 + a3*(-0.162268)) * (b1 + b3*0.3955)
 
 abHiLo  := (a1 + a3*0.837732) * (b1 + b3*(-0.4045))
 abHiMid := (a1 + a3*0.837732) * (b1 + b3*(-0.0045))
 abHiHi  := (a1 + a3*0.837732) * (b1 + b3*0.3955)
 "
\end{Sinput}
\end{Schunk}


\end{frame}


\begin{frame}{Practice}
  
\begin{Schunk}
\begin{Sinput}
 parameterEstimates(out2.2, boot = bootType)[ , -c(1 : 3)]
\end{Sinput}
\begin{Soutput}
   label    est    se      z pvalue ci.lower ci.upper
1     cp  0.135 0.083  1.638  0.102   -0.031    0.286
2     b2  0.597 0.137  4.359  0.000    0.311    0.858
3     a1 -0.266 0.061 -4.353  0.000   -0.392   -0.148
4    d21 -0.367 0.098 -3.764  0.000   -0.546   -0.166
5         0.987 0.166  5.946  0.000    0.731    1.402
6         0.719 0.116  6.218  0.000    0.527    0.991
7         0.705 0.094  7.520  0.000    0.552    0.926
8         2.444 0.000     NA     NA    2.444    2.444
9 fullIE  0.058 0.025  2.318  0.020    0.019    0.123
\end{Soutput}
\end{Schunk}


Suppose:
\begin{enumerate}
\item $\Sigma$ is given by:
\begin{Schunk}
\begin{Sinput}
 ## Test Differences between Indirect Effects
 ## in Serial Multiple Mediator Model (Method 1):
 mod2.3 <- "
 policy ~ cp*polAffil + b1*merit + b2*sysRac
 merit ~ a1*polAffil
 sysRac ~ a2*polAffil + d21*merit
 
 ab1 := a1*b1
 ab2 := a2*b2
 fullIE := a1*d21*b2
 totalIE := ab1 + ab2 + fullIE 
 
 fullIE == ab1
 fullIE == ab2
 "
 out2.3 <- 
     sem(mod2.3, data = dat1, se = "boot", boot = nBoot)
 summary(out2.3)
\end{Sinput}
\begin{Soutput}
lavaan (0.5-20) converged normally after 213 iterations

  Number of observations                            87

  Estimator                                         ML
  Minimum Function Test Statistic                1.334
  Degrees of freedom                                 2
  P-value (Chi-square)                           0.513

Parameter Estimates:

  Information                                 Observed
  Standard Errors                            Bootstrap
  Number of requested bootstrap draws             2500
  Number of successful bootstrap draws            2500

Regressions:
                   Estimate  Std.Err  Z-value  P(>|z|)
  policy ~                                            
    polAffil  (cp)    0.108    0.084    1.281    0.200
    merit     (b1)   -0.150    0.047   -3.183    0.001
    sysRac    (b2)    0.521    0.125    4.157    0.000
  merit ~                                             
    polAffil  (a1)   -0.271    0.057   -4.750    0.000
  sysRac ~                                            
    polAffil  (a2)    0.078    0.025    3.125    0.002
    merit    (d21)   -0.287    0.075   -3.814    0.000

Variances:
                   Estimate  Std.Err  Z-value  P(>|z|)
    policy            1.001    0.171    5.854    0.000
    merit             0.719    0.114    6.330    0.000
    sysRac            0.690    0.090    7.632    0.000

Defined Parameters:
                   Estimate  Std.Err  Z-value  P(>|z|)
    ab1               0.041    0.014    2.983    0.003
    ab2               0.041    0.014    2.983    0.003
    fullIE            0.041    0.014    2.983    0.003
    totalIE           0.122    0.041    2.983    0.003

Constraints:
                                               |Slack|
    fullIE - (ab1)                               0.000
    fullIE - (ab2)                               0.000
\end{Soutput}
\begin{Sinput}
 ## Conduct a chi-squared difference test:
 chiDiff <- fitMeasures(out2.3)["chisq"] - 
     fitMeasures(out2.1)["chisq"]
 dfDiff <- fitMeasures(out2.3)["df"] - 
     fitMeasures(out2.1)["df"]
 pchisq(chiDiff, dfDiff, lower = FALSE)
\end{Sinput}
\begin{Soutput}
    chisq 
0.5131246 
\end{Soutput}
\end{Schunk}

\va
\item The estimated paths are:
  \begin{itemize}
  \item $a = $ 0.2
    \vb
  \item $b = $ 0.246
    \vb
  \item $ab = $ 0.049
  \end{itemize}
  \end{enumerate}
  \va
  Compute $\kappa^2$ for the estimated $ab$.
  
\end{frame}

\begin{frame}{References}
\bibliographystyle{apacite}
\bibliography{../../bibtexStuff/dissRefsList}
\end{frame}


\end{document}
