\documentclass{beamer}
\usetheme{ttuStatsCamp}
\usefonttheme{serif}
\usepackage[T1]{fontenc}
\usepackage[utf8]{inputenc}
\usepackage{url}
\usepackage{graphicx}
\usepackage{setspace}
\usepackage[natbibapa]{apacite}
\usepackage{color}
\usepackage{amsmath}
\usepackage{amsfonts}
\usepackage{Sweavel}
\usepackage{listings}
\usepackage{fancybox}

\def\Sweavesize{\scriptsize}
\def\Rcolor{\color{black}}
%\def\Routcolor{\color{red}}
\def\Rcommentcolor{\color{violet}}
\def\Rbackground{\color[gray]{0.85}}
\def\Routbackground{\color[gray]{0.85}}

\lstset{tabsize=2, breaklines=true, style=Rstyle}



\newcommand{\red}[0]{\textcolor{red}}
\newcommand{\green}[0]{\textcolor{green}}
\newcommand{\blue}[0]{\textcolor{blue}}
\newcommand{\comment}[1]{}
\newcommand{\va}[0]{\vspace{12pt}}
\newcommand{\vb}[0]{\vspace{6pt}}
\newcommand{\vc}[0]{\vspace{3pt}}
\newcommand{\vx}[1]{\vspace{#1pt}}

\title[Lecture 12]{Lecture 12: Loose Ends}

\author{Kyle M. Lang}

\institute[TTU IMMAP]{
  Institute for Measurement, Methodology, Analysis \& Policy\\
  Texas Tech University\\
  Lubbock, TX
}

\date{2016 Stats Camp}

\setbeamertemplate{frametitle continuation}{}

\begin{document}

\setkeys{Gin}{width=\textwidth}

\input{sweaveFiles/-001}


\begin{frame}[plain]
  
  \titlepage
  
\end{frame}



\begin{frame}{Outline}

  \begin{itemize}
  \item Latent variable interactions
    \va
  \item Moderated logistic regression
    \va
  \item Effect size for conditional process analysis
  \end{itemize}
  
\end{frame}


\begin{frame}[shrink = 5]{Latent Variable Interactions}
  
  When we have two observed variables interacting to predict a latent
  variable, our job is easy:
  \begin{enumerate}
    \item Construct the product term of the observed focal and
      moderator variables
    \item Use the observed focal, moderator, and interaction variables
      to predict the latent DV
  \end{enumerate}\\
  \va If we want to model moderation when at least on of the
  predictors is latent, things get more difficult.
  \begin{itemize}
  \item If the moderator is observed and discrete, we can use multiple
    group modeling
  \item If the moderator is continuous and/or latent, then we need
    fancier methods
  \end{itemize}\\
  \va
  Two basic approaches:
  \begin{enumerate}
  \item Methods based on products of manifest variables
  \item Methods based on directly estimating the products of latent
    variables
  \end{enumerate}

\end{frame}


\begin{frame}{Estimating Products of Latent Variables}
  
  We can directly estimate the interaction between two latent
  variables with the \emph{latent moderated structural equations}
  (LMS) method.
  \va
  \begin{itemize}
  \item Introduced by \citet{kleinEtAl:1997} and formalized by
    \citet{kleinMoosbrugger:2000} 
    \vb
  \item Currently only available in Mplus (via the \texttt{Xwith}
    command).
    \vb
  \item Uses numerical integration to estimate the unobserved latent
    interaction term
  \end{itemize}
  
\end{frame}


\begin{frame}{Estimating Products of Latent Variables}
  
  \textsc{LMS Strengths:}
  \begin{itemize}
    \item Tends to perform the best out of all available methods
      \vb
    \item No need to pre-process the data by manually computing
      product terms
      \vb
    \item Pretty easy to implement if you have Mplus (see users guide
      for examples).
  \end{itemize}
  \va
  \textsc{LMS Weaknesses:}
  \begin{itemize}
  \item Only available in one (proprietary) software package
    \vb
  \item Numerical integration is very slow and precludes calculation
    of most fit indices
    \vb
  \item LMS does not work with categorical observed moderators
  \end{itemize}
  
\end{frame}


\begin{frame}{Computing Interaction Indicators}
  
  The alternative to the LMS-type approach is to create observed
  product terms and directly use those terms as indicators of the
  interaction construct.
  \vb
  \begin{itemize}
  \item Naively indicating an interaction construct with the raw
    product terms is probably sub-optimal
    \vb
  \item Collinearity among the interaction indicators and the raw
    items can cause estimation problems
    \vb
  \item From a modeling perspective, we'd like to interpret out final
    model holistically
  \end{itemize}
  \va
  Two recommended approaches:
  \vb
  \begin{enumerate}
  \item Orthogonalization through residual centering
    \citep{littleEtAl:2006}.
    \vb
  \item Double mean centering \citep{linEtAl:2010}.
  \end{enumerate}
  
\end{frame}


\begin{frame}[allowframebreaks]{Orthogonalization}
  
  Say we want to estimate the moderated effect of $Z$ on the $X
  \rightarrow Y$ effect, where $X$, $Y$, and $Z$ are latent variables
  indicated by $\{x_1, x_2, x_3\}$, $\{y_1, y_2, y_3\}$, and $\{z_1,
  z_2, z_3\}$, respectively.\\
  \va
  Orthogonalization is performed by:
  \vb
  \begin{enumerate}
  \item Construct all possible product terms: $\{x_1z_1, x_1z_2, x_1z_3, x_2z_1, x_2z_2, x_2z_3, x_3z_1, x_3z_2, x_3z_3\}$.
    \vb
  \item Regress each product term onto all observed indicators of $X$ and $Z$:
    \begin{align*}
      \widehat{x_1z_1} &= \alpha + \beta_1x_1 + \beta_2x_2 + \beta_2x_3 + 
      \beta_4z_1 + \beta_5z_2 + \beta_6z_3\\
      \widehat{x_2z_1} &= \alpha + \beta_1x_1 + \beta_2x_2 + \beta_2x_3 + 
      \beta_4z_1 + \beta_5z_2 + \beta_6z_3\\
      &~~~\vdots\\
      \widehat{x_3z_3} &= \alpha + \beta_1x_1 + \beta_2x_2 + \beta_2x_3 + 
      \beta_4z_1 + \beta_5z_2 + \beta_6z_3
    \end{align*}
    
    \pagebreak
    
  \item Calculate each product term's residual:
    \begin{align*}
      \delta_{x1z1} &= x_1z_1 - \widehat{x_1z_1}\\
      \delta_{x1z1} &= x_2z_1 - \widehat{x_2z_1}\\
      &~~~\vdots\\
      \delta_{x3z3} &= x_3z_3 - \widehat{x_3z_3}
    \end{align*}
    \vb
  \item Use these residuals to indicate a latent interaction construct
    as represented in the following figure.
  \end{enumerate}
  
\end{frame}


\begin{frame}{Orthogonalization}
  
  \begin{figure}
    \includegraphics[width=\textwidth]{figures/orthoDiagram.pdf} 
  \end{figure}
  
\end{frame}



\begin{frame}[allowframebreaks]{Example}
  
\begin{Schunk}
\begin{Sinput}
 dat1 <- readRDS("../data/adamsKlpsScaleScore.rds")
 ## Partial out the mediator's effect:
 mod1 <- lm(policy ~ sysRac + polAffil, data = dat1)
 mod2 <- lm(sysRac ~ polAffil, data = dat1)
 summary(mod1)$coef
\end{Sinput}
\begin{Soutput}
              Estimate Std. Error   t value     Pr(>|t|)
(Intercept) 0.83265885 0.41246491 2.0187386 4.670428e-02
sysRac      0.72235878 0.11147514 6.4799987 5.930291e-09
polAffil    0.05121251 0.06998433 0.7317711 4.663450e-01
\end{Soutput}
\begin{Sinput}
 summary(mod2)$coef
\end{Sinput}
\begin{Soutput}
             Estimate Std. Error  t value     Pr(>|t|)
(Intercept) 2.6060451 0.28489391 9.147423 2.715546e-14
polAffil    0.2568494 0.06213495 4.133735 8.336022e-05
\end{Soutput}
\begin{Sinput}
 ## Extract important parameter estimates:
 a <- coef(mod2)["polAffil"]
 b <- coef(mod1)["sysRac"]
 ## Compute indirect effect:
 ieProd <- a * b
 ieProd
\end{Sinput}
\begin{Soutput}
 polAffil 
0.1855374 
\end{Soutput}
\begin{Sinput}
 ## Calculate Sobel's Z:
 seA <- sqrt(diag(vcov(mod2)))["polAffil"]
 seB <- sqrt(diag(vcov(mod1)))["sysRac"]
 sobelSE <- sqrt(b^2 * seA^2 + a^2 * seB^2)
 sobelZ <- ieProd / sobelSE
 sobelZ
\end{Sinput}
\begin{Soutput}
polAffil 
 3.48501 
\end{Soutput}
\begin{Sinput}
 sobelP <- 2 * pnorm(sobelZ, lower = FALSE)
 sobelP
\end{Sinput}
\begin{Soutput}
    polAffil 
0.0004921178 
\end{Soutput}
\begin{Sinput}
 sobelUB <- ieProd + 1.96 * sobelSE
 sobelLB <- ieProd - 1.96 * sobelSE
 ## 95% Sobel CI:
 c(sobelLB, sobelUB)
\end{Sinput}
\begin{Soutput}
  polAffil   polAffil 
0.08118957 0.28988525 
\end{Soutput}
\end{Schunk}


\pagebreak

\begin{Schunk}
\begin{Sinput}
 parameterEstimates(out1, 
                    boot = "bca.simple")[-c(6 : 13), -c(1 : 3)]
\end{Sinput}
\begin{Soutput}
   label   est    se     z pvalue ci.lower ci.upper
1     cp 0.082 0.259 0.318  0.751   -0.408    0.641
2      b 1.390 0.215 6.465  0.000    0.939    1.786
3     a1 0.729 0.091 8.014  0.000    0.539    0.908
4     a2 0.641 0.089 7.183  0.000    0.469    0.818
5     a3 0.451 0.097 4.643  0.000    0.223    0.629
14   imm 0.628 0.169 3.706  0.000    0.296    0.980
\end{Soutput}
\end{Schunk}


\end{frame}


\begin{frame}[allowframebreaks]{Example}
  
\begin{Schunk}
\begin{Sinput}
 mod3 <- "
 att3 ~ att2 + b2*conf2 + cp2*horn2
 att2 ~ att1 + b1*conf1 + cp1*horn1
 
 conf3 ~ conf2 + a2*horn2
 conf2 ~ conf1 + a1*horn1
 
 horn3 ~ horn2
 horn2 ~ horn1
 
 horn3 ~~ conf3 + att3
 conf3 ~~ att3
 
 horn2 ~~ conf2 + att2
 conf2 ~~ att2
 
 a1 == a2
 b1 == b2
 cp1 == cp2
 "
 out3 <- sem(mod3, data = dat1)
 summary(out3)
\end{Sinput}
\begin{Soutput}
lavaan (0.5-20) converged normally after  46 iterations

  Number of observations                           500

  Estimator                                         ML
  Minimum Function Test Statistic              294.220
  Degrees of freedom                                18
  P-value (Chi-square)                           0.000

Parameter Estimates:

  Information                                 Expected
  Standard Errors                             Standard

Regressions:
                   Estimate  Std.Err  Z-value  P(>|z|)
  att3 ~                                              
    att2              0.497    0.035   14.234    0.000
    conf2     (b2)    0.098    0.019    5.200    0.000
    horn2    (cp2)    0.083    0.072    1.157    0.247
  att2 ~                                              
    att1              0.530    0.040   13.345    0.000
    conf1     (b1)    0.098    0.019    5.200    0.000
    horn1    (cp1)    0.083    0.072    1.157    0.247
  conf3 ~                                             
    conf2             0.684    0.035   19.602    0.000
    horn2     (a2)    0.493    0.107    4.596    0.000
  conf2 ~                                             
    conf1             0.623    0.032   19.546    0.000
    horn1     (a1)    0.493    0.107    4.596    0.000
  horn3 ~                                             
    horn2             0.826    0.030   27.609    0.000
  horn2 ~                                             
    horn1             0.714    0.024   29.181    0.000

Covariances:
                   Estimate  Std.Err  Z-value  P(>|z|)
  conf3 ~~                                            
    horn3             1.016    0.155    6.556    0.000
  att3 ~~                                             
    horn3             0.322    0.093    3.483    0.000
    conf3             3.574    0.465    7.691    0.000
  conf2 ~~                                            
    horn2             0.836    0.124    6.721    0.000
  att2 ~~                                             
    horn2             0.273    0.083    3.289    0.001
    conf2             2.027    0.400    5.067    0.000

Variances:
                   Estimate  Std.Err  Z-value  P(>|z|)
    att3              6.019    0.381   15.811    0.000
    att2              6.041    0.382   15.811    0.000
    conf3            15.814    1.000   15.811    0.000
    conf2            12.570    0.795   15.811    0.000
    horn3             0.695    0.044   15.811    0.000
    horn2             0.560    0.035   15.811    0.000

Constraints:
                                               |Slack|
    a1 - (a2)                                    0.000
    b1 - (b2)                                    0.000
    cp1 - (cp2)                                  0.000
\end{Soutput}
\begin{Sinput}
 chiDiff <- fitMeasures(out3)["chisq"] -
     fitMeasures(out1)["chisq"]
 dfDiff <- fitMeasures(out3)["df"] -
     fitMeasures(out1)["df"]
 pchisq(chiDiff, dfDiff, lower = FALSE)
\end{Sinput}
\begin{Soutput}
     chisq 
0.02684148 
\end{Soutput}
\end{Schunk}


\pagebreak

\begin{Schunk}
\begin{Sinput}
 parameterEstimates(out2, 
                    boot = "bca.simple")[-c(7 : 18), -c(1 : 3)]
\end{Sinput}
\begin{Soutput}
   label    est    se      z pvalue ci.lower ci.upper
1    cp1 -0.103 0.190 -0.543  0.587   -0.473    0.259
2    cp2  1.099 0.204  5.380  0.000    0.639    1.472
3     b1  1.615 0.167  9.649  0.000    1.270    1.912
4     b2  0.381 0.173  2.206  0.027    0.034    0.720
5     b3  0.571 0.173  3.297  0.001    0.220    0.921
6      a  0.741 0.131  5.638  0.000    0.471    0.984
19   imm  0.424 0.153  2.773  0.006    0.174    0.781
\end{Soutput}
\end{Schunk}


\end{frame}


\begin{frame}[allowframebreaks]{Example}
  
\begin{Schunk}
\begin{Sinput}
 ## Completely Standardized:
 abCS <- (sdX * ab) / sdY
 abCS
\end{Sinput}
\begin{Soutput}
[1] 0.1345859
\end{Soutput}
\begin{Sinput}
 cPrimeCS <- (sdX * cPrime) / sdY
 cPrimeCS
\end{Sinput}
\begin{Soutput}
       cp 
0.1790413 
\end{Soutput}
\begin{Sinput}
 cCS <- abCS + cPrimeCS
 cCS
\end{Sinput}
\begin{Soutput}
       cp 
0.3136272 
\end{Soutput}
\end{Schunk}


\pagebreak

\begin{Schunk}
\begin{Sinput}
 sum(out3.1$fitted - out3.3$fitted)
\end{Sinput}
\begin{Soutput}
[1] -2.785328e-12
\end{Soutput}
\begin{Sinput}
 summary(out3.1)$r.squared
\end{Sinput}
\begin{Soutput}
[1] 0.9999439
\end{Soutput}
\begin{Sinput}
 summary(out3.3)$r.squared
\end{Sinput}
\begin{Soutput}
[1] 0.9999439
\end{Soutput}
\end{Schunk}


\pagebreak

\begin{Schunk}
\begin{Sinput}
 nSams <- 1000
 abVec <- rep(NA, nSams)
 for(i in 1 : nSams) {
     ## Resample the data:
     bootSam <- 
         dat1[sample(c(1 : nrow(dat1)), replace = TRUE), ]
     ## Fit the path model:
     bootOut <- sem(mod2, data = bootSam)
     ## Store the estimated indirect effect:
     abVec[i] <- prod(coef(bootOut)[c("a", "b")])
 }
\end{Sinput}
\end{Schunk}


\end{frame}



\begin{frame}{Matched Pair Variation}
  
  If you are willing to assume exchangeable indicators (i.e.,
  \emph{essential tau equivalence}), then you don't need to compute
  all possible interaction terms.\\ 
  \va 
  The so-called \emph{matched pair} strategy suggests constructing only 
  three product variables (when each first order construct has three indicators).
  \va
  \begin{itemize}
    \item Each product variable is simply constructed from paired
      indicators of the two first-order constructs:
      \begin{align*}
        x_1z_1 &= x_1 \times z_1\\
        x_2z_2 &= x_2 \times z_2\\
        x_3z_3 &= x_3 \times z_3
      \end{align*}
  \end{itemize}   

\end{frame}
  


\begin{frame}[allowframebreaks]{Example}
  
\begin{Schunk}
\begin{Sinput}
 ## Conditional process model with a, b, c paths moderated:
 mod4 <- "
 agree ~ b1*open + b2*consc + cp1*extra + cp2*neuro + 
         cp3*extraXneuro + b3*openXconsc
 open ~ a1*extra + a2*neuro + a3*extraXneuro
 
 cpLo  := cp1 + cp3*(-0.962268)
 cpMid := cp1 + cp3*(-0.162268)
 cpHi  := cp1 + cp3*0.837732
 
 abLoLo  := (a1 + a3*(-0.962268)) * (b1 + b3*(-0.4045))
 abLoMid := (a1 + a3*(-0.962268)) * (b1 + b3*(-0.0045))
 abLoHi  := (a1 + a3*(-0.962268)) * (b1 + b3*0.3955)
 
 abMidLo  := (a1 + a3*(-0.162268)) * (b1 + b3*(-0.4045))
 abMidMid := (a1 + a3*(-0.162268)) * (b1 + b3*(-0.0045))
 abMidHi  := (a1 + a3*(-0.162268)) * (b1 + b3*0.3955)
 
 abHiLo  := (a1 + a3*0.837732) * (b1 + b3*(-0.4045))
 abHiMid := (a1 + a3*0.837732) * (b1 + b3*(-0.0045))
 abHiHi  := (a1 + a3*0.837732) * (b1 + b3*0.3955)
 "
\end{Sinput}
\end{Schunk}


\pagebreak

\begin{Schunk}
\begin{Sinput}
 parameterEstimates(out2.2, boot = bootType)[ , -c(1 : 3)]
\end{Sinput}
\begin{Soutput}
   label    est    se      z pvalue ci.lower ci.upper
1     cp  0.135 0.083  1.638  0.102   -0.031    0.286
2     b2  0.597 0.137  4.359  0.000    0.311    0.858
3     a1 -0.266 0.061 -4.353  0.000   -0.392   -0.148
4    d21 -0.367 0.098 -3.764  0.000   -0.546   -0.166
5         0.987 0.166  5.946  0.000    0.731    1.402
6         0.719 0.116  6.218  0.000    0.527    0.991
7         0.705 0.094  7.520  0.000    0.552    0.926
8         2.444 0.000     NA     NA    2.444    2.444
9 fullIE  0.058 0.025  2.318  0.020    0.019    0.123
\end{Soutput}
\end{Schunk}


\end{frame}



\begin{frame}[allowframebreaks]{Double Mean Centering}
  
  Using the same problem setup as above, we could perform double mean
  centering by:\\ 
  \vb
  \begin{enumerate}
  \item Mean center every indicator of $X$ and $Z$:
    \begin{align*}
      x_1^c &= x_1 - \bar{x}_1\\
      &~~~\vdots\\
      z_1^c &= z_1 - \bar{z}_1\\
      &~~~\vdots
    \end{align*}
  \item Use the centered indicators to construct all possible product
    terms: $\{x_1^cz_1^c,$ $x_1^cz_2^c,$ $x_1^cz_3^c,$ $x_2^cz_1^c,$
    $x_2^cz_2^c,$ $x_2^cz_3^c,$ $x_3^cz_1^c,$ $x_3^cz_2^c,$ $x_3^cz_3^c\}$.
    
    \pagebreak
    
  \item Mean center each product term:
    \begin{align*}
      (x_1z_1)^c &= x_1^cz_1^c - \overline{x_1^cz_1^c}\\
      (x_1z_2)^c &= x_1^cz_2^c - \overline{x_1^cz_2^c}\\
      &~~~\vdots\\
      (x_3z_3)^c &= x_3^cz_3^c - \overline{x_3^cz_3^c}
    \end{align*}
    \vb
  \item Use the mean centered indicators of $X$ and $Z$, and the
    ``double mean centered'' product terms to specify the latent
    interaction model as represented in the following figure.
  \end{enumerate}
  
\end{frame}


\begin{frame}{Double Mean Centering}
  
  \begin{figure}
    \includegraphics[width=\textwidth]{figures/dmcDiagram.pdf}
  \end{figure}
  
\end{frame}


\begin{frame}[allowframebreaks]{Example}
  
\begin{Schunk}
\begin{Sinput}
 ## Test Differences between Indirect Effects
 ## in Serial Multiple Mediator Model (Method 1):
 mod2.3 <- "
 policy ~ cp*polAffil + b1*merit + b2*sysRac
 merit ~ a1*polAffil
 sysRac ~ a2*polAffil + d21*merit
 
 ab1 := a1*b1
 ab2 := a2*b2
 fullIE := a1*d21*b2
 totalIE := ab1 + ab2 + fullIE 
 
 fullIE == ab1
 fullIE == ab2
 "
 out2.3 <- 
     sem(mod2.3, data = dat1, se = "boot", boot = nBoot)
 summary(out2.3)
\end{Sinput}
\begin{Soutput}
lavaan (0.5-20) converged normally after 213 iterations

  Number of observations                            87

  Estimator                                         ML
  Minimum Function Test Statistic                1.334
  Degrees of freedom                                 2
  P-value (Chi-square)                           0.513

Parameter Estimates:

  Information                                 Observed
  Standard Errors                            Bootstrap
  Number of requested bootstrap draws             2500
  Number of successful bootstrap draws            2500

Regressions:
                   Estimate  Std.Err  Z-value  P(>|z|)
  policy ~                                            
    polAffil  (cp)    0.108    0.084    1.281    0.200
    merit     (b1)   -0.150    0.047   -3.183    0.001
    sysRac    (b2)    0.521    0.125    4.157    0.000
  merit ~                                             
    polAffil  (a1)   -0.271    0.057   -4.750    0.000
  sysRac ~                                            
    polAffil  (a2)    0.078    0.025    3.125    0.002
    merit    (d21)   -0.287    0.075   -3.814    0.000

Variances:
                   Estimate  Std.Err  Z-value  P(>|z|)
    policy            1.001    0.171    5.854    0.000
    merit             0.719    0.114    6.330    0.000
    sysRac            0.690    0.090    7.632    0.000

Defined Parameters:
                   Estimate  Std.Err  Z-value  P(>|z|)
    ab1               0.041    0.014    2.983    0.003
    ab2               0.041    0.014    2.983    0.003
    fullIE            0.041    0.014    2.983    0.003
    totalIE           0.122    0.041    2.983    0.003

Constraints:
                                               |Slack|
    fullIE - (ab1)                               0.000
    fullIE - (ab2)                               0.000
\end{Soutput}
\begin{Sinput}
 ## Conduct a chi-squared difference test:
 chiDiff <- fitMeasures(out2.3)["chisq"] - 
     fitMeasures(out2.1)["chisq"]
 dfDiff <- fitMeasures(out2.3)["df"] - 
     fitMeasures(out2.1)["df"]
 pchisq(chiDiff, dfDiff, lower = FALSE)
\end{Sinput}
\begin{Soutput}
    chisq 
0.5131246 
\end{Soutput}
\end{Schunk}


\pagebreak

\begin{Schunk}
\begin{Sinput}
 ## Serial Multiple Mediator Model with 3 Mediators:
 mod3.1 <- "
 policy ~ b1*merit + b2*sysRac + b3*revDisc + cp*polAffil
 revDisc ~ d31*merit + d32*sysRac + a3*polAffil
 sysRac ~ d21*merit + a2*polAffil
 merit ~ a1*polAffil
 
 ab1 := a1*b1
 ab2 := a2*b2
 ab3 := a3*b3
 
 partIE1 := a1*d31*b3
 partIE2 := a1*d21*b2
 partIE3 := a2*d32*b3
 
 fullIE := a1*d21*d32*b3
 
 totalIE := ab1 + ab2 + ab3 + partIE1 + partIE2 + partIE3 + fullIE 
 "
 out3.1 <- 
     sem(mod3.1, data = dat1, se = "boot", boot = nBoot)
 summary(out3.1)
\end{Sinput}
\begin{Soutput}
lavaan (0.5-20) converged normally after  23 iterations

  Number of observations                            87

  Estimator                                         ML
  Minimum Function Test Statistic                0.000
  Degrees of freedom                                 0

Parameter Estimates:

  Information                                 Observed
  Standard Errors                            Bootstrap
  Number of requested bootstrap draws             2500
  Number of successful bootstrap draws            2498

Regressions:
                   Estimate  Std.Err  Z-value  P(>|z|)
  policy ~                                            
    merit     (b1)    0.005    0.144    0.035    0.972
    sysRac    (b2)    0.589    0.151    3.895    0.000
    revDisc   (b3)   -0.026    0.080   -0.330    0.741
    polAffil  (cp)    0.130    0.080    1.616    0.106
  revDisc ~                                           
    merit    (d31)    0.473    0.190    2.490    0.013
    sysRac   (d32)   -0.196    0.243   -0.806    0.420
    polAffil  (a3)   -0.149    0.131   -1.140    0.254
  sysRac ~                                            
    merit    (d21)   -0.301    0.109   -2.765    0.006
    polAffil  (a2)    0.090    0.071    1.270    0.204
  merit ~                                             
    polAffil  (a1)   -0.266    0.061   -4.340    0.000

Variances:
                   Estimate  Std.Err  Z-value  P(>|z|)
    policy            0.985    0.164    6.023    0.000
    revDisc           2.361    0.307    7.698    0.000
    sysRac            0.689    0.091    7.612    0.000
    merit             0.719    0.111    6.482    0.000

Defined Parameters:
                   Estimate  Std.Err  Z-value  P(>|z|)
    ab1              -0.001    0.040   -0.033    0.973
    ab2               0.053    0.043    1.224    0.221
    ab3               0.004    0.016    0.244    0.807
    partIE1           0.003    0.012    0.273    0.785
    partIE2           0.047    0.026    1.831    0.067
    partIE3           0.000    0.003    0.150    0.881
    fullIE            0.000    0.002    0.191    0.849
    totalIE           0.107    0.052    2.052    0.040
\end{Soutput}
\end{Schunk}


\pagebreak

\begin{Schunk}
\begin{Sinput}
 parameterEstimates(out6, 
                    boot = "bca.simple")[-c(11 : 23), -c(1 : 3)]
\end{Sinput}
\begin{Soutput}
     label    est    se      z pvalue ci.lower ci.upper
1       cp  0.055 0.232  0.238  0.811   -0.431    0.474
2       b1  0.427 0.252  1.693  0.090   -0.068    0.929
3       b2  0.763 0.204  3.735  0.000    0.334    1.151
4       a2  0.473 0.073  6.480  0.000    0.329    0.625
5       d1  0.689 0.090  7.652  0.000    0.509    0.870
6       d2  0.863 0.110  7.843  0.000    0.670    1.116
7       d3  0.400 0.065  6.139  0.000    0.257    0.528
8       a1  0.720 0.089  8.110  0.000    0.521    0.883
9       a2  0.473 0.073  6.480  0.000    0.329    0.625
10      a3  0.476 0.098  4.871  0.000    0.296    0.674
24    imm1  0.203 0.128  1.580  0.114   -0.013    0.518
25     ab2  0.361 0.108  3.342  0.001    0.164    0.596
26 fullIE1  0.019 0.034  0.551  0.582   -0.020    0.134
27 fullIE2  1.295 0.387  3.344  0.001    0.562    2.102
28  mmTest -1.276 0.388 -3.291  0.001   -2.082   -0.547
\end{Soutput}
\end{Schunk}


\end{frame}


\begin{frame}[allowframebreaks]{Example}
  
\begin{Schunk}
\begin{Sinput}
 par(family = "serif", cex = 0.75)
 library(rockchalk)
 ## First we need to create a 'plotSlopes' object:
 plotOut <- plotSlopes(model = out1,
                       plotx = "ratioC",
                       modx = "perceptionC",
                       plotPoints = FALSE)
 ## Then we modify 'plotOut' to get the J-N test:
 testOut <- testSlopes(plotOut)
\end{Sinput}
\begin{Soutput}
Values of perceptionC OUTSIDE this interval:
      lo       hi 
1.327479 2.452667 
cause the slope of (b1 + b2*perceptionC)ratioC to be statistically significant
\end{Soutput}
\end{Schunk}


\pagebreak

\begin{Schunk}
\begin{Sinput}
 ## Construct product terms to facilitate J-N technique:
 dat1$openXneuro <- with(dat1, neuro*open)
 dat1$concXneuro <- with(dat1, neuro*conc)
 dat1$openXconc <- with(dat1, open*conc)
 dat1$openXconcXneuro <- with(dat1, open*conc*neuro)
\end{Sinput}
\end{Schunk}


\end{frame}


\begin{frame}{Orthogonalization vs. Double Mean Centering}
  
  Orthogonalization and double mean centering tend to behave
  comparably, but each has its own strengths: 
  \vb
  \begin{itemize}
    \item When $X$ and $Z$ are bivariate normally distributed, both
      methods produce the same results.
      \vb
    \item As $X$ and/or $Z$ stray from normality, orthogonalization
      produces biased estimates of the interaction effect, but double
      mean centering does not.
      \vb
    \item Orthogonalization ensures that the latent $XZ$ is perfectly
      independent of $X$ and $Z$.
      \vc
      \begin{itemize}
        \item The $X$ and $Z$ parameters can be directly interpreted,
          without any conditioning
      \end{itemize}
  \end{itemize}
  
\end{frame}



\begin{frame}[allowframebreaks]{Example}
  
\begin{Schunk}
\begin{Sinput}
 ## Calculate the percentile CI:
 lb <- sort(abVec)[0.025 * nSams]
 ub <- sort(abVec)[0.975 * nSams]
 c(lb, ub)
\end{Sinput}
\begin{Soutput}
[1] 0.08845936 0.29432389
\end{Soutput}
\end{Schunk}


\end{frame}


\begin{frame}{Other Things}
  
  Moderation in logistic regression:
  \vb
  \begin{itemize}
    \item Nothing special
      \vc
    \item Just include the product term as a predictor
      \vc
    \item Make sure to keep track of the weird ``multiplicative change
      in log-odds'' interpretation of your coefficients
  \end{itemize}
  \va
  \pause
  Effect size for conditional process analysis:
  \vb
  \begin{itemize}
    \item We don't know
      \vc
    \item I could not find any work directly addressing the issue
      \vc
    \item Fully and partially standardized indirect effects seem like
      they should still work
      \vc
    \item $\kappa^2$ and the various flavors of $R^2$ aren't so
      clear-cut.
  \end{itemize}
  
\end{frame}



\begin{frame}{References}

  \bibliographystyle{apacite}
  \bibliography{../../bibtexStuff/lecture12Refs.bib}

\end{frame}


\end{document}
