\documentclass{beamer}
\usetheme{ttuStatsCamp}
\usefonttheme{serif}
\usepackage[T1]{fontenc}
\usepackage[utf8]{inputenc}
\usepackage{url}
\usepackage{graphicx}
\usepackage{setspace}
\usepackage[natbibapa]{apacite}
\usepackage{color}
\usepackage{amsmath}
\usepackage{amsfonts}
\usepackage{Sweavel}
\usepackage{listings}
\usepackage{fancybox}

\def\Sweavesize{\scriptsize}
\def\Rcolor{\color{black}}
%\def\Routcolor{\color{red}}
\def\Rcommentcolor{\color{violet}}
\def\Rbackground{\color[gray]{0.85}}
\def\Routbackground{\color[gray]{0.85}}

\lstset{tabsize=2, breaklines=true, style=Rstyle}



\newcommand{\red}[0]{\textcolor{red}}
\newcommand{\green}[0]{\textcolor{green}}
\newcommand{\blue}[0]{\textcolor{blue}}
\newcommand{\comment}[1]{}
\newcommand{\va}[0]{\vspace{12pt}}
\newcommand{\vb}[0]{\vspace{6pt}}
\newcommand{\vc}[0]{\vspace{3pt}}
\newcommand{\vx}[1]{\vspace{#1pt}}

\title[Lecture 8]{Lecture 8: Probing Moderation}

\author{Kyle M. Lang}

\institute[TTU IMMAP]{
  Institute for Measurement, Methodology, Analysis \& Policy\\
  Texas Tech University\\
  Lubbock, TX
}

\date{2016 Stats Camp}

\setbeamertemplate{frametitle continuation}{}

\begin{document}

\setkeys{Gin}{width=\textwidth}

\input{sweaveFiles/-001}



\begin{frame}[plain]
  
  \titlepage
  
\end{frame}



\begin{frame}{Outline}

  \begin{itemize}
  \item Probing moderation via centering
    \va
  \item Alternative probing strategies
    \va
  \item Confidence bands for simple slopes
  \end{itemize}

\end{frame}



\begin{frame}{Starting Point}

  Last time, we fit the following model:

  \begin{figure}
    \includegraphics[width=\textwidth]{figures/modExample1.pdf}
  \end{figure}

\end{frame}



\begin{frame}{Interaction Probing}
  
  We probed the interaction with the \emph{Pick-a-point} approach.
  \va
  \begin{itemize}
    \item Choose interesting values of the moderator $Z$
      \vb
    \item Check the significance of the focal effect $X \rightarrow Y$
      at the values we choose for $Z$.
      \vb
    \item Gives us an idea of where in $Z$'s distribution the focal
      effect is/is not significant.
  \end{itemize}
  \va
  Previously, we manually calculated the all of the quantities we
  needed, including a SE for the conditional focal effect.
  \vb
  \begin{itemize}
    \item There is a simpler way: \textsc{Centering}
  \end{itemize}
  
\end{frame}



\begin{frame}{Centering}
  
  Centering transforms a variable by subtracting a constant (e.g., the
  variable's mean) from each observation of the variable
  \va
  \begin{itemize}
  \item The most familiar form of center is \emph{mean centering}
    \vb
  \item We can center on any value
    \vb
    \begin{itemize}
    \item When probing interactions, we can center $Z$ on the
      interesting values we choose during the pick-a-point approach
      \vb
    \item Running the model with $Z$ centered on specific values
      automatically provides tests of the simple slope conditional
      on those values of $Z$
    \end{itemize}
  \end{itemize}
  
\end{frame}



\begin{frame}{Probing via Centering}
  
  Say we want to do a simple slopes analysis to test the conditional
  effect of $X$ on $Y$ at three levels of $Z = \{Z_1, Z_2,
  Z_3\}$.\\ 
  \va 
  Then, all we need to do is fit the following three
  models:
  \begin{align*}
    Y = \alpha + \beta_1X + \beta_2(Z - Z_1) + \beta_3 X(Z - Z_1) + e\\
    \\
    Y = \alpha + \beta_1X + \beta_2(Z - Z_2) + \beta_3 X(Z - Z_2) + e\\
    \\
    Y = \alpha + \beta_1X + \beta_2(Z - Z_3) + \beta_3 X(Z - Z_3) + e
  \end{align*}
  The default output for $\beta_1$ provides tests of the simple
  slopes.
  
\end{frame}



\begin{frame}[allowframebreaks]{Example}

\begin{Schunk}
\begin{Sinput}
 dat1 <- readRDS("../data/adamsKlpsScaleScore.rds")
 ## Partial out the mediator's effect:
 mod1 <- lm(policy ~ sysRac + polAffil, data = dat1)
 mod2 <- lm(sysRac ~ polAffil, data = dat1)
 summary(mod1)$coef
\end{Sinput}
\begin{Soutput}
              Estimate Std. Error   t value     Pr(>|t|)
(Intercept) 0.83265885 0.41246491 2.0187386 4.670428e-02
sysRac      0.72235878 0.11147514 6.4799987 5.930291e-09
polAffil    0.05121251 0.06998433 0.7317711 4.663450e-01
\end{Soutput}
\begin{Sinput}
 summary(mod2)$coef
\end{Sinput}
\begin{Soutput}
             Estimate Std. Error  t value     Pr(>|t|)
(Intercept) 2.6060451 0.28489391 9.147423 2.715546e-14
polAffil    0.2568494 0.06213495 4.133735 8.336022e-05
\end{Soutput}
\begin{Sinput}
 ## Extract important parameter estimates:
 a <- coef(mod2)["polAffil"]
 b <- coef(mod1)["sysRac"]
 ## Compute indirect effect:
 ieProd <- a * b
 ieProd
\end{Sinput}
\begin{Soutput}
 polAffil 
0.1855374 
\end{Soutput}
\begin{Sinput}
 ## Calculate Sobel's Z:
 seA <- sqrt(diag(vcov(mod2)))["polAffil"]
 seB <- sqrt(diag(vcov(mod1)))["sysRac"]
 sobelSE <- sqrt(b^2 * seA^2 + a^2 * seB^2)
 sobelZ <- ieProd / sobelSE
 sobelZ
\end{Sinput}
\begin{Soutput}
polAffil 
 3.48501 
\end{Soutput}
\begin{Sinput}
 sobelP <- 2 * pnorm(sobelZ, lower = FALSE)
 sobelP
\end{Sinput}
\begin{Soutput}
    polAffil 
0.0004921178 
\end{Soutput}
\begin{Sinput}
 sobelUB <- ieProd + 1.96 * sobelSE
 sobelLB <- ieProd - 1.96 * sobelSE
 ## 95% Sobel CI:
 c(sobelLB, sobelUB)
\end{Sinput}
\begin{Soutput}
  polAffil   polAffil 
0.08118957 0.28988525 
\end{Soutput}
\end{Schunk}


\end{frame}


\begin{frame}{Compare Approaches}
  
  The manual and the centering approaches give identical answers,
  barring rounding errors with the manual approach: 
  \va
\begin{Schunk}
\begin{Sinput}
 parameterEstimates(out1, 
                    boot = "bca.simple")[-c(6 : 13), -c(1 : 3)]
\end{Sinput}
\begin{Soutput}
   label   est    se     z pvalue ci.lower ci.upper
1     cp 0.082 0.259 0.318  0.751   -0.408    0.641
2      b 1.390 0.215 6.465  0.000    0.939    1.786
3     a1 0.729 0.091 8.014  0.000    0.539    0.908
4     a2 0.641 0.089 7.183  0.000    0.469    0.818
5     a3 0.451 0.097 4.643  0.000    0.223    0.629
14   imm 0.628 0.169 3.706  0.000    0.296    0.980
\end{Soutput}
\end{Schunk}


\end{frame}



\begin{frame}[allowframebreaks]{A Few Comments on Centering}
  
  You will often hear mean centering touted as absolutely necessary or
  absolutely unnecessary for moderation analysis.\\ 
  \va 
  Both sides are partially correct.\\ 
  \va 
  Two effects are usually ascribed to mean
  centering in moderation analysis: 
  \vb
  \begin{enumerate}
    \item Improved interpretation of the conditional effects 
      \vb
    \item Reduced multicollinearity between lower-order effects and
      the interaction term
  \end{enumerate}
  
  \pagebreak
  
  Mean center absolutely \emph{does} have the potential to improve
  parameter interpretations
  \vb
  \begin{itemize}
  \item When $X = 0$ or $Z = 0$ are not sensible values, centering is
    necessary for any plausible interpretation of $\beta_1$ or
    $\beta_2$.
  \end{itemize}
  \va 
  Mean centering \emph{can} remove collinearity between lower-order
  terms and the interaction term
  \vb
  \begin{itemize}
    \item \textbf{BUT}, we don't care
  \end{itemize}
  \va 
  We can get a better sense of what's going on with a synthetic
  example.
\end{frame}


\begin{frame}[allowframebreaks]{Example}
    
\begin{Schunk}
\begin{Sinput}
 mod3 <- "
 att3 ~ att2 + b2*conf2 + cp2*horn2
 att2 ~ att1 + b1*conf1 + cp1*horn1
 
 conf3 ~ conf2 + a2*horn2
 conf2 ~ conf1 + a1*horn1
 
 horn3 ~ horn2
 horn2 ~ horn1
 
 horn3 ~~ conf3 + att3
 conf3 ~~ att3
 
 horn2 ~~ conf2 + att2
 conf2 ~~ att2
 
 a1 == a2
 b1 == b2
 cp1 == cp2
 "
 out3 <- sem(mod3, data = dat1)
 summary(out3)
\end{Sinput}
\begin{Soutput}
lavaan (0.5-20) converged normally after  46 iterations

  Number of observations                           500

  Estimator                                         ML
  Minimum Function Test Statistic              294.220
  Degrees of freedom                                18
  P-value (Chi-square)                           0.000

Parameter Estimates:

  Information                                 Expected
  Standard Errors                             Standard

Regressions:
                   Estimate  Std.Err  Z-value  P(>|z|)
  att3 ~                                              
    att2              0.497    0.035   14.234    0.000
    conf2     (b2)    0.098    0.019    5.200    0.000
    horn2    (cp2)    0.083    0.072    1.157    0.247
  att2 ~                                              
    att1              0.530    0.040   13.345    0.000
    conf1     (b1)    0.098    0.019    5.200    0.000
    horn1    (cp1)    0.083    0.072    1.157    0.247
  conf3 ~                                             
    conf2             0.684    0.035   19.602    0.000
    horn2     (a2)    0.493    0.107    4.596    0.000
  conf2 ~                                             
    conf1             0.623    0.032   19.546    0.000
    horn1     (a1)    0.493    0.107    4.596    0.000
  horn3 ~                                             
    horn2             0.826    0.030   27.609    0.000
  horn2 ~                                             
    horn1             0.714    0.024   29.181    0.000

Covariances:
                   Estimate  Std.Err  Z-value  P(>|z|)
  conf3 ~~                                            
    horn3             1.016    0.155    6.556    0.000
  att3 ~~                                             
    horn3             0.322    0.093    3.483    0.000
    conf3             3.574    0.465    7.691    0.000
  conf2 ~~                                            
    horn2             0.836    0.124    6.721    0.000
  att2 ~~                                             
    horn2             0.273    0.083    3.289    0.001
    conf2             2.027    0.400    5.067    0.000

Variances:
                   Estimate  Std.Err  Z-value  P(>|z|)
    att3              6.019    0.381   15.811    0.000
    att2              6.041    0.382   15.811    0.000
    conf3            15.814    1.000   15.811    0.000
    conf2            12.570    0.795   15.811    0.000
    horn3             0.695    0.044   15.811    0.000
    horn2             0.560    0.035   15.811    0.000

Constraints:
                                               |Slack|
    a1 - (a2)                                    0.000
    b1 - (b2)                                    0.000
    cp1 - (cp2)                                  0.000
\end{Soutput}
\begin{Sinput}
 chiDiff <- fitMeasures(out3)["chisq"] -
     fitMeasures(out1)["chisq"]
 dfDiff <- fitMeasures(out3)["df"] -
     fitMeasures(out1)["df"]
 pchisq(chiDiff, dfDiff, lower = FALSE)
\end{Sinput}
\begin{Soutput}
     chisq 
0.02684148 
\end{Soutput}
\end{Schunk}


\pagebreak

\begin{Schunk}
\begin{Sinput}
 parameterEstimates(out2, 
                    boot = "bca.simple")[-c(7 : 18), -c(1 : 3)]
\end{Sinput}
\begin{Soutput}
   label    est    se      z pvalue ci.lower ci.upper
1    cp1 -0.103 0.190 -0.543  0.587   -0.473    0.259
2    cp2  1.099 0.204  5.380  0.000    0.639    1.472
3     b1  1.615 0.167  9.649  0.000    1.270    1.912
4     b2  0.381 0.173  2.206  0.027    0.034    0.720
5     b3  0.571 0.173  3.297  0.001    0.220    0.921
6      a  0.741 0.131  5.638  0.000    0.471    0.984
19   imm  0.424 0.153  2.773  0.006    0.174    0.781
\end{Soutput}
\end{Schunk}

\includegraphics{sweaveFiles/-005}

\end{frame}


\begin{frame}[shrink = 10]{Example}
  
\begin{Schunk}
\begin{Sinput}
 ## Completely Standardized:
 abCS <- (sdX * ab) / sdY
 abCS
\end{Sinput}
\begin{Soutput}
[1] 0.1345859
\end{Soutput}
\begin{Sinput}
 cPrimeCS <- (sdX * cPrime) / sdY
 cPrimeCS
\end{Sinput}
\begin{Soutput}
       cp 
0.1790413 
\end{Soutput}
\begin{Sinput}
 cCS <- abCS + cPrimeCS
 cCS
\end{Sinput}
\begin{Soutput}
       cp 
0.3136272 
\end{Soutput}
\end{Schunk}


\end{frame}


\begin{frame}[allowframebreaks]{Example}
  
\begin{Schunk}
\begin{Sinput}
 sum(out3.1$fitted - out3.3$fitted)
\end{Sinput}
\begin{Soutput}
[1] -2.785328e-12
\end{Soutput}
\begin{Sinput}
 summary(out3.1)$r.squared
\end{Sinput}
\begin{Soutput}
[1] 0.9999439
\end{Soutput}
\begin{Sinput}
 summary(out3.3)$r.squared
\end{Sinput}
\begin{Soutput}
[1] 0.9999439
\end{Soutput}
\end{Schunk}


\pagebreak

\begin{Schunk}
\begin{Sinput}
 nSams <- 1000
 abVec <- rep(NA, nSams)
 for(i in 1 : nSams) {
     ## Resample the data:
     bootSam <- 
         dat1[sample(c(1 : nrow(dat1)), replace = TRUE), ]
     ## Fit the path model:
     bootOut <- sem(mod2, data = bootSam)
     ## Store the estimated indirect effect:
     abVec[i] <- prod(coef(bootOut)[c("a", "b")])
 }
\end{Sinput}
\end{Schunk}


\pagebreak

\begin{Schunk}
\begin{Sinput}
 ## Conditional process model with a, b, c paths moderated:
 mod4 <- "
 agree ~ b1*open + b2*consc + cp1*extra + cp2*neuro + 
         cp3*extraXneuro + b3*openXconsc
 open ~ a1*extra + a2*neuro + a3*extraXneuro
 
 cpLo  := cp1 + cp3*(-0.962268)
 cpMid := cp1 + cp3*(-0.162268)
 cpHi  := cp1 + cp3*0.837732
 
 abLoLo  := (a1 + a3*(-0.962268)) * (b1 + b3*(-0.4045))
 abLoMid := (a1 + a3*(-0.962268)) * (b1 + b3*(-0.0045))
 abLoHi  := (a1 + a3*(-0.962268)) * (b1 + b3*0.3955)
 
 abMidLo  := (a1 + a3*(-0.162268)) * (b1 + b3*(-0.4045))
 abMidMid := (a1 + a3*(-0.162268)) * (b1 + b3*(-0.0045))
 abMidHi  := (a1 + a3*(-0.162268)) * (b1 + b3*0.3955)
 
 abHiLo  := (a1 + a3*0.837732) * (b1 + b3*(-0.4045))
 abHiMid := (a1 + a3*0.837732) * (b1 + b3*(-0.0045))
 abHiHi  := (a1 + a3*0.837732) * (b1 + b3*0.3955)
 "
\end{Sinput}
\end{Schunk}


\pagebreak

\begin{Schunk}
\begin{Sinput}
 parameterEstimates(out2.2, boot = bootType)[ , -c(1 : 3)]
\end{Sinput}
\begin{Soutput}
   label    est    se      z pvalue ci.lower ci.upper
1     cp  0.135 0.083  1.638  0.102   -0.031    0.286
2     b2  0.597 0.137  4.359  0.000    0.311    0.858
3     a1 -0.266 0.061 -4.353  0.000   -0.392   -0.148
4    d21 -0.367 0.098 -3.764  0.000   -0.546   -0.166
5         0.987 0.166  5.946  0.000    0.731    1.402
6         0.719 0.116  6.218  0.000    0.527    0.991
7         0.705 0.094  7.520  0.000    0.552    0.926
8         2.444 0.000     NA     NA    2.444    2.444
9 fullIE  0.058 0.025  2.318  0.020    0.019    0.123
\end{Soutput}
\end{Schunk}


\pagebreak

\begin{Schunk}
\begin{Sinput}
 ## Test Differences between Indirect Effects
 ## in Serial Multiple Mediator Model (Method 1):
 mod2.3 <- "
 policy ~ cp*polAffil + b1*merit + b2*sysRac
 merit ~ a1*polAffil
 sysRac ~ a2*polAffil + d21*merit
 
 ab1 := a1*b1
 ab2 := a2*b2
 fullIE := a1*d21*b2
 totalIE := ab1 + ab2 + fullIE 
 
 fullIE == ab1
 fullIE == ab2
 "
 out2.3 <- 
     sem(mod2.3, data = dat1, se = "boot", boot = nBoot)
 summary(out2.3)
\end{Sinput}
\begin{Soutput}
lavaan (0.5-20) converged normally after 213 iterations

  Number of observations                            87

  Estimator                                         ML
  Minimum Function Test Statistic                1.334
  Degrees of freedom                                 2
  P-value (Chi-square)                           0.513

Parameter Estimates:

  Information                                 Observed
  Standard Errors                            Bootstrap
  Number of requested bootstrap draws             2500
  Number of successful bootstrap draws            2500

Regressions:
                   Estimate  Std.Err  Z-value  P(>|z|)
  policy ~                                            
    polAffil  (cp)    0.108    0.084    1.281    0.200
    merit     (b1)   -0.150    0.047   -3.183    0.001
    sysRac    (b2)    0.521    0.125    4.157    0.000
  merit ~                                             
    polAffil  (a1)   -0.271    0.057   -4.750    0.000
  sysRac ~                                            
    polAffil  (a2)    0.078    0.025    3.125    0.002
    merit    (d21)   -0.287    0.075   -3.814    0.000

Variances:
                   Estimate  Std.Err  Z-value  P(>|z|)
    policy            1.001    0.171    5.854    0.000
    merit             0.719    0.114    6.330    0.000
    sysRac            0.690    0.090    7.632    0.000

Defined Parameters:
                   Estimate  Std.Err  Z-value  P(>|z|)
    ab1               0.041    0.014    2.983    0.003
    ab2               0.041    0.014    2.983    0.003
    fullIE            0.041    0.014    2.983    0.003
    totalIE           0.122    0.041    2.983    0.003

Constraints:
                                               |Slack|
    fullIE - (ab1)                               0.000
    fullIE - (ab2)                               0.000
\end{Soutput}
\begin{Sinput}
 ## Conduct a chi-squared difference test:
 chiDiff <- fitMeasures(out2.3)["chisq"] - 
     fitMeasures(out2.1)["chisq"]
 dfDiff <- fitMeasures(out2.3)["df"] - 
     fitMeasures(out2.1)["df"]
 pchisq(chiDiff, dfDiff, lower = FALSE)
\end{Sinput}
\begin{Soutput}
    chisq 
0.5131246 
\end{Soutput}
\end{Schunk}


\end{frame}



\begin{frame}{Back to Work}
  
  \textsc{Question:} Okay, so what about our example analysis? Should
  we center the predictors in this model:
  \begin{align*}
    Depress = \alpha + \beta_1Ratio + \beta_2Perception + \beta_3 Ratio \times Perception + e
  \end{align*}

  \pause
  
  \textsc{Answer:} Yes.\\\va
  
  \pause
  
  \textsc{Question Mark II:} Why?\\\va
  
  \pause 
  
  \textsc{Answer the Second:} Because a \emph{Weight:Height} ratio of zero
  is nonsensical and zero is outside the range of \emph{Perception}.
  
\end{frame}


\begin{frame}[shrink = 10]{Example}

\begin{Schunk}
\begin{Sinput}
 ## Serial Multiple Mediator Model with 3 Mediators:
 mod3.1 <- "
 policy ~ b1*merit + b2*sysRac + b3*revDisc + cp*polAffil
 revDisc ~ d31*merit + d32*sysRac + a3*polAffil
 sysRac ~ d21*merit + a2*polAffil
 merit ~ a1*polAffil
 
 ab1 := a1*b1
 ab2 := a2*b2
 ab3 := a3*b3
 
 partIE1 := a1*d31*b3
 partIE2 := a1*d21*b2
 partIE3 := a2*d32*b3
 
 fullIE := a1*d21*d32*b3
 
 totalIE := ab1 + ab2 + ab3 + partIE1 + partIE2 + partIE3 + fullIE 
 "
 out3.1 <- 
     sem(mod3.1, data = dat1, se = "boot", boot = nBoot)
 summary(out3.1)
\end{Sinput}
\begin{Soutput}
lavaan (0.5-20) converged normally after  23 iterations

  Number of observations                            87

  Estimator                                         ML
  Minimum Function Test Statistic                0.000
  Degrees of freedom                                 0

Parameter Estimates:

  Information                                 Observed
  Standard Errors                            Bootstrap
  Number of requested bootstrap draws             2500
  Number of successful bootstrap draws            2498

Regressions:
                   Estimate  Std.Err  Z-value  P(>|z|)
  policy ~                                            
    merit     (b1)    0.005    0.144    0.035    0.972
    sysRac    (b2)    0.589    0.151    3.895    0.000
    revDisc   (b3)   -0.026    0.080   -0.330    0.741
    polAffil  (cp)    0.130    0.080    1.616    0.106
  revDisc ~                                           
    merit    (d31)    0.473    0.190    2.490    0.013
    sysRac   (d32)   -0.196    0.243   -0.806    0.420
    polAffil  (a3)   -0.149    0.131   -1.140    0.254
  sysRac ~                                            
    merit    (d21)   -0.301    0.109   -2.765    0.006
    polAffil  (a2)    0.090    0.071    1.270    0.204
  merit ~                                             
    polAffil  (a1)   -0.266    0.061   -4.340    0.000

Variances:
                   Estimate  Std.Err  Z-value  P(>|z|)
    policy            0.985    0.164    6.023    0.000
    revDisc           2.361    0.307    7.698    0.000
    sysRac            0.689    0.091    7.612    0.000
    merit             0.719    0.111    6.482    0.000

Defined Parameters:
                   Estimate  Std.Err  Z-value  P(>|z|)
    ab1              -0.001    0.040   -0.033    0.973
    ab2               0.053    0.043    1.224    0.221
    ab3               0.004    0.016    0.244    0.807
    partIE1           0.003    0.012    0.273    0.785
    partIE2           0.047    0.026    1.831    0.067
    partIE3           0.000    0.003    0.150    0.881
    fullIE            0.000    0.002    0.191    0.849
    totalIE           0.107    0.052    2.052    0.040
\end{Soutput}
\end{Schunk}


\end{frame}


\begin{frame}[shrink = 10]{Example}
  
\begin{Schunk}
\begin{Sinput}
 parameterEstimates(out6, 
                    boot = "bca.simple")[-c(11 : 23), -c(1 : 3)]
\end{Sinput}
\begin{Soutput}
     label    est    se      z pvalue ci.lower ci.upper
1       cp  0.055 0.232  0.238  0.811   -0.431    0.474
2       b1  0.427 0.252  1.693  0.090   -0.068    0.929
3       b2  0.763 0.204  3.735  0.000    0.334    1.151
4       a2  0.473 0.073  6.480  0.000    0.329    0.625
5       d1  0.689 0.090  7.652  0.000    0.509    0.870
6       d2  0.863 0.110  7.843  0.000    0.670    1.116
7       d3  0.400 0.065  6.139  0.000    0.257    0.528
8       a1  0.720 0.089  8.110  0.000    0.521    0.883
9       a2  0.473 0.073  6.480  0.000    0.329    0.625
10      a3  0.476 0.098  4.871  0.000    0.296    0.674
24    imm1  0.203 0.128  1.580  0.114   -0.013    0.518
25     ab2  0.361 0.108  3.342  0.001    0.164    0.596
26 fullIE1  0.019 0.034  0.551  0.582   -0.020    0.134
27 fullIE2  1.295 0.387  3.344  0.001    0.562    2.102
28  mmTest -1.276 0.388 -3.291  0.001   -2.082   -0.547
\end{Soutput}
\end{Schunk}


\end{frame}



\begin{frame}{Alternative Probing Strategies}
  
  The pick-a-point approach is nice due to its simplicity and ease of
  interpretation, but the points we choose are totally arbitrary.
  \vb
  \begin{itemize}
    \item We may be missing important nuances that occur in some of
      the areas of $Z$'s distribution that we \emph{did not} pick.
  \end{itemize}
  \va
  \pause
  The \emph{Johnson-Neyman} technique is an alternative approach that
  removes the arbitrary choices necessary for pick-a-point.
  \vb
  \begin{itemize}
    \item Johnson-Neyman finds the \emph{region of significance}
      wherein the conditional effect of $X$ on $Y$ is statistically
      significant
      \vb
    \item Inverts the pick-a-point approach to find what cut-points on
      the moderator correspond to a critical $t$ value for the
      conditional $\beta_1$.
  \end{itemize}

\end{frame}


\begin{frame}{Johnson-Neyman Technique}
  
  With pick-a-point, we:
  \vb
  \begin{enumerate}
    \item Choose conditional values of $Z$, say $Z_1$
      \vb
    \item Calculate the simple slope $SS_1$ and standard error
      $SE_{SS1}$ associated with $Z_1$
      \vb
    \item Test $SS_1$ for significance via a simple Wald-type test:
      \begin{align}
        t = \frac{SS_1}{SE_{SS1}} \label{waldEq}
      \end{align}
  \end{enumerate}

\end{frame}


\begin{frame}{Johnson-Neyman Technique}
  
  With Johnson-Neyman, we:
  \vb
  \begin{enumerate}
  \item Choose an $\alpha$ level for our test and the corresponding
    critical value of $t$, say $t_{crit} = 1.96$ to give $\alpha =
    0.05$ in large samples.
    \vb
  \item Re-arrange Equation \ref{waldEq} into the following quadratic
    form:
    \begin{align}
      t_{crit}^2 SE_{SS}^2 - SS^2 = 0 \label{quadEq}
    \end{align}
    \vc
  \item Solve Equation \ref{quadEq} to find the two values of $Z$ that
    produce critical $t$ statistics for the conditional focal effect.
  \end{enumerate}
  
\end{frame}


\begin{frame}{Johnson-Neyman Technique}
  
  The roots produced by the Johnson-Neyman technique delineate the
  \emph{region of significance}.  
  \vb
  \begin{itemize}
  \item The conditional effect of $X$ on $Y$ is either significant
    everywhere between these two points or everywhere outside of these
    two points.  
    \vb
  \item If only one of the points falls within the observed range of
    $Z$, ignore the other point 
    \vc
    \begin{itemize}
    \item The region of significant is either everywhere above or
      below the legal root
    \end{itemize}
    \vb
  \item If neither of the roots fall within the observed range of $Z$
    then, either: 
    \vc
    \begin{enumerate}
    \item The focal effect is significant across the entire range of
      $Z$, or 
      \vc
    \item The focal effect is not significant anywhere within the
      range of $Z$
    \end{enumerate}
  \end{itemize}
  
\end{frame}


\begin{frame}{Perspectives on Simple Slopes}
  
  Recall the formula for a simple slope:
  \begin{align*}
    SS = \beta_1 + \beta_3Z
  \end{align*}
  \vc From a graphical perspective, we can think about $SS$ in, at
  least, two different ways:
  \vb
  \begin{enumerate}
  \item As a weight for $X$ that we can use to get plots of the
    conditional effect of $X$ on $Y$ at different levels of $Z$.
    \vb
  \item We can also consider how $SS$, itself, smoothly changes as a
    function of $Z$.  
  \end{enumerate}
  \va
  The latter perspective embodies the spirit of the Johnson-Neyman
  technique.
  
\end{frame}


\begin{frame}{Confidence Bands}
  
  A natural quantity to consider is a confidence interval for $SS$:
  \begin{align*}
    CI_{SS} = SS \pm t_{crit} \cdot SE_{SS}
  \end{align*}
  \vc 
  Last time, we computed a few such intervals for the interesting
  values of $Z$ we chose for the pick-a-point analysis.\\ 
  \va 
  When doing Johnson-Neyman, we can considering the values of $CI_{SS}$
  for the entire range of $Z$.  
  \vb
  \begin{itemize}
  \item These CI values define the \emph{confidence bands} of SS and
    show, for any value of $Z$, the corresponding CI for $SS$
    \vb
    \begin{itemize}
    \item As a result, we can immediately check any value of $Z$ for
      a significant simple slope
    \end{itemize}
  \end{itemize}
  
\end{frame}

  
\begin{frame}[allowframebreaks]{Example}
  
  Implementing the Johnson-Neyman technique by hand is a pain, but we
  can easily do so by using the \textbf{rockchalk} package in
  \textsf{R}\\ 
  \va
  
\begin{Schunk}
\begin{Sinput}
 par(family = "serif", cex = 0.75)
 library(rockchalk)
 ## First we need to create a 'plotSlopes' object:
 plotOut <- plotSlopes(model = out1,
                       plotx = "ratioC",
                       modx = "perceptionC",
                       plotPoints = FALSE)
 ## Then we modify 'plotOut' to get the J-N test:
 testOut <- testSlopes(plotOut)
\end{Sinput}
\begin{Soutput}
Values of perceptionC OUTSIDE this interval:
      lo       hi 
1.327479 2.452667 
cause the slope of (b1 + b2*perceptionC)ratioC to be statistically significant
\end{Soutput}
\end{Schunk}

\includegraphics{sweaveFiles/-014}
\pagebreak

We can see the significance boundaries by extracting the roots from
'testOut'
\begin{Schunk}
\begin{Sinput}
 ## Construct product terms to facilitate J-N technique:
 dat1$openXneuro <- with(dat1, neuro*open)
 dat1$concXneuro <- with(dat1, neuro*conc)
 dat1$openXconc <- with(dat1, open*conc)
 dat1$openXconcXneuro <- with(dat1, open*conc*neuro)
\end{Sinput}
\end{Schunk}

\va
We can plot the result:
\begin{Schunk}
\begin{Sinput}
 ## Calculate the percentile CI:
 lb <- sort(abVec)[0.025 * nSams]
 ub <- sort(abVec)[0.975 * nSams]
 c(lb, ub)
\end{Sinput}
\begin{Soutput}
[1] 0.08845936 0.29432389
\end{Soutput}
\end{Schunk}

\includegraphics{sweaveFiles/-016}

\end{frame}


\end{document}
